\documentclass[12pt,usenames,dvipsnames,a4paper]{book}
\usepackage[T1]{fontenc}
\usepackage[utf8]{inputenc}
\usepackage[spanish, es-tabla]{babel}
\usepackage{mathtools}
\usepackage{amsthm}
\usepackage{amsmath}
\usepackage{amsfonts}
\usepackage{amssymb}
\usepackage{makeidx}
\usepackage{graphicx}
\usepackage{xcolor}
\usepackage{wrapfig}
\usepackage{lmodern}
\usepackage{marginnote}
\usepackage{fixltx2e}
\usepackage{cancel}
\usepackage{tikz}
\usepackage{pgfkeys} % LATEX
\usepackage{pgffor}
\usepackage{empheq}
\usepackage{caption}
\usepackage{subcaption}
\usepackage{xfrac}
\usepackage{fancybox}
\usepackage{pgfplots}
\pgfplotsset{compat=newest}
\usepackage{bm}
\usepackage{siunitx}
\usepackage{empheq}
\usepackage{hyperref}
\usepackage{xfrac}
\usepackage[italicdiff]{physics}
\usepackage[bottom=4cm, top = 4cm]{geometry}

\usetikzlibrary{babel,arrows.meta,decorations}

\usepgfplotslibrary{units, fillbetween}
\setlength\intextsep{0pt}

\renewcommand{\theequation}{\thepart.\arabic{equation}}


%------------------------------------------------------------------------------

%                             Encabezado y pie de página

%------------------------------------------------------------------------------

\usepackage{fancyhdr}

% aqui definimos el encabezado de las paginas pares e impares.
\lhead[]{\thepage}
\chead[]{}
\rhead[\thepage]{}
\lfoot[]{}
\cfoot[]{}
\rfoot[]{}
\renewcommand{\footrulewidth}{0pt}
\renewcommand{\headrulewidth}{0pt}

% aqui definimos el encabezado y pie de pagina de la pagina inicial de un capitulo.
\fancypagestyle{plain}{
	\fancyhead[L]{}
	\fancyhead[C]{}
	\fancyhead[R]{}
	\fancyfoot[L]{}
	\fancyfoot[C]{}
	\fancyfoot[R]{}
	\renewcommand{\headrulewidth}{0pt}
	\renewcommand{\footrulewidth}{0pt}
}

\pagestyle{fancy}

\renewcommand{\thechapter}{\arabic{chapter}}
\renewcommand{\theequation}{\arabic{chapter}.\arabic{equation}}


\begin{document}

\pagenumbering{gobble}% Remove page numbers (and reset to 1)
\thispagestyle{empty}
\pagecolor{Cyan!15}
\newgeometry{centering}

\vspace*{5.75cm}
\begin{center}

	\vspace*{.5cm}
	{\huge Física}

	\vspace*{.5cm}
	{\huge Estadística}

	\vspace*{1cm}
	{\Large Daniel Alfonso Durán García}

\end{center}

\restoregeometry


\pagecolor{White}
%\tableofcontents
\newpage

\pagenumbering{arabic}
\renewcommand{\headrulewidth}{0.5pt}
\cfoot[\textcolor{gray}{\textit{Física estadística}}]{\textcolor{gray}{\textit{Física estadística}}}

%--------------------------------------
%  Parte I
%--------------------------------------

\part{Mecánica Estadística Clásica}

%--------------------------------------
%  Cap 1
%--------------------------------------
\chapter{Colectividad microcanónica}
\chead[\thechapter. \textsc{Colectividad microcanónica}]{\thechapter. \textsc{Colectividad microcanónica}}

El objeto de la Mecánica Estadística consiste en deducir e interpretar las leyes que rigen el comportamiento de los sistemas macroscópicos a partir de una descripción microscópica de los mismos. Es decir que la Mecánica Estadística considera a los sistemas constituidos por un gran número de partículas ---átomos o moléculas--- cuyo comportamiento viene regido por las leyes de la Mecánica, y trata de obtener a partir de esa descripción las leyes fenomenológicas de la Termodinámica, el Magnetismo, etc.
A primera vista puede parecer que el camino para realizar este programa consistiría en tratar de resolver las ecuaciones del movimiento para el conjunto de partículas que componen el sistema. Sin embargo, la posibilidad de realizar tal cálculo explícito resulta ilusoria si recordamos que un sistema macroscópico contiene un número de partículas que es el del orden de $10^{23}$.

Por otro lado, es fácil prever que una resolución exacta del problema mecánico resultaría, en el caso de ser posible, innecesaria para nuestros fines.
Desde un punto de vista matemático, el estado dinámico de un sistema constituido por $N$ partículas exige para su especificación el conocimiento de las posiciones y velocidades de cada una de ellas, es decir, en general $6N$ parámetros distintos, mientras que el estado macroscópico de un sistema se caracteriza por un pequeño número de parámetros.
Resulta entonces evidente que al pasar de la escala microscópica a la macroscópica se efectúa una contracción en la descripción del sistema, seleccionando parte de la información contenida en la descripción microscópica.

\newpage
\section{Descripción macroscópica y microscópica}

El modelo microscópico o atomístico de un sistema físico se construye teniendo en cuenta la estructura de las partículas que lo componen, las fuerzas de interacción entre ellas y por último el ordenamiento espacial de las mismas en el caso de sólidos cristalinos.
Estos datos se deducen parcialmente a partir de los resultados experimentales y se completan mediante las hipótesis adecuadas.

Hay que señalar que, en la mayor parte de los casos, resulta imposible trabajar con modelos muy cercanos a los sistemas reales, es decir que presenten con gran aproximación todas las propiedades de estos últimos sistemas, debido a su gran complejidad matemática.
Resulta entonces útil obtener información a partir de modelos simplifica dos que presenten al menos cualitativamente alguna de las propiedades de un sistema real.
Estos modelos son utilizados de hecho con frecuencia en todas las partes de la
Física (movimiento de un sólido sin rozamiento, fluidos sin viscosidad, etc.).

Entre la descripción macroscópica del estado de un sistema físico y la descripción del estado de un modelo asociado existen diferencias muy importantes. La descripción macroscópica de un sistema se hace mediante un número muy reducido de parámetros; concretamente la Termodinámica admite que un estado de equilibrio queda totalmente especificado, por ejemplo, mediante los valores de los parámetros externos del sistema y de la temperatura.
Por el contrario, admitiendo que las partículas que componen el sistema obedecen las leyes de la mecánica clásica, sabemos que para especificar el estado microscópico del sistema nos serán necesarias $f$ coordenadas generalizadas y
$f$ momentos generalizados, siendo $f$ el número de grados de libertad del sistema.
Si tenemos en cuenta que para sistemas poco densos, como son los gases, el número de partículas que componen un sistema macroscópico es del orden $10^{23}$, podremos estimar la gran cantidad de parámetros que son necesarios para
especificar el estado de un sistema en una descripción microscópica del mismo.
Resulta claro que la especificación del estado macroscópico de un sistema no puede ser suficiente para la determinación de un estado microscópico, o dicho de otra manera existe un gran número de estados microscópicos compatibles con un estado macroscópico dado. A partir de ahora denominaremos a los estados definidos macroscópicamente \emph{macroestados} y a los definidos sobre una escala microscópica \emph{microestados}.

Planteemos ahora la cuestión siguiente: dado el macroestado de un sistema, ¿en cuál de todos los posibles microestados compatibles con él se encuentra el sistema?
Evidentemente ni la Mecánica ni la Termodinámica pueden contestar a esta pregunta.
El punto de vista adoptado por la Mecánica Estadística consiste en atribuir unas ciertas probabilidades a priori a cada uno de los microestados accesibles al sistema es decir, compatibles con el macroestado dado.
Como siempre, la justificación última de la distribución de probabilidades postulada radicará en la comparación de los resultados obtenidos a partir de ella con los resultados de las experiencias macroscópicas.

Dado que estamos utilizando una descripción clásica del sistema y que en la Mecánica Clásica las coordenadas generalizadas $q_i$ y los momentos generalizados $p_i$ son variables continuas, lo que la Mecánica Estadística va a postular para cada sistema macroscópico es una función densidad de probabilidad para las variables coordenadas
\begin{equation}\label{rho_t1}
	\rho({q_i},{p_i}; t) = \rho(q,p; t)
\end{equation}
de manera que, por la definición de densidad de probabilidad,
\begin{equation}
	\rho({q_i},{p_i}; t)dq_1\cdots dq_f dp_1\cdots dp_f \equiv \rho(q,p; t) dqdp
\end{equation}
represente la probabilidad de que en un instante $t$ dado, las coordenadas y momentos del sistema tengan valores comprendidos en los intervalos
$$(q_1, q_1 + dq_1), (q_2, q_2 + dq_2), ... , (q_f, q_f + dq_f)$$
y
$$(p_1, p_1 + dp_1), (p_2, p_2 + dp_2), ... , (p_f, p_f + dp_f)$$
respectivamente. De acuerdo con esta definición, p(q, P; t) deberá, en todo instante,
cumplir la condición de normalización
\begin{equation}
	\int dqdp \rho(q,p; t) = 1
\end{equation}

Desde luego la función \eqref{rho_t1} que se postule debe ser nula para aquellos valores de $p$, $q$ y $t$ que llevan a un microestado no compatible con el macroestado en que se encuentra el sistema en el instante $t$.

\section{Postulados de la física estadística. Colectividad microcanónica}

A partir de ahora y mientras no se indique explícitamente lo contrario nos vamos a limitar a sistemas que se encuentran en equilibrio. 
Por \emph{equilibrio} en Mecánica Estadística se entiende que la función de distribución de probabilidades $\rho$ es independiente del tiempo.
Esta definición implica directamente que al pasar a una descripción macroscópica todos los parámetros que definen el estado del sistema serán independientes del tiempo, de acuerdo con el concepto de equilibrio termodinámico.

La razón por la cual nos limitamos de momento a sistemas en equilibrio es que para ellos existe una formulación sistemática de la Mecánica Estadística que desarrollaremos en los próximos capítulos, mientras que tal formulación aún no se ha alcanzado para sistemas que no están en equilibrio. No obstante, el Capítulo 11 se dedicará al estudio elemental de sistemas de esta última clase.

Consideremos un \emph{sistema aislado}, es decir un sistema que no puede intercambiar ni materia ni energía con sus alrededores, y por lo tanto poseerá una energía rigurosamente constante.
Parece que estas condiciones son las más sencillas por lo que a relación con sus alrededores o su entorno se refiere, y por ello tomaremos el sistema aislado como punto de partida para el desarrollo de la Mecánica Estadística.
Establecemos entonces el siguiente postulado sobre la forma de la función de distribución de probabilidades:
\vspace{-10pt}
\begin{center}
\fbox{
	\parbox{\textwidth}{
	\textbf{Primer postulado:} \textit{A un estado de equilibrio macroscópico de un sistema aislado corresponde una descripción microscópica en la que todos los microestados accesibles al sistema son igualmente probables.}
	}
}
\end{center}
Esta es una de las manera de enunciar el famoso \emph{postulado de igualdad de probabilidades a priori} en Mecánica Estadística Clásica.

Puede parecer un tanto arbitrario al asignar probabilidades iguales a todos los microestados accesibles de un sistema aislado en equilibrio, y de hecho así es parcialmente.
Sin embargo diversas razones fundamentan esta elección:
\begin{enumerate}
	\item No existe nada dentro de las leyes de la Mecánica que nos indique que el sistema deba encontrarse en uno de los microestados accesibles con preferencia a los demás.
	
	\item Veremos en seguida que, como consecuencia de los resultados del apartado anterior, si esta igualdad de probabilidades se admite en un instante dado, se mantiene en el transcurso del tiempo de acuerdo con nuestra definición de equilibrio.
	
	\item Los resultados que se obtienen a partir de este postulado están de acuerdo con la Termodinámica y la experiencia. Esta es sin duda la razón más importante desde el punto de vista físico y la que justifica plenamente el postulado.
\end{enumerate}

Veamos ahora cómo se traduce matemáticamente nuestro postulado de igualdad de probabilidades a priori, es decir, veamos la forma que tiene $\rho(q, p)$ en el caso de un sistema aislado en equilibrio.

Supongamos que sabemos que la energía del sistema está comprendida entre $E$ y $E + \Delta E$.
Clásicamente $\Delta E$ se puede hacer tan pequeño como queramos, es decir se puede tomar el límite $\Delta E \rightarrow 0$.
Resulta entonces que 
\vspace{-10pt}
\begin{center}
	\fbox{
		\parbox{\textwidth}{
			\textbf{Segundo postulado:} \textit{$\rho(q, p)$ ha de ser una distribución de probabilidades de valor constante para todos los microestados en los que $H(q,p)$ está comprendido entre $E$ y $E + \Delta E$ con $\Delta E \rightarrow 0$, y de valor nulo fuera de ese intervalo.}}
		}
\end{center}

Recordando las propiedades de la función delta de Dirac, vemos que ese comportamiento lo obtenemos si tomamos
\begin{equation}\label{eq:rho_1}
	\rho(q,p) \propto \delta \left[ H(q,p;X_\alpha) - E \right] 
\end{equation}

La relación exacta vendrá dada por una constante $C$ añadida.
Para su determinación no tenemos más que echar mano de las propiedades de las densidades de probabilidad.
Si $\rho$ debe estar normalizada, definimos dicha constante $C$ como
\begin{equation}\label{eq:Omega_t1}
	1 = \int dq dp \, C \rho \Rightarrow C = \frac{1}{\int dq dp \ \delta \left[ H(q,p;X_\alpha) - E \right]} \equiv \frac{1}{\Omega(E,N,V)} 
\end{equation}

Una vez aquí debemos pararnos en la integral del denominador. Dicha integral la entenderemos siempre extendida a todo el espacio fásico, lo cual quiere decir que las restricciones que impongan los parámetros externos se encuentran incluidas en el hamiltoniano, que dependerá de ellos aun cuando de momento no hayamos 'indicado explícitamente esta dependencia en \eqref{eq:rho_1}.
Un ejemplo servirá para aclarar lo que acabamos de indicar.
Imaginemos un sistema de partículas encerradas en un volumen $V$.
La imposibilidad de que las partículas se encuentren en las paredes o en el exterior del sistema puede representarse mediante un término de energía potencial en el hamiltoniano que sea constante ---nulo con un origen adecuado--- en todo el volumen considerado e infinito fuera de él.

La dependencia respecto a los parámetros externos del hamiltoniano implica que también dependerán de ellos $\rho$ y $\Omega$. Con esta observación el significado de $\Omega$ es claro: dado que asignamos una probabilidad constante a los estados accesibles y nula a los no accesibles y $\Omega(E)$ viene a ser una medida del número de microestados accesibles al sistema, es decir compatibles con las condiciones que lo delimitan.

Introduzcamos ahora otra magnitud. Imaginemos que $E_0$ representa el valor más bajo de la energía que puede
poseer el sistema para un valor dado de los parámetros externos. Definimos
\begin{align}
	\Gamma(E,N,V) &= \int_{E_0}^{E} d\mathcal{E} \ \Omega(\mathcal{E},N,V) \nonumber \\
	&= \int_{E_0}^{E} d\mathcal{E} \int dq dp \ \delta \left[ H(q,p;X_\alpha) - \mathcal{E} \right] \nonumber \\
	&= \int dq dp \int_{E_0}^{E} d\mathcal{E} \ \delta \left[ H(q,p;X_\alpha) - \mathcal{E} \right]\\
	&= \mkern-18mu \int\limits_{E_0 \le H \le E} \mkern-18mu dq dp \nonumber
\end{align}

Y así, $\Gamma$ representa el volumen del espacio fásico entre las hipersuperficies que delimitan $E_0$ y $E$. Se le suele llamar \emph{volumen fásico}.

Consideremos ahora dos hipersuperficies infinitesimalmente próximas, una con energía $E$ y la otra con $E + dE$ Y formemos la diferencia $\Gamma(E + dE) - \Gamma(E)$. Desarrollando $\Gamma(E + dE)$ en serie hasta el primer orden
$$\Gamma(E + dE,N,V) \approx \Gamma(E,N,V) + \eval{\pdv{\Gamma(E + dE,N,V)}{E}}_{\Delta E = 0} $$
$$\Gamma(E + dE,N,V) - \Gamma(E,N,V) \approx \eval{\pdv{\Gamma(E + dE,N,V)}{E}}_{\Delta E = 0} = \Omega(E,N,V) dE = \mkern-18mu \int\limits_{E_0 \le H \le E} \mkern-18mu dq dp$$

lo que nos dice que $\Omega(E) dE$ representa el volumen de espacio fásico encerrado entre dos hipersuperficies muy próximas que corresponden a energías $E$ y $E+dE$ constantes.
Obsérvese que $\Omega(E)$ puede también interpretarse como una medida del área de la hipersuperficie de energía $E$.

\section{Dependencia del volumen fásico y del número de estados respecto a la energía}

En nuestros razonamientos posteriores va a jugar un papel muy importante el hecho de que el número de microestados accesibles al sistema, o dicho de otra forma la región de espacio fásico accesible, es una función rápidamente creciente de la energía. 
El objeto de este apartado va a ser estudiar explícitamente la dependencia de $\Gamma(E)$ y $\Omega(E)$ respecto de E en un caso particularmente sencillo: el gas monoatómico ideal.

Consideremos un conjunto de $N$ partículas puntuales iguales, no interaccionantes entre sí y encerradas en un volumen $V$. 
Este sistema constituye el modelo microscópico de un gas ideal monoatómico, Representaremos la masa de cada partícula por $m$ y la energía total del sistema por $E$. 
Dado que despreciamos las interacciones entre las partículas ---lo que denominamos \emph{gas diluido}---, éstas poseerán únicamente energía cinética y el hamiltoniano del sistema puede escribirse $$H(p,q) = \sum_{i=1}^{N} \frac{d{\vec{p}_{i}}^{\,2}}{2m}$$

A partir de la definición de $\Gamma$, y separando los diferenciales:
\begin{align}
	\Gamma(E,N,V) &= \int dq_1 dq_2 \cdots dq_N dp_1 dp_2 \cdots dp_N \nonumber \\
				  &= \int d{\vec{r}_{1}}^{\,3} d{\vec{r}_{2}}^{\,3} \cdots d{\vec{r}_{N}}^{\,3} d{\vec{p}_{1}}^{\,3} d{\vec{p}_{2}}^{\,3} \cdots d{\vec{p}_{N}}^{\,3} \\
				  &= V^N \int d{\vec{p}_{1}}^{\,3} d{\vec{p}_{2}}^{\,3} \cdots d{\vec{p}_{N}}^{\,3} \nonumber
\end{align}
y con el hamiltoniano definido al principio, donde vemos que $\sum d{\vec{p}_{i}}^{\,2} = 2mE$. 
La integral que queda en el desarrollo indica el volumen en el espacio fásico, con radio $R = \sqrt{2mE}$.
Así, tenemos que 
\begin{equation}
	\Gamma(E,N,V) = CV^N(2mE)^\frac{3N}{2}
\end{equation}
con una constante a determinar que dependerá de la dimensión del volumen del que hablemos. Su desarrollo se encuentra en el \hyperref[Anx1]{Anexo 1}. En nuestro caso, con dimensión $N$ obtenemos 
\begin{equation}
	\Gamma(E,N,V) = V^N\frac{\pi^{\frac{3N}{2}}}{\Gamma(\frac{3N}{2}+1)}(2mE)^\frac{3N}{2}
\end{equation}

Ahora, con la definición de $\Omega$:
\begin{align}
	\Omega(E,N,V) &= \eval{\pdv{\Gamma(E,N,V)}{E}}_{N,V} = V^N\frac{\pi^{\frac{3N}{2}}}{\Gamma(\frac{3N}{2}+1)}(2m)^\frac{3N}{2}\frac{3N}{2}E^{\frac{3N}{2}-1}  \nonumber\\
				  &= V^N\frac{\pi^{\frac{3N}{2}}}{\Gamma(\frac{3N}{2}+1)}(2m)^\frac{3N}{2}\frac{3N}{2}\frac{2m}{2mE} \\
				  &= \Gamma(E,N,V)\frac{3N}{2E}  \nonumber
\end{align}

Tomando logaritmos
\begin{equation}
	\ln \Omega(E,N,V) = \ln \Gamma(E,N,V) + \ln (3N) - \ln (2E)
\end{equation}
y despreciando los últimos términos por la gran velocidad de crecimiento de $\Omega$ y $\Gamma$ con $E$ tenemos que, para estados microscópicos podemos aproximar
\begin{equation}
	\boxed{\ln \Omega(E,N,V) \approx \ln \Gamma(E,N,V)}
\end{equation}

\section{Invarianza adiabática del volumen fásico}

Vamos ahora a estudiar cómo varía en un proceso adiabático de este tipo el volumen fásico $\Gamma$, que sabemos es función de la energía del sistema y de los parámetros externos $X_\alpha$. Utilizando la expresión de la diferencial de una función de dos variables tenemos
\begin{equation}
	d\Gamma(E,X_\alpha) = \left( \pdv{\Gamma}{E} \right)_{X_\alpha} dE + \left( \pdv{\Gamma}{X_\alpha} \right)_{E} dX_\alpha
\end{equation}

La primera de las derivadas que aparecen en el segundo miembro ya la obtuvimos 
\begin{equation}\label{eq:Gamma_E_t1}
	\left( \pdv{\Gamma}{E} \right)_{X_\alpha} = \Omega(E,X_\alpha) 
\end{equation}
pero para la segunda necesitamos la expresión de la derivada de una integral, en la que los límites de integración y el integrado dependen de la variable respecto de la que se deriva
\begin{equation}
	\pdv{y} \int^{g(y)}_{f(y)} dx F(x,y) = \int^{g(y)}_{f(y)} \pdv{F}{y} + \left\lbrace \pdv{g}{y}F(g(y),y)- \pdv{f}{y}F(g(y),y) \right\rbrace
\end{equation}

Utilizando esta expresión y derivando respecto de $X_\alpha$ obtenemos

\begin{align*}
	\left( \pdv{\Gamma}{X_\alpha} \right)_{E} &= \pdv{X_\alpha} \int^E_{E_0} dE' \int dq dp \, \delta \left[ E' - H(q,p;X_\alpha) \right] \\
	&=  \int^E_{E_0} dE' \int dq dp \, \pdv{X_\alpha} \delta \left[ E' - H(q,p;X_\alpha) \right]  - \pdv{E_0}{X_\alpha} \int dq dp \, \pdv{X_\alpha} \delta \left[ E' - H(q,p;X_\alpha) \right] \\
	&=  - \int^E_{E_0} dE' \int dq dp \, \pdv{H}{X_\alpha} \pdv{E'} \delta \left[ E' - H(q,p;X_\alpha) \right] \\ 
	& \quad - \pdv{E_0}{X_\alpha} \int dq dp \, \pdv{X_\alpha} \delta \left[ E' - H(q,p;X_\alpha) \right]
\end{align*}

Ahora, usando
$$\int^E_{E_0} \dd{E'} \pdv{E'} \delta \left[ E' - H(q,p;X_\alpha) \right] = \delta \left[ E - H(q,p;X_\alpha) \right]  - \delta \left[ E_0 - H(q,p;X_\alpha) \right] $$
y agrupando los términos resultantes de la integración acabamos con

\begin{align} \label{eq:Gamma_X_a_t1}
	\left( \pdv{\Gamma}{X_\alpha} \right)_{E} &= - \int^E_{E_0} dE' \int dq dp \pdv{H}{X_\alpha} \delta \left[ E' - H(q,p;X_\alpha) \right]  \nonumber \\ 
	&\quad - \int^E_{E_0} dE' \int dq dp \left[ \pdv{E_0}{X_\alpha} - \pdv{H}{X_\alpha} \right] \delta \left[ E_0 - H(q,p;X_\alpha) \right] \\
	&= -\Omega(E,N,V) \expval{\pdv{H}{X_\alpha}} \nonumber
\end{align}

Una vez tenemos ambos términos podemos evaluar finalmente el diferencial con el que comenzamos
\begin{equation}
	d\Gamma(E,X_\alpha) = \Omega(E,N,V)\left[dE - \expval{\pdv{H}{X_\alpha}} dX_\alpha \right]
\end{equation}

Identificado la variación de energía macroscópica dE con la variación media de la energía microscópica $dE$, y utilizando la identidad de $dE$ del inicio de la sección resulta que $$d\Gamma(E,X_\alpha) = 0$$ para los procesos que consisten en la variación cuasiestática de un parámetro externo en un sistema térmicamente aislado. Dicho de otra manera, $d\Gamma(E,X_\alpha)$ es un invariante para estos procesos.

En Termodinamica la única magnitud que no varía, o sea, el único invariante que posee un sistema térmicamente aislado en un proceso cuasiestático, es la \emph{entropía}. Podemos entonces prever que una definición consistente de entropía en Mecánica Estadistica debe en principio estar relacionada con esta magnitud $\Gamma$.

\section{Entropía y temperatura absoluta}

Estudiemos ahora la dependencia del logaritmo neperiano del volumen fásico respecto de la energía y de los parámetros externos, Para ello consideremos la expresión
\begin{equation}
	d\Gamma(E,X_\alpha) = \frac{1}{\Gamma} \left[ \left( \pdv{\Gamma}{E} \right)_{X_\alpha} dE + \left( \pdv{\Gamma}{X_\alpha} \right)_{E} dX_\alpha\right]
\end{equation}
que podemos reescribir con \eqref{eq:Gamma_E_t1} y \eqref{eq:Gamma_X_a_t1} como
\begin{equation}\label{eq:dgamma_t1}
	d\Gamma(E,X_\alpha) = \frac{\Omega}{\Gamma} \left[ dE + \expval{Y_\alpha} dX_\alpha\right]
\end{equation}

Obsérvese que esta expresión entre paréntesis no es nula como sucedía en el apartado anterior, pues ahora dE proviene tanto de un intercambio de energía en forma de calor como en forma de trabajo.

La expresión \eqref{eq:dgamma_t1} puede escribirse recordando la relación entre $\Gamma$ y $\Omega$ como
\begin{equation}
	d\Gamma(E,X_\alpha) = \left( \pdv{\ln \Gamma}{E} \right)_{X_\alpha} \left[ dE + \expval{Y_\alpha} dX_\alpha\right]
\end{equation}
de donde, con la primera ley de la termodinámica $dE \equiv dU = đQ - đW$ se obtiene
\begin{equation}
	d\Gamma(E,X_\alpha) = \left( \pdv{\ln \Gamma}{E} \right)_{X_\alpha} đQ
\end{equation}

Como el primer miembro de esta igualdad es una diferencial exacta, resulta que el segundo también lo será, es decir, que $đQ$ admite un factor integrante que es $$\left( \pdv{\ln \Gamma}{E} \right)_{X_\alpha}$$

Ahora bien, en Termodinámica $đQ$ también posee un factor integrante, que es el inverso de la temperatura absoluta, y la magnitud cuya diferencial exacta se obtiene es la entropía $S$, que resulta ser de este modo una función de estado.
Así, podemos identificar ambas magnitudes
\begin{empheq}[box=\fbox]{align}
	S &= k_B \ln \Gamma(E,X_\alpha) \\
	T &= \left[ k_B \left( \pdv{\ln \Gamma}{E} \right)_{X_\alpha} \right]^{-1}
\end{empheq}

Se suele utilizar para una mayor sencillez en la escritura un parámetro $\beta$ definido como $$\beta = \frac{1}{k_B T}$$

\emph{[Nota mental: revisar el final porque puede que quede algo por poner]}

\section{Gas monoatómico ideal}

Una vez efectuada la conexión entre las descripciones macroscópica y microscópica de un sistema, vamos aplicar los resultados obtenidos al cálculo de las magnitudes termodinámicas de un sistema concreto: el gas ideal monoatómico.

Anteriormente obtuvimos la expresión del volumen fásico
$$\Gamma(E,N,V) = V^N\frac{\pi^{\frac{3N}{2}}}{h^{3N}\Gamma(\frac{3N}{2}+1)}(2mE)^\frac{3N}{2}$$
donde hemos añadido el factor $h^f = h^{3N}$. A partir de la expresión anterior, resulta para la temperatura
$$\beta = \frac{1}{k_B T} = \left( \pdv{\ln \Gamma}{E} \right)_V = \frac{3N}{2E} \Rightarrow \frac{E}{N} = \frac{3}{2}k_B T$$

Este resultado puede interpretarse diciendo que a cada partícula corresponde en valor medio una energía igual a $3k_BT /2$. De hecho puede establecerse un resultado análogo mucho más general que se conoce con el nombre de \textit{teorema de equipartición de la energía}, del que esta igualdad representa un caso concreto.

La ecuación térmica de estado de un gas ideal podemos deducirla a partir de la expresión que nos da la presión
$$\expval{Y_{\alpha}} \equiv \expval{p} = \frac{1}{\beta} \left( \pdv{\ln \Gamma(E, V)}{E} \right)_V = k_B T \left( \pdv{\ln \Gamma(E, V)}{E} \right)_V$$ donde, sustituyendo con la expresión del volumen fásico que recordamos al inicio de la sección obtenemos la ecuación de estado térmica de los gases ideales
$$\expval{p} v = N k_B T = RT$$

Calculemos ahora la expresión de la entropía, usando la \emph{fórmula de Stirling} (\hyperref[Anx2]{Anexo 2})
\begin{align}
	S(E,V) &= k_B \ln \Gamma(E, V)  \nonumber \\
	&= k_B \left[ \frac{3N}{2}\ln\pi -\frac{3N}{2}\ln \frac{3N}{2} + \frac{3N}{2} - 3N\ln h + N\ln V + \frac{3N}{2}\ln(2mE) \right]  \nonumber \\
	&\quad \boxed{\sigma \equiv \frac{3}{2} \ln \left( \frac{2mk_B}{h^2} \right) + \frac{3}{2}}  \nonumber \\
	&= N k_B \left[ \frac{3}{2}\ln T +\ln V + \sigma \right]
\end{align}

Esta expresión de la entropía, totalmente análoga a la que se obtiene por razonamientos termodinámicos, no es satisfactoria, ya que conduce a la denominada \emph{paradoja de Gibbs}. 
Dejando a un lado la exposición cualitativa de dicha paradoja y que puede verse en cualquier texto de Termodinámica, digamos que se presenta debido a que la expresión obtenida no es aditiva o extensiva.
En efecto si, manteniendo la temperatura constante, aumentamos el volumen y el número de partículas del sistema en un factor a (es decir, unimos a sistemas idénticos) la nueva entropía sí viene relacionada con la del sistema original $S$ por
$$S' = \alpha S + \alpha N k_B \ln \alpha$$
en lugar de $S' = \alpha S$ que debería resultar si la entropía satisficiera el requirimiento de extensividad.

¿Qué es lo que ha fallado en nuestro razonamiento? Es fácil ver que la introducción de un término adicional de la forma $-N k_B \ln N$ resuelve la dificultad pues obtendríamos
\begin{equation}
	S(E, V) =  N k_B \left[ \frac{3}{2}\ln T +\ln \frac{V}{N} + \sigma \right]
\end{equation}
expresión que es aditiva.

La opinión generalmente aceptada por los distintos autores es que las dificultades provienen de un concepto demasiado \emph{fino} de microestado, Según esta idea no hay razón para considerar como distintos dos microestados que únicamente se diferencian en el intercambio de partículas iguales. 
Es decir, que todas las situaciones que se obtienen permutando entre sí las $N$ partículas que componen el gas deben considerarse como un único microestado del sistema.

Resulta entonces que al determinar $\Omega(E)$ por la condición de normalización habría que dividir por $N!$, número de permutaciones de $N$ partículas, pues de la función $\rho(q, p)$ definida en secciones anteriores hemos pasado a otra que está definida en un volumen del espacio fásico $N!$ veces menor. Lo mismo sucedería con el volumen fásico definido al inicio,
\begin{equation}
	\Gamma ' (E,V) = \frac{1}{h^{3f}N!} \int dqdp = \frac{1}{N!}\Gamma
\end{equation}

Con estas definiciones, que matemáticamente pueden entenderse como una métrica especial del espacio fásico, ninguno de los resultados generales obtenidos hasta ahora sufriría modificación, pues hasta este apartado solamente hemos considerado explícitamente la dependencia respecto de $N$ en el cálculo de $\Gamma$ para el caso de un gas ideal monoatómico.

La entropía resultará ser
\begin{equation}
	S'(E, V) = N k_B \left[ \frac{3}{2}\ln T +\ln \frac{V}{N} + \sigma' \right] \quad | \quad \sigma' = \sigma + 1
\end{equation}

De este modo puede resolverse la paradoja de Gibbs. Parece, sin embargo, que la solución no es del todo satisfactoria, pues ha sido introducida un tanto forzadamente.
Desde luego puede justificarse, y así se hace, diciendo que la Mecánica Clásica no es apropiada para describir los sistemas estadísticos constituidos por partículas iguales (lo que en el fondo equivale a decir que no es apropiada para describir ningún sistema estadístico).

A pesar de todo ello se nos plantea la cuestión del origen de esta dificultad si nosotros hemos comprobado la aditividad de la definión de entropía dada. 
Un análisis de los desarrollos efectuados proporciona la respuesta: hemos estudiado la aditividad de la entropía en subsistemas que podían intercambiar calor y trabajo, pero nunca cuando era posible el intercambio de partículas.
Es decir hemos comprobado que nuestra definición de entropía presentaba una dependencia correcta respecto de la energía (temperatura) y respecto del volumen, pero no hemos estudiado si es correcta su dependencia respecto del número de partículas.

\section{Interacción en general: equilibrio}

Definida la entropía en apartados anteriores, vamos a tratar de establecer ahora si posee la propiedad de aditividad que goza en Termodinámica. Dicha propiedad afirma que si un sistema termodinámico se descompone en dos subsistemas cuyas entropías son $S_1$ y $S_2$ entonces la entropía del sistema total es $S_1 + S_2$ , lo cual equivale a decir que la entropía es una magnitud extensiva (Notemos que subsistema se utiliza aquí en un sentido macroscópico).

Consideremos primero dos sistemas $A_1$ y $A_2$ puestos en contacto mediante una pared diaterma rígida, es decir una pared que permite/la interacción térmica entre ellos pero no la interacción mecánica. El sistema compuesto $A$ formado por ambos está aislado y por tanto corresponde a una energía constante $E$.

El hamiltoniano general del sistema será
$$H(q,Q,p,P) = H_1(p,q) + H_2 (Q,P) + H_{12}(q,Q,p,P)$$
donde $H_{12}$ representa la energía de interacción entre $A_1$ y $A_2$. Nosotros vamos a admitir que esta energía es despreciable frente a la de cualquiera de los dos sistemas.

Desde luego el hamiltoniano de interacción no puede ser estrictamente nulo, ya que entonces se negaría la posibilidad de un intercambio de energía entre $A_1$ y $A_2$.
Puede verse fácilmente que si $H$ tuviese únicamente la suma de los dos subsistemas cada uno de los sistemas $A_1$ y $A_2$ evolucionaría separadamente, siendo cada uno de ellos conservativo.
No obstante, es justificable despreciar la energía de interacción en sistemas macroscópicos si las fuerzas de interacción que se consideran poseen un alcance limitado. 
En este caso $E_1$ y $E_2$ son proporcionales a los volúmenes de $A_1$ y $A_2$, mientras que la energía de interacción es proporcional a la superficie de separación.

Al sistema total $A$ le es aplicable la distribución microcanónica de manera que le asignaremos una densidad de probabilidad en el espacio de las fases
\begin{equation}
	\rho(q,Q,p,P) = \frac{1}{h^f \Omega(E)} \delta \left[ E - H_1(q,p;X_\alpha) - H_2(Q,P;X_\alpha) \right]
\end{equation}
donde $f = f_1 + f_2$.

De acuerdo con esta distribución resulta que son posibles en principio cualquier
par de valores $E_1$ y $E_2$ con la única condición de que
$$E = E_1+ E_2$$

En este sentido es posible que toda la energía $E$ del sistema total se concentre a $A_1$, quedando $A_2$ con una energía nula. Parece entonces que la entropía de, por ejemplo, el sistema $A_1$ no está definida pues como sabemos el volumen fásico $\Gamma$ es una función de la energía.
La situación sin embargo, no es ésta, pues no debemos olvidar el hecho fundamental de que la entropía es un concepto macroscópico, que no está ligado a ningún microestado en concreto sino al conjunto de todos los microestados del sistema.
Vamos a ver que, si se tiene presente esta idea, las entropías de $A_1$ y $A_2$ están definidas de manera única como consecuencia de que también lo están las energías.
Para ello vamos a plantearnos la cuestión de cuál es la probabilidad $\omega(E_1) dE_1$ de que el sistema total se encuentre en equilibrio en un estado tal que la energía del subsistema $A_1$ esté comprendida entre $E_1$ y $E_1 + dE_1$.

La densidad de probabilidad $\rho(q, p)$ de que el sistema Al se encuentre en el microestado definido por $(q, p)$ independientemente de cuál sea el estado de $A_2$ se obtendrá integrando la distribución microcanónica para todos los valores de $Q$ y $P$
\begin{align}
	\rho(q,Q,p,P) &= \frac{1}{h^f \Omega(E)} \int dQ dP\delta \left[ E - H_1(q,p;X_\alpha) - H_2(Q,P;X_\alpha) \right]  \nonumber \\
			      &= \frac{1}{h^f \Omega(E)} \Omega_2 \left[ E - H_1(q,p;X_\alpha) \right]
\end{align}
donde $\Omega_2 \left[ E - H_1(q,p;X_\alpha) \right]$ representa la derivada respecto de la energía del volumen encerrado en el espacio $Q - P$ por la hipersuperficie
$$H_2 (P, Q) = E - H_1(q, p)$$

Con este resultado es evidente
\begin{align}
	\omega(E_1) dE_1 &= \mkern-18mu \int\limits_{E_1 \le H_1(q, p) \le E_1+dE_1} \mkern-50mu dq dp \mkern10mu \rho(q, p)  \nonumber\\
					 &= \frac{1}{h^{f_1} \Omega(E)} \Omega_2 \left[ E - E_1 \right] \mkern-28mu \int\limits_{E_1 \le H_1(q, p) \le E_1+dE_1} \mkern-40mu dq dp \\
					 &= \frac{1}{\Omega(E)} \Omega_1(E) \Omega_2 \left[ E - E_1 \right] dE_1  \nonumber
\end{align}

%-------------------Gráfica---------------------------
\pgfmathdeclarefunction{gauss}{2}{%
	\pgfmathparse{1/(#2*sqrt(2*pi))*exp(-((x-#1)^2)/(2*#2^2))}%
}

\begin{wrapfigure}{l}{0.25\textwidth}
	\centering
	\hspace{-2.5cm}
	\begin{tikzpicture}
		\begin{axis}[
			scale = 0.65,
			xmin = 0, xmax = 3,
			every axis plot post/.append style={
			mark=none,samples=80,smooth},
			axis lines=left, % no box around the plot, only x and y axis
			xlabel={$E_1$},
			xlabel style={at=(current axis.right of origin), anchor=west},
			xtick = {1.5},
			xticklabel = {$\widetilde{E_1}$},
			ytick=\empty,
			ylabel={$\omega(E_1)$},
			ylabel style={at=(current axis.above origin), anchor=south,rotate=-90},
			enlargelimits=upper] % extend the axes a bit to the right and top
			
			\addplot[name path=f, NavyBlue, ultra thick] {gauss(1.5,0.25)};
				
			\addplot [fill=NavyBlue!20, draw=none, domain=0:3] {gauss(1.5,0.25)} \closedcycle;
			
		\end{axis}
	\end{tikzpicture}
	\vspace{-0.5cm}
\end{wrapfigure}

%----------------------------------------------------------


Analicemos cualitativamente el resultado obtenido.
Hemos visto en secciones anteriores que en el caso de sistemas macroscópicos $\Omega(E)$ es una función rápidamente creciente de la energía.
Resulta entonces que si aumentamos $E_1$ manteniendo $E$ constante $\Omega(E_1)$ crecerá rápidamente, mientras que $\Omega(E_2)$ disminuirá también con gran rapidez.
El resultado es que $\omega(E_1)$ presentará un máximo extremadamente agudo para un cierto valor $\widetilde{E_1}$ de la energía, es decir, un comportamiento del tipo representado en la figura adyacente.
En esta distribución será
$$\frac{\Delta^* E_1}{\widetilde{E_1}} \ll 1$$

¿Qué consecuencias físicas podemos obtener de este resultado? En primer lugar que al pasar a una descripción macroscópica el sistema $A_1$ tendrá una energía $\widetilde{E_1}$ y el $A_2$ una energía $\widetilde{E_2} = E-\widetilde{E_1}$, y además estos valores presentan fluctuaciones despreciablemente pequeñas.

Para comprobar la aditividad de la entropía partimos de la relación evidente ---basta por ejemplo exigir la normalización---
\begin{equation}
	\Omega(E) = \int dE_1 \Omega_1(E_1) \Omega_2(E - E_1)
\end{equation}

Recordando el concepto de desviación cuadrática media podemos escribir (ver Fig)
$$\Omega(E) = \Omega_1(\widetilde{E_1}) \Omega_2(\widetilde{E_2})n\Delta^*E_1$$
siendo $n$ del orden de unas pocas unidades.
Tomando logaritmos se obtiene
\begin{equation}
	\ln \Omega(E) = \ln \Omega_1(\widetilde{E_1})  + \ln \Omega_2(\widetilde{E_2}) + \ln(n\Delta^*E_1)
\end{equation}

Ahora bien, sabemos que 
$$\ln \Omega_1(\widetilde{E_1}) \sim f_1\ln \widetilde{E_1}$$
y como por $\Delta^*E_1$ es despreciable frente a $E_1$ resulta que mucho más lo será $\ln (n\Delta^*E_1)$ frente a $f_1 \ln E_1$ (recuérdese que $n$ es del orden de unas pocas unidades mientras que $h$ es del orden de $10^{23}$ ).
En consecuencia podemos escribir
\begin{equation}
	\ln \Omega(E) = \ln \Omega_1(\widetilde{E_1})  + \ln \Omega_2(\widetilde{E_2})
\end{equation}
con lo que se demuestra la propiedad de aditividad.

Más aún, unan vez nos encontramos en el equilibrio podemos llegar a otra propiedad. En efecto, podemos
determinar $\widetilde{E_1}$ para lo cual igualamos a cero la primera derivada de $\ln \omega(E_1)$ respecto
a $E_1$ (manteniendo constantes los parámetros externos):
$$0 = \left( \pdv{\ln \omega(E_1)}{E_1}\right)_{E_1 = \widetilde{E_1}}  = \left( \pdv{\Omega_1(E_1)}{E_1} \right)_{E_1 = \widetilde{E_1}} + \left( \pdv{\Omega_2(E_2)}{E_1} \right)_{E_2 = \widetilde{E_2}}$$
y, como $\pdv{E_1}{E_2} = -1 $, resulta
\begin{equation}
	\left( \pdv{\Omega_1(E_1)}{E_1}\right)_{E_1 = \widetilde{E_1}} = \left(  \pdv{\Omega_2(E_2)}{E_1}\right)_{E_2 = \widetilde{E_2}}
\end{equation}

Con la definición de $\beta$ dada en secciones anteriores se obtiene
\begin{equation}
	\beta_1 = \beta_2 \Rightarrow \boxed{T_1 = T_2}
\end{equation}
y de manera análoga, con la diferenciación respecto de los movimientos generalizados tendremos la igualdad en las fuerzas generalizadas.

Así, pues, hemos obtenido que, si dos sistemas en equilibrio están además en equilibrio térmico entre sí, la entropía del sistema total es igual a la suma de las entropía que cada uno de ellos tendría si estuviera aislado con un valor de la energía igual a su energía media.
Hemos visto además que las temperaturas de los dos sistemas son iguales.

En nuestros razonamientos ha jugado un importante papel el hecho de que el sistema total estaba en equilibrio; en particular solamente en ese caso hemos definido su entropía, y solamente en ese caso se le puede aplicar la distribución microcanónica.








%--------------------------------------
%  Cap 2
%--------------------------------------

\chapter{Colectividad canónica}
\chead[\thechapter. \textsc{Colectividad canónica}]{\thechapter. \textsc{Colectividad canónica}}

\newpage
\section{Colectividad canónica}

Consideremos dos sistemas $A_1$ y $A_2$ en contacto térmico. Este caso fue discutido ya en el capítulo anterior, pero ahora vamos a suponer que uno de los sistemas es mucho más pequeño que el otro, es decir, que posee muchos menos grados de libertad.
Utilizaremos la misma nomenclatura que en el capítulo anterior identificando además el sistema $A_1$ como el menor.
La probabilidad de encontrar el sistema $A_1$ en un microestado definido por unas posiciones comprendidas entre $q$ y $q + dq$ y unos momentos comprendidos en el intervalo entre $p$ y $p + dp$ viene dada por
\begin{equation}
	\rho(q,p)dqdp = \frac{1}{h^f \Omega(E)} \Omega_2 \left[ E - H_1(q,p;X_\alpha) \right] dqdp
\end{equation}

Esta probabilidad hemos visto que es prácticamente nula excepto cuando el microestado corresponda a una energía $E$, definida por
$$\left(  \frac{\partial \Omega_1(E_1)}{\partial E_1}\right)_{E_1 = \widetilde{E_1}} = \left(  \frac{\partial \Omega_2(E_2)}{\partial E_1}\right)_{E_2 = \widetilde{E_2} = E-\widetilde{E_1}}$$

A fin de realizar un análisis cualitativo de esta igualdad recordemos que $\Omega(E) \sim f^{\nu}$ ,con lo que obtenemos la relación entre órdenes de magnitud
$$\frac{\nu_1 f_1}{\widetilde{E_1}} \sim \frac{\nu_2 f_2}{\widetilde{E_2}} \Rightarrow \frac{\widetilde{E_1}}{\widetilde{E_2}} \sim \frac{f_1}{f_2}$$
ya que $\nu_1$ y $\nu_2$ son del orden de la unidad.

Nosotros estamos considerando $f_1 \ll f_2$, luego en la región en que $\rho(q,Q,p,P)$ es distinto de cero se cumplirá que
\begin{equation}
	H_1(q,p) \ll H_2(Q,P) \simeq H_2(Q,P) + H_1(q,p) = E
\end{equation}

Este hecho nos permite desarrollar el logaritmo neperiano de $\Omega_2[E - H_1(q, p)]$ en potencias de $H_1$, reteniendo sólo el menor orden
\begin{equation}
	\ln \Omega_2[E - H_1(q, p)] \simeq \ln \Omega_2(E) - \left(  \pdv{\Omega_2(E_2)}{E_1} \right)_{E_2 = E}
\end{equation}

La derivada
$$\left(  \pdv{\Omega_2(E_2)}{E_1} \right)_{E_2 = E} = \beta$$
define la temperatura del sistema mayor.
Obsérvese que la derivada está calculada para $E_2 = E$ y no para $E_2 = \widetilde{E_2}$. Sin embargo, si el sistema $A_2$ es suficientemente grande ambos resultados coincidirán evidentemente.
Cuando un sistema $A_2$ es tan grande respecto a otro $A_1$, que su parámetro de temperatura permanece esencialmente invariante al, ponerlos en contacto térmico, cualquiera que sea la cantidad de energía que intercambian, se dice que el sistema $A_2$ actúa como \emph{foco térmico} respecto del sistema $A_1$.

Con estos dos últimos resultados obtenemos
\begin{equation}
	\Omega_2[E - H_1(q, p)] = \Omega_2(E)\exp \left[ -\beta H_1(q,p) \right]
\end{equation}
obteniendo la densidad de probabilidad
\begin{equation}
	\rho(q,p) = C \exp \left[ -\beta H_1(q,p) \right] dqdp
\end{equation}
donde $C$ es una constante independiente de $q$ y de $p$, que puede determinarse por la condición de normalización
$$\int dqdp \, \rho(q,p) = 1 \, \Rightarrow \, C^{-1} = \int dqdp \exp \left[ -\beta H_1(q,p) \right]$$
de modo que podemos escribir
\begin{equation}
	\rho(q,p) = \frac{\exp \left[ -\beta H_1(q,p) \right] dqdp}{\int dqdp \exp \left[ -\beta H_1(q,p) \right]}
\end{equation}

Esta distribución de probabilidades se denomina \emph{canónica}, y un conjunto de microestados con esta distribución de probabilidades se llama \emph{conjunto} o \emph{colectividad canónica}.
A partir de la densidad de probabilidad puede escribirse la distribución de probabilidades para la energía del sistema, sin más que integrar $\rho(q,p)$ para toda la región del espacio fásico definida por
$$E_1 \leq H_1(q, p) \leq E_1 + dE_1$$

Los razonamientos son análogos a los que utilizamos en la colectividad microcanónica, resultando
\begin{equation}
	\omega_1(E_1)dE_1 = C \Omega_1(E_1) \exp \left[ -\beta H_1(q,p) \right] dE_1
\end{equation}
o escribiendo explícitamente el valor de la constante de normalización $C$,
\begin{equation}
	\omega_1(E_1)dE_1 = \frac{\Omega_1(E_1) \exp \left[ -\beta H_1(q,p) \right] dE_1}{\int dqdp \exp \left[ -\beta H_1(q,p) \right]}
\end{equation}

\section{Función de partición y cálculo de valores medios}
En el apartado anterior hemos introducido la distribución canónica. Vamos a ver ahora que todas las propiedades macroscópicas del sistema se expresan de una manera sencilla en función del denominador de esa expresión, o más concretamente de la denominada \emph{función de partición} que se representa por la letra $Z$ y se define como ---prescindiremos a partir de ahora del subíndice 1---
\begin{equation}\label{eq:Zdef}
	Z(T, X_\alpha, N) = \frac{1}{h^f} \int dqdp \, \exp \left[ -\beta H(q,p) \right]
\end{equation}
donde hemos introducido el factor $h^f$ para hacer la función de partición una magnitud adimensional (tal y como hicimos con $\Gamma$ anteriormente). La función de partición dependerá en general del parámetro de temperatura y, a través del hamiltoniano, de los parámetros externos $X_\alpha$ como, por ejemplo, el volumen o un campo magnético o eléctrico externo y del número de partículas $N$.

Con \eqref{eq:Zdef}, la distribución canónica se escribe
\begin{equation}
	\rho(q,p) = \frac{\exp \left[ -\beta H_1(q,p) \right] dqdp}{h^f Z}
\end{equation}

Calculemos la energía media del sistema 
$$\expval{E} = \frac{\int dqdp \, H(q,p) \exp \left[ -\beta H(q,p) \right]}{\int dqdp \exp \left[ -\beta H(q,p) \right]}$$
relación que puede escribirse en forma más compacta observando que la integral del numerador puede ponerse fácilmente en función de la integral del denominador, que es precisamente la función de partición. En efecto
$$\int dqdp \, H(q,p) \exp \left[ -\beta H(q,p) \right] = - \int dqdp \, \pdv{\beta} \exp \left[ -\beta H(q,p) \right] = - h^f\left( \pdv{Z}{\beta} \right)_{X_\alpha}$$
y, por tanto,
\begin{equation}
	\expval{E} = -\frac{1}{Z} \left( \pdv{Z}{\beta} \right)_{X_\alpha} = - \left( \pdv{\ln Z}{\beta} \right)_{X_\alpha}
\end{equation}
donde se ha indicado explícitamente que las derivadas parciales se calculan a parámetros externos constantes.

Análogamente, para el valor medio de la fuerza generalizada $\expval{Y_\alpha}$ asociada con el parámetro externo $X_\alpha$
\begin{equation}
	\expval{Y_\alpha} =  \frac{\int dqdp \, -\pdv{H}{X_\alpha} \exp \left[ -\beta H(q,p) \right]}{\int dqdp \exp \left[ -\beta H(q,p) \right]} = - \frac{1}{\beta} \left( \pdv{\ln Z}{X_\alpha} \right)_{T}
\end{equation}
donde ahora se mantiene constante la temperatura.
En el caso concreto del volumen $X_\alpha = V$, esta relación proporciona una expresión para la presión media, que debe
considerarse como una \emph{ecuación de estado}.

También se pueden expresar fácilmente mediante la función de partición las dispersiones. En efecto, sabemos que
$$(\Delta E)^2 = \expval{E^2}  - \expval{E} ^2$$
y
\begin{eqnarray}
	\expval{E^2} = \frac{\int dqdp \, [H(q,p)]^2 \exp \left[ -\beta H(q,p) \right]}{\int dqdp \exp \left[ -\beta H(q,p) \right]} = -\frac{1}{Z^2} \left( \frac{\partial^2 Z}{\partial \beta^2} \right)_{X_\alpha}
\end{eqnarray}
que puede reescribirse como 
$$\expval{E^2} = \pdv{\beta} \left( \frac{1}{Z} \pdv{Z}{\beta} \right)_{X_\alpha} + \frac{1}{Z} \left( \pdv{Z}{\beta} \right)_{X_\alpha}^2$$
y sustituyendo se obtiene
\begin{equation}
	(\Delta E)^2 = \pdv{\beta} \left( \frac{1}{Z} \pdv{Z}{\beta} \right)_{X_\alpha} = \left( \pdv[2]{\ln Z}{\beta} \right)_{X_\alpha} = - \left( \pdv{\expval{E}}{\beta} \right)_{X_\alpha}
\end{equation}

Obtendremos finalmente una importante propiedad de las funciones de partición de ciertos tipos de sistemas.
Consideremos un sistema $A$ compuesto de dos sistemas $A_1$ y $A_2$ débilmente interaccionantes (como siempre, en el sentido de que la energía de interacción es \emph{despreciable} frente a la energía de cualquiera de los dos sistemas).
Sabemos que esta propiedad se traduce en la aditividad de las energías, o sea de los hamiltonianos:
\begin{equation}
	H(q,Q,p,P) = H_1(q,p) + H_2(Q,P)
\end{equation}
donde hemos utilizado la misma notación que en el capítulo anterior.
La función de partición del sistema total viene dada por
\begin{align}
	Z &= \frac{1}{h^{f_1+f_2}} \int dqdQdpdP \, \exp \left[ -\beta H(q,p,Q,P) \right] \nonumber \\
	  &= \frac{1}{h^{f_1+f_2}} \int dqdQdpdP \, \exp \left[ -\beta  (H_1(q,p) + H_2(Q,P)) \right] \\
	  &= \left[ \frac{1}{h^{f_1}} \int dqdp \, \exp \left[ -\beta H_1(q,p) \right]\right]  \left[ \frac{1}{h^{f_2}} \int dQdP \, \exp \left[ -\beta H_2(Q,P) \right] \right] \nonumber
\end{align}
o sea
\begin{equation}
	\boxed{Z = Z_1 Z_2}
\end{equation}
donde $Z_1$ y $Z_2$ son las funciones de partición de $A_1$ y $A_2$, respectivamente.
Es decir, que la función de partición del sistema total es igual al producto de las funciones de las partes débilmente interaccionantes.
Esta propiedad de la función de partición es la que garantiza la propiedad de aditividad de las funciones termodinámicas, ya que éstas dependen del logaritmo de la función de partición que cumplirá para sistemas débilmente interaccionantes
\begin{equation}
	\ln Z = \ln Z_1 + \ln Z_2
\end{equation}

\section{Conexión con la termodinámica}

La conexión entre la colectividad canónica y la Termodinámica puede establecerse de un modo directo, análogamente a como se procedió para la colectividad microcanónica.
En efecto, recordando la dependencia de $Z$ respecto de la temperatura y los parámetros externos, tenemos que
\begin{equation}
	d\ln Z(E,X_\alpha) = \left( \pdv{\ln Z}{\beta} \right)_{X_\alpha} d\beta + \sum_{\alpha} \left( \pdv{\ln Z}{X_\alpha} \right)_{E} dX_\alpha
\end{equation}

Consideremos un proceso cuasiestático en el que $\beta$ y $X_\alpha$ varían tan lentamente que el sistema esté prácticamente en equilibrio, y por consiguiente sea lícito suponerlo distribuido en todo momento de acuerdo con la distribución canónica. En este caso,
\begin{equation}
	d\ln Z(E,X_\alpha) = \expval{E} d\beta + \beta \sum_{\alpha} \left\langle Y_\alpha \right\rangle  dX_\alpha
\end{equation}

Si queremos encontrar una expresión para la entropía hemos de buscar una diferencial exacta que provenga de $đQ$ mediante un factor integrante que ha de ser el inverso de la temperatura absoluta $T$. Para ello basta sumar $d(\beta E)$ a los dos miembros de la igualdad anterior con lo que resulta
\begin{align}
	d\ln Z(E,X_\alpha + \beta E) &= d(\beta E) - \expval{E} d\beta + \beta \sum_{\alpha} \left\langle Y_\alpha \right\rangle  dX_\alpha \nonumber \\
					 			 &=\beta \left(  \expval{E} + \beta \sum_{\alpha} \left\langle Y_\alpha \right\rangle  dX_\alpha \right)  \\
					 			 &=\beta \left( \expval{E} + đW \right) = \beta đQ \nonumber
\end{align}

De esta ecuación deducimos para la entropía
\begin{equation}
	\boxed{S = k_B (\ln Z  + \beta \expval{E})}
\end{equation}

Esta definición se corresponde con las definiciones de la entropía en la colectividad microcanónica, y por las mismas razones discutidas entonces no resulta adecuada cuando se considera la dependencia, respecto de los números de partículas de cada una de las especies que constituyen el sistema.
El mismo tipo de razonamientos utilizado allí nos lleva a la expresión correcta
\begin{equation}
	S = k_B \ln \left( \frac{Z}{\prod_i N_i!}  + \beta \expval{E} \right)
\end{equation}

\section{Gas ideal monoatómico}

A fin de aclarar ideas, vamos a aplicar el colectivo canónico a un caso concreto que ya estudiamos con el microcanónico.
Consideremos un gas ideal monoatómico en equilibrio, encerrado en un recipiente de volumen $V$ a la temperatura $T$.
El hamiltoniano del sistema es, llamando $N$ al número de partículas
$$H(p,q) = \sum_{i=1}^N \frac{p_i^2}{2m}$$

La función de partición del sistema será
\begin{align}
	Z &= \frac{1}{h^{3N}N!} \int d^3r_1 \cdots d^3r_N d^3p_1 \cdots d^3p_N \exp \left[ -\beta \frac{p_1^2 + \cdots + p_N^2}{2m}\right] \nonumber \\
	  &=  \frac{1}{h^{3N}N!} \int d^3p_1\exp \left[ -\beta \frac{p_1^2}{2m} \right] \cdots \int d^3p_N\exp \left[ -\beta \frac{p_N^2}{2m} \right] \underbracket{\int d^3r_1 \cdots d^3r_N}_{V^N} 
\end{align}
la integración respecto de las cantidades de movimiento es un producto de $N$ integrales, idénticas todas salvo en la notación de la variable de integración, e iguales a
$$\int d^3\mathbf{p} \exp \left[ -\beta \frac{\mathbf{p}^2}{2m} \right]$$

En resumen, $Z$ se transforma en un producto de la forma
\begin{equation}
	\boxed{Z = \frac{\zeta^N}{N!}}
\end{equation}
donde
$$\zeta = \frac{V}{N} \int_{-\infty}^{\infty} d^3\mathbf{p} \exp \left[ -\beta \frac{\mathbf{p}^2}{2m} \right] $$
es la \emph{función de partición para una única molécula}. La integral que aparece en esta expresión tiene el valor
\begin{align}
	\int_{-\infty}^{\infty} d^3\mathbf{p} \exp \left[ -\beta \frac{\mathbf{p}^2}{2m} \right] &= \iiint_{-\infty}^{\infty} dp_x dp_y dp_z \exp \left[ -\beta \frac{p_x^2 + p_y^2 + p_z^2}{2m} \right] \nonumber \\
	&= \left( \int_{-\infty}^{\infty} dp \exp \left[ -\beta \frac{p^2}{2m} \right] \right) ^3 = \left( \sqrt{\frac{2\pi m}{\beta}} \right) ^3
\end{align}
y. así,
\begin{align}
	\ln Z &= N\ln \zeta - \ln N! = N(N\ln \zeta - \ln N - 1 ) \nonumber \\
	&=  N \left[ \ln \frac{V}{N} - \frac{3}{2}\ln \beta + \frac{3}{2}\ln \left( \frac{2\pi m}{h^2} \right)  + 1 \right] 
\end{align}

A partir de la función de partición podemos calcular todas las propiedades termodinámicas del sistema. La ecuación de estado la obtendremos a partir de la expresión de la presión media
\begin{equation}
	\expval{p} = \frac{1}{\beta} \left( \pdv{\ln Z}{V} \right)_{T,N} = \frac{1}{\beta} \frac{N}{V}
\end{equation}

Por tanto,
\begin{equation}
	\expval{p} V = Nk_BT
\end{equation}
la ecuación de estado ya obtenida.

La energía media será
\begin{equation}
	\expval{E} = - \left( \frac{\partial \ln Z}{\partial \beta}\right)_{T,V} = \frac{3}{2} \frac{N}{\beta} = \frac{3}{2} N k_B T
\end{equation}

\section{Teorema de equipartición generalizado}

Vamos a calcular el valor medio sobre la distribución canónica del producto
$$x_i\pdv{H}{x_j}$$
donde $x_i$ y $x_j$ pueden ser cualquier coordenada o momento generalizado del sistema.
Por definición, se tiene
\begin{equation}
	\expval{x_i\pdv{H}{x_j}}  = C^{-1} \int dqdp \, x_i\pdv{H}{x_j} e^{-\beta H}
\end{equation}
siendo
$$C = \int dqdp \, e^{-\beta H} = Z$$

Es fácil ver que esta igualdad puede escribirse como \footnote{Escribimos $q_1 \cdots q_f$ como $x_1 \cdots x_f$ y $p_1 \cdots p_f$ como $x_{f+1} \cdots x_{2f}$}
\begin{equation}
	\expval{x_i\pdv{H}{x_j}}  = C^{-1} \int dx_1\cdots dx_f d_{f+1} \cdots d_{2f} dx_i \, x_i\left( \frac{1}{\beta}\pdv{x_j} \right) e^{-\beta H}
\end{equation}

Consideremos en primer lugar la integración respecto de $x_j$ e integremos por partes
$$\int_{-\infty}^{\infty} dx_j x_i\pdv{x_j}e^{-\beta H} = \left[ x_ie^{-\beta H} \right]_{x_j = -\infty}^{x_j = \infty} - \int_{-\infty}^{\infty} dx_j e^{-\beta H}\pdv{x_i}{x_j}$$

Observemos que los límites de integración respecto de $x_j$ los hemos fijado en $-\infty$ y $+\infty$.
Es evidente que este es el caso para las cantidades de movimiento, y también para las coordenadas cartesianas, ya que por convenio introducimos siempre los términos de corte en el hamiltoniano.
Sin embargo, si las coordenadas generalizadas son de tipo angular no es posible esta extensión y nuestros resultados no serán aplicables.
Limitándonos al tipo de coordenadas fásicas indicado, el primer sumando del segundo miembro es nulo.
En efecto, el hecho de que los valores medios de las variables dinámicas sobre la distribución canónica hayan de ser finitos (y han de serlo por razones físicas evidentes) implica que
$$\lim\limits_{x_j\rightarrow \pm \infty} f(x_1,\ldots,x_{2f}) e^{-\beta H} = 0$$
siendo $f(x_1,\ldots,x_{2f})$ cualquier función que represente una variable del sistema (energía cinética, energía potencial, momento cinético, etc,), ya que en otro caso sería
$$\int dx_1\cdots x_{2f} \, f(x_1,\ldots,x_{2f})e^{-\beta H} \rightarrow \infty$$

Como caso particular, se tiene evidentemente que
$$\left[ x_ie^{-\beta H} \right]_{x_j = -\infty}^{x_j = \infty} = 0$$
y por otro lado,
$$\frac{\partial x_i}{\partial x_j} = \delta_{ij}$$
donde $\delta_{ij}$ es la \emph{delta de Kronecker}.

Finalmente, uniendo todos los resultados
\begin{equation}
	\expval{x_i\frac{\partial H}{\partial x_j}} = k_B T \delta_{ij}
\end{equation}
expresión que constituye el \emph{teorema de equipartición generalizado}. Aunque lo hemos demostrado para el caso de la distribución canónica, este teorema es válido también en el caso de la colectividad microcanónica.

Cuando el resultado se aplica a una coordenada generalizada $x_i = x_j = q_i$ toma la forma
\begin{equation}
	\expval{q_i\frac{\partial H}{\partial q_i}} = k_B T
\end{equation}
y se le denomina \emph{teorema del virial}, mientras que aplicado a un momento generalizado $x_i = x_j = p_i$
\begin{equation}
	\expval{p_i\frac{\partial H}{\partial p_i}} = k_B T
\end{equation}
y se le denomina \emph{teorema de equipartición}.Vamos a ver el significado físico de estos
teoremas, lo que además nos aclarará la razón de estas denominaciones.
A partir de la definición de hamiltoniano conocida de la Mecánica Analítica y de las ecuaciones de Hamilton, tenemos
\begin{equation}
	H = \sum p_i\dot{q_i} - L = \sum_i p_i\frac{\partial H}{\partial p_i} - L
\end{equation}
o sea
\begin{equation}
	\sum_i \frac{\partial H}{\partial p_i} = H + L = H + K - U = 2K
\end{equation}
ya que para el tipo de modelos mecánicos que estamos considerando sabemos que el hamiltoniano es la suma de la energía cinética y la potencial) es decir, $H = K + U$.
Esta expresión nos dice que la energía cinética asociada con cada grado de libertad $i$ vale
$$\frac{1}{2}p_i\frac{\partial H}{\partial p_i}$$
de forma que el teorema de equipartición afirma que, en valor medio, la energía cinética se reparte por igual entre todos los grados de libertad, correspondiéndole a cada uno de ellos un valor $1/2 k_B T$.





%--------------------------------------
%  Cap 3
%--------------------------------------

\chapter{Colectividad canónica generalizada}
\chead[\thechapter. \textsc{Colectividad canónica generalizada}]{\thechapter. \textsc{Colectividad canónica generalizada}}

Al discutir la paradoja de Gibbs y la expresión correcta de la entropía en Mecánica Estadística clásica , pusimos de manifiesto que una descripción completa del equilibrio de un sistema exige considerar también la dependencia de las distintas magnitudes respecto del número de partículas que componen el sistema.
Siguiendo en esa línea de razonamiento, vamos a generalizar la distribución canónica a fin de poder describir estadísticamente el equilibrio de sistemas abiertos o heterogéneos.

\newpage
\section{Colectividad canónica generalizada. La gran función de partición}

Consideremos un sistema en contacto con un foco térmico en el sentido especificado en el capítulo anterior, pero ahora vamos a suponer además que el foco térmico actúa también como un \emph{foco de partículas} con el que el sistema puede intercambiar partículas.
En seguida precisaremos el concepto de foco de partículas, que es totalmente análogo al de foco térmico. 
Al sistema total  lo supondremos aislado y en equilibrio, de manera que le es aplicable la distribución microcanónica.
Utilizaremos la misma notación que siempre, caracterizando con el subíndice 1 el sistema en consideración $A_1$,
con el subíndice 2 el foco $A_2$ y dejando sin subíndices las magnitudes que se refieran al sistema total $A$.

La densidad de probabilidad de encontrar al sistema $A_1$ en un microestado tal que contenga $N_1$ partículas dadas, con posiciones comprendidas en el intervalo $(q, q + dq)$ y cantidades de movimiento dentro del margen $(p, p+dp)$, viene dada por ---en las expresiones de $\rho$ fijamos primero el número de partículas y después las coordenadas y momentos generalizados, pues éste es el orden lógico; además, por sencillez, consideramos un único tipo de partículas---
\begin{equation}
	\rho^0(N_1,q,p) = \frac{1}{h^{f_1} \Omega(E)} \Omega_2 \left[ E - H_{N_1}(q,p;X_\alpha), N-N_1 \right]
\end{equation}

De acuerdo con lo que hemos visto en el Capítulo 1, el segundo miembro de esta igualdad no depende de cuáles sean las $N_1$ partículas seleccionadas, si admitimos que todas ellas son iguales.
Consecuentemente pasamos a considerar, al igual que hicimos allí, la densidad de probabilidad de encontrar $N_1$ partículas cualesquiera, que evidentemente será
\begin{align}
	\rho(N_1,q,p) &= \frac{N!}{N_1!(N-N_1)!}\rho^0(N_1,q,p) \nonumber \\ 
				  &= \frac{N!}{N_1!(N-N_1)!} \frac{\Omega_2 \left[ E - H_{N_1}(q,p;X_\alpha), N-N_1 \right]}{h^{f_1} \Omega(E)}
\end{align}

Esta expresión puede escribirse en la forma compacta
\begin{equation}
\rho(N_1,q,p) = B\frac{1}{h^{f_1} \Omega(E)} \frac{\Omega_2 \left[ E - H_{N_1}(q,p;X_\alpha), N-N_1 \right]}{(N-N_1)!}
\end{equation}
siendo $B$ independiente de $N_1$, $q$ y $p$.

Ahora bien, hemos visto que esta densidad de probabilidad es prácticamente nula excepto cuando corresponda a valores muy próximos a $\widetilde{N_1}$ y $\widetilde{E_1}$ determinados por ---actuando de la misma manera que en la colectividad canónica---.

Además, dado el carácter de foco del sistema $A_2$ es decir, dado que se trata de un sistema con muchos más grados de libertad que $A_1$, los razonamientos sobre órdenes de magnitud efectuados reiteradamente nos muestran que en la región en que $\rho$ es distinta de cero se tiene
$$E_1 \ll E \quad | \quad N_1 \ll N$$

Podemos entonces desarrollar el último factor del segundo miembro de $\rho$ reteniendo únicamente el primer orden ---recordar que $N_2 = N - N_1$---
$$\ln \frac{\Omega_2 \left[ E - H_{N_1}(q,p;X_\alpha), N-N_1 \right]}{h^{f_1} \Omega(E)} = \ln \frac{\Omega_2 \left[ E \right] }{N!} - \left( \pdv{E_2} \ln\frac{\Omega_2}{N_2!}\right)_{\substack{E_2 = E \\ N_2 = N}} \mkern-8mu H_{N_1} + \left( \pdv{N_2} \ln\frac{\Omega_2}{N_2!}\right)_{\substack{E_2 = E \\ N_2 = N}} \mkern-8mu N_1$$

La derivada
\begin{equation}
	\beta = \left( \pdv{E_2} \ln\frac{\Omega_2}{N_2!}\right)_{\substack{E_2 = E \\ N_2 = N}}
\end{equation}
define la temperatura del foco como ya discutimos, mientras que
\begin{equation}
-\beta \mu = \left( \pdv{N_2} \ln\frac{\Omega_2}{N_2!}\right)_{\substack{E_2 = E \\ N_2 = N}}
\end{equation}
define su potencial químico $\mu$.

Obsérvese que la derivada está calculada en $E$ y $N$ no en $\widetilde{E}$ y $\widetilde{N}$. Este hecho es el que nos va a permitir precisar más el concepto de foco de partículas.
Cuando al poner en contacto dos sistemas, el potencial químico de uno de ellos permanezca esencialmente invariable cualquiera que sean los intercambios de energía y partículas entre ellos, se dice que este sistema actúa como \emph{foco de partículas} respecto del otro.

Con estas identidades resulta
\begin{equation}
	\ln \frac{\Omega_2 \left[ E - H_{N_1}(q,p;X_\alpha), N-N_1 \right]}{h^{f_1} \Omega(E)} = \frac{\Omega_2(E,N)}{N!} \exp\left[ \beta H_{N_1}(q,p) + \beta\mu N_1 \right]
\end{equation}
con lo que obtenemos
\begin{equation}
	\rho(N_1,q,p) = C\frac{1}{h^{f_1}N_1!}\exp\left[ \beta \left( H_{N_1}(q,p) - \mu N_1\right)  \right]
\end{equation}
donde
$$C = B\frac{\Omega_2(E,N)}{N!}$$
y es independiente de $N_1$, $q$ y $p$. A veces esta expresión se escribe de la forma
\begin{equation}
	\rho(N_1,q,p) = C\frac{1}{h^{f_1}N_1!}\exp\left[ \beta H_{N_1}(q,p) - \alpha N_1  \right]
\end{equation}
donde hemos introducido
\begin{equation}
	\alpha = \beta\mu = \left( \pdv{N} \ln\frac{\Omega}{N!}\right)_{\substack{E = \widetilde{E} \\ N = \widetilde{N}}}
\end{equation}

La magnitud $z \equiv e^{-\alpha}$ recibe el nombre de \emph{fugacidad}.

La constante $C$ puede determinarse por la condición de normalización sumando para todos los posibles números de partículas y todas las posibles configuraciones
$$C^{-1} = \sum_{N_1 = 0}^{\infty} \frac{e^{\mu\beta N_1}}{h^{f_1} N_1!} \int dqdp \, \exp\left[ -\beta H_{N_1}(q,p) \right]$$
y por extensión de la función de partición, introducimos la \emph{función de partición generalizada} o \textbf{\emph{gran función de partición}} definida como
\begin{equation}
	C^{-1} \equiv  \boxed{Q = \sum_{N = 0}^{\infty} \frac{e^{-\alpha N}}{h^{f} N!} \int dqdp \, \exp\left[ -\beta H_{N}(q,p) \right]}
\end{equation}
habiendo prescindido del subíndice 1. Esta distribución de probabilidades se denomina \textit{distribución macrocanónica}, \textit{gran canónica} o \textit{canónica generalizada} y está normalizada de forma que
$$\sum_{N = 0}^\infty \int dqdp \, \rho(N; q,p) = 1$$

Podemos también escribir la distribución de probabilidades para la energía en este caso de una colectividad canónica generalizada, para lo cual no habrá más que integrar $\rho$ para todos los valores de $q$ y $p$ que hacen $H_N(q, p) = E$.
El resultado es
\begin{equation}
	\omega(N; E) = \Omega(N; E) \frac{1}{N!}\frac{1}{Q} \exp\left[ -\beta H_{N_1}(q,p) \right]
\end{equation}

La generalización al caso de varios tipos de partículas es trivial.
Las expresiones en este caso general toman la forma
\begin{equation}
	Q = \sum_{N_1 = 0}^\infty \cdots \sum_{N_s = 0}^\infty \frac{e^{-\sum_{i=1}^s \alpha_i N_i}}{h^{f_1 + \cdots + f_s} N_1!\cdots N_s!} \int dqdp \, \exp\left[ -\beta\left(  H_{N_1}(q,p) + \cdots + H_{N_s} \right) \right]
\end{equation}
siendo
$$\alpha_i = \left( \pdv{N_i} \ln \frac{\Omega}{N_i!} \right)_{\substack{E = \widetilde{E} \\ N_j = \widetilde{N_j}}} $$
y $s$ el número de clases de partículas distintas que existen en el sistema.

Para no complicar la notación no hemos desdoblado explícitamente el conjunto de variables $(q, p)$, pero desde luego tendrá la forma
\begin{align}
	(q, p) = (&q_1, \ldots ,q_{N_1},q_{N_1 + 1}, \ldots,q_{N_1 + N_2} ,q_{N_1 + N_2 +1},\ldots; \nonumber\\
	&p_1, \ldots, p_{N_1},p_{N_1 + 1}, \ldots,p_{N_1 + N_2} ,p_{N_1 + N_2 +1},\ldots)
\end{align}

En los razonamientos que siguen utilizaremos indistintamente la distribución correspondiente a un solo tipo de partículas y la general, pues todos los resultados obtenidos mediante la primera se generalizan sin dificultad a sistemas multicomponentes.

\section{Cálculo de valores medios y fluctuaciones}

Las expresiones de los valores en el conjunto canónico generalizado son completamente análogas a las obtenidas para la distribución canónica. Comenzamos por la energía media
\begin{align}
	\expval{E} &= \sum_{N = 0}^\infty \int dqdp \, H_N(q,p) \rho(N; q,p) \nonumber\\
							     &= \frac{1}{Q} \sum_{N = 0}^\infty \frac{e^{-\alpha N}}{h^f N!}\int dqdp \, H_N(q,p) e^{-\beta H_N(q,p)} \\
							     &= - \frac{1}{Q} \left( \pdv{Q}{\beta} \right)_{\alpha, X} = \left( \pdv{\ln Q}{\beta} \right)_{\alpha, X} \nonumber
\end{align}
Debe observarse, para evitar posibles confusiones, que esta derivada parcial \textbf{debe calcularse a $\bm{\alpha}$ constante, y no a potencial químico constante}; si la derivada fuese a $\mu$ constante la expresión sería
\begin{equation}
	\expval{E} = \left( \pdv{\ln Q}{\beta} \right)_{\mu, X} - \mu \expval{N}
\end{equation}
donde $\expval{N}$ es el número medio de partículas que calcularemos un poco más adelante.

Siguiendo la misma línea de razonamientos se demuestra que
$$\expval{E^2} = \frac{1}{Q} \left( \pdv[2]{Q}{\beta} \right)_{\alpha, X}$$
de forma análoga al resultado de la colectividad canónica.

Así, la dispersión de la energía será\footnote{Obsérvese que la expresión para la dispersión de la energía es similar a la que se obtuvo en el colectivo canónico, pero hay una diferencia fundamental.
Anteriormente la derivada se tomaba, aparte de a parámetros externos constantes, a $N$ constante ---en el colectivo canónico $N$ es un parámetro fijo---, mientras que en ahora la derivada se toma a $\alpha$ constante.}
\begin{align}
	(\Delta E)^2 = \expval{E^2} - \expval{E}^2 &= \sum_{N = 0}^\infty \int dqdp \, H_N(q,p) \rho(N; q,p) \nonumber\\
		&= \frac{1}{Q} \left( \pdv[2]{Q}{\beta} \right)_{\alpha, X} -  \left( \pdv{\ln Q}{\beta} \right)_{\alpha, X}^2 \\
		&= \left( \pdv[2]{\ln Q}{\beta} \right)_{\alpha, X} = - \left( \pdv{\expval{E}}{\beta} \right)_{\alpha, X} \nonumber
\end{align}

Ahora podemos calcular el valor medio del número de partículas. Viene dado por
\begin{align}
	\expval{N} &= \sum_{N = 0}^\infty \int dqdp \, N \rho(N; q,p) \nonumber \\
								&= \frac{1}{Q} \sum_{N = 0}^\infty \frac{e^{-\alpha N}}{h^f N!}\int dqdp \, e^{-\beta H_N(q,p)} \\
								&= - \frac{1}{Q} \left( \pdv{Q}{\alpha} \right)_{\beta, X} = \left( \pdv{\ln Q}{\alpha} \right)_{\beta, X} \nonumber
\end{align}
y de la misma manera que con la energía el segundo orden se obtiene fácilmente
\begin{equation}
	\expval{N^2} = \frac{1}{Q} \left( \pdv[2]{Q}{\alpha} \right)_{\beta, X}
\end{equation}
y la dispersión será
\begin{align}
(\Delta N)^2 &= \expval{N^2} - \expval{N}^2 =  \left( \pdv[2]{\ln Q}{\alpha} \right)_{\beta, X} \nonumber\\ 
			&= \left( \pdv{\expval{N}}{\beta} \right)_{\beta, X} = k_B T \left( \pdv{\expval{N}}{\mu} \right)_{\beta, X}
\end{align}

Por último, podemos calcular los momentos generalizados
\begin{align}
\expval{Y_\alpha} &= - \sum_{N = 0}^\infty \int dqdp \, \pdv{H}{X_k} \rho(N; q,p) \nonumber \\
					&= - \frac{1}{Q} \sum_{N = 0}^\infty \frac{e^{-\alpha N}}{h^f N!}\int dqdp \, \pdv{H}{X_k} e^{-\beta H_N(q,p)} \\
					&= - \frac{1}{Q} \frac{1}{\beta} \left( \pdv{Q}{X_k} \right)_{\alpha, \beta, X_{i \neq k}} = \frac{1}{\beta} \left( \pdv{\ln Q}{X_k} \right)_{\alpha, \beta, X_{i \neq k}} \nonumber
\end{align}

\section{Relación con la termodinámica}

La gran función de partición será función en general de los parámetros externos, de las $\alpha_i$ y de la temperatura.
Por lo tanto
$$d\ln Q = \left( \pdv{Q}{\beta} \right)_{\alpha_i, X_k} d\beta + \sum_i \left( \pdv{Q}{\alpha_i} \right)_{\alpha_{j\neq i}, X_k, \beta} d\alpha_i + \sum_k \left(  \pdv{Q}{X_k} \right)_{\alpha_i, X_{i \neq k}, \beta} dX_k$$
y aprovechando los resultados anteriores
\begin{equation}
	d\ln Q = -\expval{E} d\beta + \sum_i \expval{N_i} d\alpha_i + \sum_k \expval{Y_k} dX_k
\end{equation}

Si queremos hallar una expresión para la entropía hemos de buscar, igual que hicimos en el capítulo anterior, una diferencial exacta que provenga de $\dj Q$\footnote{Por el contexto se deduce cuándo $Q$ se refiere al calor o a la gran función de partición.} mediante el inverso de la temperatura como factor integrante.
Es fácil obtener que
\begin{equation}
	d\left( \ln Q  + \beta \expval{E} -\sum_i \alpha_i \expval{N_i} \right) = \beta \left( d\expval{E} - \sum_i \alpha_i  d\expval{N_i} + \dj W \right)
\end{equation}
donde hemos introducido
$$\dj W = \sum_k \expval{Y_k} dX_k$$

Considerando que la variación de la energía de un sistema puede provenir de la realización de un trabajo, de un intercambio de partículas, o de una transferencia de calor, parece lógico identificar
$$\dj Q = d\expval{E} - \sum_i \alpha_i  d\expval{N_i} + \dj W$$
y, consecuentemente, identificamos
\begin{equation}\label{eq:S_T3}
	S = k_B \left( \ln Q  + \beta \expval{E} -\beta \sum_i \mu_i \expval{N_i} \right)
\end{equation}

\begin{center}
	\rule[0.5ex]{5em}{0.55pt}
\end{center}

Definamos ahora la función de Gibbs
\begin{equation}
	G = \expval{E} - TS + \sum_k \expval{Y_k} dX_k \label{eq:Gibb}
\end{equation}
cuya diferencial es
\begin{equation}
	dG = -SdT + \sum_k X_k d\expval{Y_k} + \sum_i \mu_i d\expval{N_i}
\end{equation}

Si comparamos esta expresión con la expresión formal de la diferencial de $G$ considerada como función de $T$, $\expval{Y_k}$ y $\expval{N_i}$
$$dG = \left( \pdv{G}{T} \right)_{\expval{Y_k}, N_i} dT + 
	\sum_k \left( \pdv{G}{Y_k} \right)_{T, \expval{Y_{i\neq k}}, \expval{N_i}} d\expval{Y_k} + 
	\sum_i \left( \pdv{G}{N_i} \right)_{T,\expval{Y_k}, \expval{N_{j\neq i}}} d\expval{N_i}$$
obtenemos
\begin{equation}
	\mu_i = \left( \pdv{G}{N_i} \right)_{T,\expval{Y_k}, \expval{N_{j\neq i}}}
\end{equation}

Como $G$ por definición es una magnitud extensiva, ha de ser una función homogénea de primer grado en las variables extensivas independientes de que dependa.
Ahora bien, en el conjunto de variables $T$, $\expval{Y_k}$ y $\expval{N_i}$ las únicas variables extensivas son las $\expval{N_i}$, por lo que, aplicando el teorema de Euler de las funciones homogéneas, ha de ser
\begin{equation}
	G = \sum_i \left( \pdv{G}{N_i}  \right)_{T,\expval{Y_k}, \expval{N_{j\neq i}}} d\expval{N_i} = \sum_i \mu_i d\expval{N_i}
\end{equation}
y con la definición del potencial de Gibbs
\begin{equation}
	\expval{E} - TS + \sum_k \expval{Y_k} dX_k = \sum_i \mu_i d\expval{N_i}
\end{equation}
y la relación con la entropía que obtuvimos antes
\begin{equation}
	\sum_k \expval{Y_k} dX_k = k_B T \ln Q
\end{equation}

Esta ecuación es muy utilizada en la práctica, pues proporciona de un modo directo la ecuación de estado. En el caso de un sistema hidroestático simple, en el que el único parámetro externo es el volumen, se convierte en
\begin{equation}
	\expval{p} V = k_B T \ln Q \label{eq:pvGb}
\end{equation}

Para un gas ideal monoatómico, se obtiene que $\ln Q = \expval{N}$, como debía ser a fin de que se convierta en la ecuación de Clapeyron.

\begin{center}
	\rule[0.5ex]{5em}{0.55pt}
\end{center}

Vamos a introducir ahora una magnitud termodinámica que está directamente relacionada con el logaritmo de la gran función de partición, en lugar de con sus derivadas como sucede con la energía interna o la entropía.
Definimos el llamado \emph{gran potencial} como ($F$ es la energía libre de Helmholtz)
\begin{equation}
	\Phi = \expval{E} - TS - \sum_i \mu_i \expval{N_i} = F - \sum_i \mu_i \expval{N_i} \label{eq:Phi}
\end{equation}
cuya diferencial es ---obsérvese que las variables independientes del gran potencial son la temperatura, los parámetros externos y los potenciales químicos---
\begin{align}
	d\Phi &= d\expval{E} - TdS - SdT - \sum_i \mu_i d\expval{N_i} - \sum_i  \expval{N_i} d\mu_i \nonumber \\
	      &= -SdT - \sum_k \expval{Y_k} dX_k - \sum_i  \expval{N_i} d\mu_i
\end{align}

Comparando esta expresión con la expresión formal de la diferencial de $\Phi$ considerada como función de $T$, $X_k$ y $\mu_i$
$$d\Phi = \left( \pdv{\Phi}{T} \right)_{\expval{Y_k}, \mu_i}dT - 
	\sum_k \left( \pdv{\Phi}{X_k} \right)_{T,\expval{X_{i\neq k}}, \expval{\mu_i}} dX_k -
	\sum_i  \left( \pdv{\Phi}{\mu_i} \right)_{T,\expval{Y_k}, \expval{\mu_{j\neq i}}} d\mu_i$$
obtenemos
$$S = - \left( \pdv{\Phi}{T} \right)_{\expval{Y_k}, \mu_i} | \quad
	\expval{Y_k} = \left( \pdv{\Phi}{X_k} \right)_{T,\expval{X_{i\neq k}}, \expval{\mu_i}} | \quad
	\expval{N_i} = \left( \pdv{\Phi}{\mu_i} \right)_{T,\expval{Y_k}, \expval{\mu_{j\neq i}}} $$
	
Por otro lado, de la comparación entre \eqref{eq:S_T3} y \eqref{eq:Phi} deducimos
\begin{equation}
	\Phi = -k_B T \ln Q
\end{equation}
que, con \eqref{eq:pvGb} pasa a ser
\begin{equation}
	\Phi = -\expval{p} V
\end{equation}






%--------------------------------------
%  Cap 4
%--------------------------------------

\chapter{Sistemas ideales}
\chead[\thechapter. \textsc{Sistemas ideales}]{\thechapter. \textsc{Sistemas ideales}}

Los modelos mecánicos más sencillos que se pueden considerar son aquellos en que se desprecian las interacciones entre las partículas que los componen.
Más exacta mente, lo que se hace es despreciar los valores de las energías de interacción frente a la energía cinética o potencial ---debida a un campo externo--- de las partículas, pues como ya hemos señalado en el Capítulo 1, las energías de interacción son conceptualmente imprescindibles para poder garantizar la tendencia del sistema a un estado de equilibrio.

A estos modelos se les denomina ideales y, aun cuando solo representan el comportamiento de los sistemas físicos reales en condiciones muy extremas, su estudio es de gran utilidad, ya que por un lado presentan aspectos comunes con los sistemas reales y por otro al poderse resolver exactamente sirven de punto de partida para el estudio de modelos en los que se tienen explícitamente en cuenta las interacciones y que por ello son matemáticamente muy complejos.
Este capítulo está dedicado al estudio de modelos ideales, mientras que en el siguiente consideraremos, desde un punto de vista elemental, el caso de los gases reales, en los que hay que tener en cuenta las interacciones.
El modelo ideal por excelencia lo constituye el gas ideal, del cual ya tratamos al estudiar el colectivo microcanónico. 

El objetivo fundamental de este capítulo es el análisis de este modelo en el colectivo canónico, pero profundizando mucho más en sus propiedades. 
No obstante, al final del mismo consideraremos otro modelo ideal: la llamada \emph{teoría clásica del paramagnetismo}.

\newpage

\section{Distribución de velocidades de Maxwell}

Consideremos de nuevo un gas ideal en equilibrio, pero ahora, con objeto de dar la mayor generalidad posible a los resultados, consideraremos al gas constituido por varios tipos distintos de moléculas, a las que, además, no exigiremos que sean monoatómicas.
Representaremos por $\mathbf{r}$ y $\mathbf{p}$ la posición y momento del centro de masas de una molécula y por $(q_{int}, p_{int})$ las coordenadas y momentos generalizados asociados con los grados internos de libertad. Admitiendo que no existen campos externos, el hamiltoniano de una molécula dada es igual a\footnote{No utilizaremos subíndices para caracterizar el tipo de molécula de que se trata, ya que como veremos en nuestros razonamientos, solo es necesario considerar cada clase de moléculas separadamente aunque los resultados son válidos para todas ellas.}
\begin{equation}
	H = \frac{\mathbf{p}^2}{2m} + H^{int}(q_{int}, p_{int})
\end{equation}
donde $H^{int}$ representa la energía interna de rotación y vibración de los átomos que componen la molécula respecto de su centro de masas.
Este término es nulo si la molécula es monoatómica.
Además, al suponer que el gas es ideal estamos admitiendo que $H^{int}$ es independiente de $\mathbf{r}$.

Supongamos que el sistema se encuentra en equilibrio a la temperatura absoluta $T$ y fijemos nuestra atención en una molécula dada.
Como la energía del sistema es igual a la suma de las energías de cada una de sus partículas, se cumple la condición de aditividad enunciada en el capítulo 2 y podemos considerar a todas las moléculas restantes como un foco térmico a la temperatura absoluta $T$.
La distribución de estados de la molécula obedecerá la distribución canónica, y, por consiguiente, la probabilidad $P(\mathbf{r}, \mathbf{p}, q_{int}, p_{int}) \, d^3 \mathbf{r} \, d^3 \mathbf{p} \, dq_{int} dp_{int}$ de encontrar la molécula con variables del centro de masas en los intervalos $(\mathbf{r}, \mathbf{r} + d\mathbf{r})$ y $(\mathbf{p}, \mathbf{p} +d\mathbf{p})$ y con un estado interno correspondiente a los intervalos $(q_{int}, q_{int} + dq_{int})$ y $(P_{int}, P_{int} + dP_{int})$ viene dada por\footnote{Nótese que no se trata de una probabilidad en el espado de las fases del sistema completo ---espacio $\Gamma$---, sino de una probabilidad en el espacio de las fases de una molécula, también denominado espacio $\mu$}
\begin{align} \label{eq:Prob4}
	& P(\mathbf{r}, \mathbf{p}, q_{int}, p_{int}) \, d^3 \mathbf{r} \, d^3 \mathbf{p} \, dq_{int} dp_{int} \nonumber \\
	& \propto \exp \left[ -\beta\left( \frac{\mathbf{p}^2}{2m} + H^{int} \right)  \right] \, d^3 \mathbf{r} \, d^3 \mathbf{p} \, dq_{int} dp_{int} \\
	&= \left[ \exp \left[ -\beta\frac{\mathbf{p}^2}{2m} \right] \, d^3 \mathbf{r} \, d^3 \mathbf{p} \right] \left[ \exp \left[ -\beta H^{int} \right] \, dq_{int} dp_{int} \right] \nonumber
\end{align}

La probabilidad $P(\mathbf{r}, \mathbf{p}) d^3 \mathbf{r} \, d^3 \mathbf{p}$ de encontrar la molécula con variables del centro de masas en los intervalos $(\mathbf{r}, \mathbf{r} + d\mathbf{r})$ y $(\mathbf{p}, \mathbf{p} +d\mathbf{p})$, independientemente del estado interno se obtendrá integrando \eqref{eq:Prob4} respecto de todos los posibles valores de $q_{int}$ y $p_{int}$. Teniendo en cuenta la forma de \eqref{eq:Prob4} resulta evidentemente
\begin{equation}\label{eq:Prob4.3}
	P(\mathbf{r}, \mathbf{p}) \, d^3 \mathbf{r} \, d^3 \mathbf{p} \propto \exp \left[ -\beta\frac{\mathbf{p}^2}{2m} \right] \, d^3 \mathbf{r} \, d^3 \mathbf{p}
\end{equation}
siendo la constante de proporcionalidad independiente de todas las coordenadas y cantidades de movimiento generalizadas.

Si multiplicamos la probabilidad \eqref{eq:Prob4.3} por el número total de moléculas $N$ del tipo que estamos considerando que existen en el sistema, se obtendrá el número medio de moléculas de esa clase en ese intervalo de posiciones y cantidades de movimiento.
Definamos para el tipo de moléculas en consideración
\begin{align}
	f(\mathbf{r}, \mathbf{v}) \, d^3 \mathbf{r} \, d^3 \mathbf{v} \equiv & \text{ número medio de moléculas cuyo centro de masas} \nonumber \\
				& \text{ ocupa una posición dentro del intervalo}  \\
				& \text{ (\textbf{r}, \textbf{r} + d\textbf{r}) y tiene una velocidad entre \textbf{v} y \textbf{v} + d\textbf{v}}. \nonumber
\end{align}

La función $f(\mathbf{r}, \mathbf{v})$ recibe el nombre de \emph{función de distribución de una partícula} o
simplemente \emph{función de distribución}.
Obsérvese, que utilizamos como variable la velocidad $\mathbf{v} = \sfrac{\mathbf{p}}{m}$ del centro de masas en lugar del momento $\mathbf{p}$.
A partir de \eqref{eq:Prob4.3} obtenemos
\begin{equation}
f(\mathbf{r}, \mathbf{v}) \, d^3 \mathbf{r} \, d^3 \mathbf{v} = C N \exp \left[ -\beta\frac{\mathbf{p}^2}{2m} \right] \, d^3 \mathbf{r} \, d^3 \mathbf{v}
\end{equation}

La constante $C$ puede determinarse por la condición de normalización
\begin{equation}
\int d^3 \mathbf{r} \int d^3 \mathbf{v} \, f(\mathbf{r}, \mathbf{v}) = N
\end{equation}
o sea
\begin{equation}
	C N \int d^3 \mathbf{r} \int d^3 \mathbf{v} \exp \left[ -\beta\frac{m\mathbf{v}^2}{2} \right] = N \label{eq:Norm4}
\end{equation}

Como el integrando no depende de $\mathbf{r}$, la integración respecto de esta variable da simplemente $V$, volumen del sistema.
Con esto \eqref{eq:Norm4} se reduce a 
\begin{equation}
	C V \int d^3 \mathbf{v} \exp \left[ -\beta\frac{m\mathbf{v}^2}{2} \right] = C V \left[ \int dv_x \exp \left[ -\beta\frac{mv_x^2}{2} \right]\right]^3 = CV \left( \frac{2\pi}{m\beta} \right) ^{\frac{3}{2}} = 1
\end{equation}
Despejando la constante $C$ resulta
\begin{equation}
	C = \frac{1}{V} \left( \frac{m\beta}{2\pi} \right) ^{\frac{3}{2}}
\end{equation}
y, por lo tanto,
\begin{equation}\label{eq:f_def}
	f(\mathbf{r}, \mathbf{v}) \, d^3 \mathbf{r} \, d^3 \mathbf{v} = n \left( \frac{m\beta}{2\pi} \right) ^{\frac{3}{2}} \exp \left[ -\beta\frac{m\mathbf{v}^2}{2} \right] \, d^3 \mathbf{r} \, d^3 \mathbf{v}
\end{equation}
siendo $n$ el número medio de moléculas del tipo considerado por unidad de volumen, o sea
$$n = \frac{N}{V}$$

Dado que $f$ no depende de $\mathbf{r}$, podemos omitir esta variable en el argumento de $f$ y escribir \eqref{eq:f_def} como
\begin{equation}\label{eq:Maxw}
f(\mathbf{v}) \, d^3 \mathbf{r} \, d^3 \mathbf{v} = n \left( \frac{m}{2\pi k_B T} \right) ^{\frac{3}{2}} \exp \left[ -\beta\frac{m\mathbf{v}^2}{2} \right] \, d^3 \mathbf{r} \, d^3 \mathbf{v}
\end{equation}

Esta es la famosa \emph{distribución de velocidades de Maxwell} para un gas diluido en equilibrio.
Obsérvese que $f$ tampoco depende de la dirección de $\mathbf{v}$, sino únicamente de su módulo.
La independencia de la función de distribución del sistema respecto del vector de posición y de la dirección de la velocidad traduce las propiedades de homogeneidad e isotropía del sistema.

\section{Cálculo de valores medios y fluctuaciones}

A partir del conocimiento de $f(\mathbf{v})$ es fácil determinar otras funciones de distribución y ciertas propiedades de los gases ideales. En primer lugar podemos obtener fácilmente la función de distribución de una componente de la velocidad, $g(v_x)$, $g(v_y)$ o $g(v_z)$, que se define del modo siguiente
\begin{align}
	g(v_x) \, dv_x \equiv & \text{ número medio de moléculas, por unidad de volumen,} \nonumber \\
		& \text{ tienen una velocidad \textbf{v}, cuya componente $x$ } \nonumber\\
		& \text{ tiene un valor comprendido entre $v_x$ y $v_x + dv_x$, }  \\
		& \text{ independientemente del valor que tomen las otras } \nonumber \\
		& \text{ dos componentes en este caso $v_y$ y $v_z$} \nonumber
\end{align}

Como es natural $g(v_y)$ y $g(v_z)$ se definen de modo semejante para las componentes respectivas de la velocidad vectorial \textbf{v}.

El cálculo de $g(v_x)$ es sencillo a partir de $f(\mathbf{v})$, si nos damos cuenta que para ello basta sumar todas las moléculas, por unidad de volumen, que tiene una componente $v_x$ en el intervalo considerado, independientemente de cuál sea el valor de las componentes $v_y$ y $v_z$.
En la práctica, esta suma se realiza integrando $f(\mathbf{v})$ para todos los valores de $v_y$ y $v_z$.
Por tanto, es
\begin{align}
	g(v_x)dv_x &= \frac{1}{V} \int d^3 \mathbf{r} \int_{v_y} \int_{v_z} d^3 \mathbf{v} \, f(\mathbf{v}) \nonumber \\
			   &= n \left( \frac{m}{2\pi k_B T} \right) ^{3/2} e^{-\beta\frac{mv_x^2}{2 k_B T}} dv_x \int_{-\infty}^{\infty} dv_y e^{-\beta\frac{mv_y^2}{2 k_B T}} \int_{-\infty}^{\infty} dv_z e^{-\beta\frac{mv_z^2}{2 k_B T}}
\end{align}

Las dos últimas integrales son iguales y cada una de ellas vale $(2\pi k_B T/m)^{1/2}$.
Sustituyendo y simplificando se obtiene
\begin{equation}\label{eq:gvx}
	g(v_x) \, dv_x = n \left( \frac{m}{2\pi k_B T} \right) ^{1/2} \exp \left[ -\beta\frac{mv_x^2}{2 k_B T} \right] dv_x
\end{equation}

Resulta entonces que la distribución de probabilidades de una componente de la velocidad es una distribución de Gauss de valor medio nulo
\begin{equation}
	\left\langle v_x \right\rangle = n \left( \frac{m\beta}{2\pi} \right) ^{1/2} \int dv_x \, v_x \exp \left[ -\beta\frac{mv_x^2}{2 k_B T} \right] = 0
\end{equation}
ya que nos encontramos con un integrando impar.
Así, la dispersión será
\begin{equation}
	(\Delta v_x)^2 = \left\langle v_x^2 \right\rangle = n \left( \frac{m\beta}{2\pi} \right) ^{1/2} \int dv_x \, v_x^2 \exp \left[ -\beta\frac{mv_x^2}{2 k_B T} \right] = \frac{k_B T}{m}
\end{equation}

Este último resultado coincide desde luego con el obtenido en el capítulo anterior aplicando el teorema de equipartición, los mismos resultados son válidos si sustituimos $v_x$ por $v_y$ o $v_z$, de manera que comparando \eqref{eq:Maxw} con \eqref{eq:gvx} podemos observar que
\begin{equation}
	\left[ \frac{f(\mathbf{v}) \, d^3 \mathbf{v}}{n} \right]  = \left[ \frac{g(v_x)dv_x}{n} \right] \left[ \frac{g(v_y)dv_y}{n} \right] \left[ \frac{g(v_z)dv_z}{n} \right] 
\end{equation}
lo cual expresa la independencia estadística de las componentes de la velocidad de una partícula.
La razón por la que aparece un divisor $n$ en cada una de las distribuciones es que no están normalizadas a la unidad, sino al número de partículas $N$, o si se prefiere podemos decir que están normalizados a $n$ cuando se integran respecto de las velocidades
\begin{equation}
	\int_{-\infty}^{\infty} dv_x \, g(v_x) = \int d^3 \mathbf{v} \, f(\mathbf{v}) = n
\end{equation}

Otra distribución de gran interés, relacionada con el módulo de la velocidad $v = |\mathbf{v}|$, es
\begin{align}
	F(v) \, dv \equiv & \text{ número medio de moléculas por unidad de volumen} \nonumber \\
		& \text{ con una velocidad cuyo módulo $v$ está comprendido}  \\
		& \text{ entre $v$ y $v$ + d$v$}. \nonumber
\end{align}

Este número se obtendrá sumando todas las moléculas por unidad de volumen con velocidades cuyo módulo esté en este intervalo, independientemente de la dirección.
Así, pues, será
\begin{equation}
	F(v)dv = \frac{1}{V} \int_{-\infty}^{\infty} d^3\mathbf{r} \int\limits_{v \le | \mathbf{v} | \le v+dv} \mkern-10mu d^3\mathbf{v} \, f(\mathbf{v}) = \int\limits_{v \le | \mathbf{v} | \le v+dv} \mkern-10mu d^3\mathbf{v} \, f(\mathbf{v})
\end{equation}

Como $f(\mathbf{v})$ depende únicamente de $\mathbf{v}$ resulta que es constante en todo el volumen (del espacio de velocidades) de integración. Este volumen vale $4\pi v^2 dv$ y, por lo tanto
\begin{equation}
	F(v)dv = 4\pi f(v) v^2 dv
\end{equation}
que utilizando \eqref{eq:f_def} toma la forma explícita
\begin{equation}
	F(v)dv = 4\pi n \left( \frac{m\beta}{2\pi} \right) ^{3/2}  v^2 \exp \left[ -\beta\frac{mv_x^2}{2} \right] dv
\end{equation}

Esta distribución presenta un máximo cuya existencia puede ponerse de manifiesto mediante un razonamiento cualitativo del tipo que hemos realizado repetidamente.
Cuando $v$ aumenta, el factor exponencial disminuye mientras que el factor $v^2$ aumenta; el resultado es un máximo en la distribución para un cierto valor $\widetilde{v}$ del módulo de la velocidad, denominada \emph{velocidad más probable}.
Para calcularla seguimos el procedimiento usual de igualar a cero la primera derivada
\begin{equation*}
	\left( \frac{d F(v)}{dv} \right)_{v=\widetilde{v}} = 0
\end{equation*}
es decir,
\begin{equation}
	2\widetilde{v} \exp \left[ -\frac{mv_x^2}{2 k_B T} \right] +\widetilde{v}^2 \frac{m\widetilde{v}}{2 k_B T} \exp \left[ -\frac{m\widetilde{v}^2}{2 k_B T} \right] = 0
\end{equation}
de donde
\begin{equation}
	\widetilde{v}^2 = 2\frac{k_B T}{m} \Rightarrow \widetilde{v} = \sqrt{2\frac{k_B T}{m}}
\end{equation}

\section{Teoría clásica del paramagnetismo}

Consideremos un modelo de gas ideal paramagnético corno un conjunto de dipolos magnéticos iguales, no interaccionantes entre sí, de momento magnético $\mu$.
El sistema se encuentra dentro de un campo magnético de intensidad $\mathbf{B}$ de forma que la energía potencial de cada dipolo vale
\begin{equation}\label{eq:u_def}
	u = -\mu \mathbf{B} = -\mu B \cos\theta
\end{equation}
donde el ángulo $\theta$ puede variar de un modo continuo de 0 a $2\pi$.

Como la interacción entre los dipolos magnéticos se considera despreciable, podemos fijar nuestra atención sobre uno de ellos, considerando entonces el resto como un foco térmico a la temperatura absoluta $T$.
Como además se cumple la condición de aditividad de la energía, podemos utilizar la distribución canónica, análogamente a lo que hicimos en el caso del gas ideal.
La probabilidad $P(\theta, \varphi) d\Omega$ de que la orientación del dipolo respecto del campo magnético aplicado esté comprendida dentro de un elemento de ángulo sólido $d\Omega = \sin \theta d\theta d\varphi$ alrededor de la dirección definida por $\theta$ y $\varphi$ se obtendrá integrando la distribución canónica respecto de todos los valores posibles de las restantes variables fásicas. 
El resultado es, evidentemente,
\begin{equation}
	P(\theta, \varphi) d\Omega \propto e^{-\beta u} d\Omega
\end{equation}

El valor medio de $\cos \theta$ vendrá entonces dado por
\begin{equation}
	\left\langle \cos \theta \right\rangle = \frac{\int \cos \theta d\Omega e^{-\beta u}}{\int d\Omega e^{-\beta u}}
\end{equation}
donde teniendo en cuenta los valores de $u$ y de $d\Omega$ resulta
\begin{equation}
	\left\langle \cos \theta \right\rangle = \frac{2\pi \int_0^{\pi} d\theta \sin \theta \cos \theta \, e^{\beta \mu B \cos\theta}}{2\pi \int_0^{\pi} d\theta \sin \theta \, e^{\beta \mu B \cos\theta}}
\end{equation}
y efectuando el cambio de variables $x = \cos \theta$,
\begin{equation}\label{eq:cos_med}
	\left\langle \cos \theta \right\rangle = \frac{\int_{-1}^{+1} x e^{\alpha x} dx}{\int_{-1}^{+1} e^{\alpha x} dx}
\end{equation}
donde
$$\alpha = \mu\beta B$$

Las integrales que aparecen en \eqref{eq:cos_med} se calculan fácilmente obteniéndose
\begin{equation}\label{eq:lang}
	\left\langle \cos \theta \right\rangle = \coth \alpha - \frac{1}{\alpha} \equiv L(\alpha)
\end{equation}

La función $L(\alpha)$ definida en \eqref{eq:lang} se denomina \emph{función de Langevin}, más ampliada en el \hyperref[Anx3]{Anexo 3}.

El momento magnético por unidad de volumen o \emph{imanación} vendrá dado si $n$ es el número de dipolos magnéticos por unidad de volumen, por\footnote{Obsérvese que solo aparece imanación en la dirección del campo, ya que los valores medios de $\sin \varphi$ y $\cos \varphi$ son nulos.}
\begin{equation}
	 M = n\mu \left\langle \cos \theta \right\rangle = n\mu L(\alpha)
\end{equation}

Para campos magnéticos muy intensos y temperaturas bajas $\alpha$ es muy grande, $\coth \simeq 1$ y $\sfrac{1}{\alpha}$ puede despreciarse.
En este caso, $L(\alpha) = 1$ y
\begin{equation}
	M(\alpha \gg 1) = n\mu
\end{equation}
que es la \emph{imanación de saturación} y, como es evidente, corresponde a todos los dipolos orientados paralelamente al campo ($	\left\langle \cos \theta \right\rangle = 1$).

En cambio, en campos débiles y altas temperaturas $\alpha$ es pequeño, $\coth \alpha \simeq \sfrac{1}{\alpha} + \sfrac{\alpha}{3} + \cdots$ y por tanto, $L(\alpha) \simeq \sfrac{\alpha}{3}$, con lo que
\begin{equation}
	M(\alpha \ll 1) = \frac{n\mu\alpha}{3} = \frac{n\mu^ 2B}{3 k_B T}
\end{equation}

Haciendo $B = \mu_0 H$, lo cual representa una excelente aproximación en el caso de las sustancias paramagnéticas, donde la imanación es muy pequeña frente al campo aplicado, resulta
\begin{equation}
	M = \frac{n\mu^2\mu_0}{3 k_B}\frac{H}{T}
\end{equation}
que comparada con la \emph{ley de Curie} de las sustancias paramagnéticas
\begin{equation}
	M = C\frac{H}{T}
\end{equation}
nos da una expresión para la \emph{constante de Curie}
\begin{equation}
	C = \frac{n\mu^2\mu_0}{3 k_B}
\end{equation}

Como hemos visto, el parámetro característica de la teoría es
$$\alpha = \beta\mu B = \frac{\mu B}{k_B T}$$
que teniendo en cuenta \eqref{eq:u_def} y el teorema de equipartición de la energía viene a ser del orden del cociente entre la energía magnética máxima y la energía de agitación térmica.
El límite $\alpha \gg 1$ corresponde entonces al caso en que la energía de agitación térmica es pequeña, los dipolos pueden alinearse paralelamente al campo con facilidad y la imanación es máxima.
Al ir aumentando la temperatura la energía térmica de los dipolos va aumentando y es más difícil orientarlos, es decir, hacen falta campos magnéticos más intensos: la imanación va disminuyendo.
Estos razonamientos explican físicamente los resultados obtenidos.





%--------------------------------------
%  Cap 5
%--------------------------------------

\chapter{Sistemas reales}
\chead[\thechapter. \textsc{Sistemas reales}]{\thechapter. \textsc{Sistemas reales}}

Hasta ahora hemos considerado únicamente sistemas ideales, despreciando todo tipo de interacción entre las partículas.
En este capítulo trataremos de ir un poco más allá analizando el comportamiento estadístico de sistemas constituidos por partículas que interaccionan entre sí.
Desde luego, el problema es de una gran complejidad matemática, siendo en general irresoluble, por lo que nos limitaremos a la consideración de un caso especialmente sencillo, aunque de un gran interés práctico: el modelo de gas real diluido.

Es importante previamente entender bien el modo de proceder ante un problema en Mecánica Estadística y en general en toda la Física.
A partir de un sistema real se construye un modelo teórico del mismo.
Este modelo viene caracterizado por una cierta descripción matemática que, en nuestro caso, es la función de partición.
Y sucede que, aunque formalmente esa descripción sea conocida, su evaluación explícita no suele ser matemáticamente posible, lo que lleva a introducir simplificaciones sobre el modelo.
Este es el punto crucial en el desarrollo de la teoría, pues es importante saber delimitar si las simplificaciones son puramente matemáticas y no tienen ninguna incidencia sobre los resultados desde un punto de vista físico, o si por el contrario limitan el dominio de validez de los resultados al ser éstos aplicables únicamente a casos o conofrimientos particulares del sistema físico considerado.
La teoría de los gases reales constituirá un claro ejemplo del modo de proceder que acabamos de indicar.

Sin embargo, no siempre es posible prever a priori las limitaciones que una cierta hipótesis simplificadora impone sobre la validez de los resultados teniéndose que esperar a la comparación con los datos experimentales ---o un desarrollo más \emph{fino} de la teoría--- para comprobar su rango de validez.

\newpage
\section{Función de partición configuracional}

Consideremos un modelo de gas monoatómico cuyas moléculas interaccionan entre sí.
Si eliminamos de nuestras consideraciones gases polarizados o ionizados) parece lógico admitir que el potencial entre dos moléculas sea de tipo central, esto es, que dependa únicamente de la distancia que separa a las dos moléculas.
Existen multitud de potenciales del tipo indicado que han sido propuestos para describir la interacción entre moléculas; algunos de ellos con un cierto fundamento teórico, pero todos esencialmente empíricos, de manera que su justificación se encuentra en la mayor o menor concordancia con los resultados experimentales de los resultados obtenidos a partir de ellos.
Su forma cualitativa es la indicada en la figura adyacente: son fuertemente repulsivos a cortas distancias, como consecuencia del volumen finito de las moléculas, y presentan para distancias mayores una zona atractiva que tiende a cero al aumentar la separación entre las moléculas.

\colorbox{red!60}{\textcolor{white}{\textit{[Esquemas, ya veré si los dejo aquí o más adelante]}}}

El hamiltoniano de un sistema de $N$ partículas interaccionantes y sin estructura interna, si admitimos que la energía potencial total es la suma de las energías potenciales asociadas a todos los pares de moléculas,\footnote{Es conveniente detenerse a pensar en este punto para convencerse de que, efectivamente, se trata de una hipótesis. Cuando una partícula se coloca en las proximidades de otras dos que están interaccionando, no solo añadirá nuevos términos a la energía de interacción, sino que en general podrá perturbar el campo de fuerzas de las dos primeras partículas.} tendrá la forma
\begin{equation}
	H(q,p) = \sum_{i=1}^N \frac{p_i^2}{2m} + \sum_{1 \leq i < j}^{N} u(|\mathbf{r}_i - \mathbf{r}_j|)
\end{equation}
o, abreviadamente,
\begin{equation}
	H = K + U
\end{equation}
siendo $K$ la energía cinética (que depende solo de las cantidades de movimiento) y $U$ la energía potencial (que depende solo de las posiciones).

De acuerdo con la teoría de la Mecánica Estadística que hemos venido desarrollando, y utilizando el colectivo canónico, sabemos que todas las propiedades termodinámicas del sistema se obtienen a partir de su función de partición
\begin{align}\label{eq:Z_5}
	Z &= \frac{1}{h^{3N} N!} \int dqdp \, \exp \left[ -\beta H(q,p) \right] \nonumber \\
	  &= \frac{1}{h^{3N} N!} \int d^3\mathbf{r}_1 \cdots d^3\mathbf{r}_N d^3\mathbf{p}_1 \cdots d^3\mathbf{p}_N \, \exp \left[ -\beta H(q,p) \right] 
\end{align}

Dada la independencia de la energía potencial respecto de las cantidades de movimiento y de la energía cinética respecto de las posiciones, podemos descomponer \eqref{eq:Z_5} en la forma
\begin{equation}\label{eq:Z_desc}
	Z = Z_T Z_U
\end{equation}
siendo
\begin{equation}
	Z_T = \frac{1}{h^{3N} N!} \int d^3\mathbf{p}_1 \cdots d^3\mathbf{p}_N \, \exp \left[ -\beta K \right] 
\end{equation}
y
\begin{equation}\label{eq:Z_u}
	Z_U = \frac{1}{h^{3N} N!} \int d^3\mathbf{r}_1 \cdots d^3\mathbf{r}_N \, \exp \left[ -\beta U \right]
\end{equation}

El término $Z_T$, llamada función de partición traslacional, es fácil de integrar; de hecho coincide salvo un factor $V^N$ con la función de partición de un gas ideal monoatómico calculada previamente en la sección 2.4, y vale
\begin{equation}\label{eq:Z_t}
	Z_T = \frac{1}{h^{3N} N!} (2\pi m k_B T)^{3N/2}
\end{equation}

Resulta entonces que el cálculo de la función de partición queda reducido al cálculo de $Z_U$, que se suele denominar \emph{función de partición configuracional} por tratarse de una integral extendida al espacio de configuraciones. Es evidente que en el caso de un gas ideal $(U = 0)$ se obtiene $Z_U = V^N$.
Sin embargo, en el caso general y para potenciales de interacción razonablemente reales ---en el sentido de estar acordes con la experiencia---, el cálculo de $Z_U$ es extremadamente complicado o imposible.
Como consecuencia resulta necesaria la consideración de situaciones físicas concretas que permitan realizar aproximaciones, tal y como expusimos en la introducción de este capítulo.

Existen desde luego numerosos procedimientos formales para el cálculo de \eqref{eq:Z_u} basados en métodos perturbativos, es decir, en desarrollos más o menos sistemáticos en potencias de un cierto parámetro que usualmente es la densidad o el potencial de interacción.
No vamos a exponer aquí en detalle ninguno de estos procedimientos que por su complejidad matemática salen fuera de nuestros objetivos, pero analizaremos cuidadosamente el contenido físico de las aproximaciones que es necesario realizar a fin de obtener el valor de $Z_U$ que permita calcular los términos correspondientes a los órdenes más bajos en la densidad.
Los resultados que obtengamos serán entonces aplicables a un gas \emph{suficientemente diluido}.
El significado de esta expresión lo irán precisando los razonamientos y aproximaciones que realicemos.

\section{Desarrollo en la densidad}

Centremos nuestra atención en dos partículas dadas, que llamaremos $i$ y $j$, de nuestro sistema de $N$ partículas encerradas en un recipiente de volumen $V$.
Nos preguntamos cuál será la probabilidad de que el sistema se encuentre en una configuración tal que estas dos partículas interaccionen entre sí. De las consideraciones del apartado anterior se deduce que esta probabilidad será igual a la probabilidad de que $j$ se encuentre a una distancia menor que $r_0$ de la partícula $i$, o dicho de otro modo, de que se encuentre dentro de la esfera del volumen $4\pi r_0^3/3$ y centro en la partícula $i$.
Evidentemente la presencia de la partícula $i$ influirá sobre la probabilidad de presencia de $j$ y en particular una distancia menor que $\sigma$ le será prácticamente inaccesible.
Podemos sin embargo realizar un interesante estudio cualitativo.

Vamos a admitir, y esta es nuestra primera hipótesis, que la parte atractiva del potencia es muy débil comparada con la cinética media de todas las partículas, es decir, que $mv^2/2 \gg 1$. Es claro que por el teorema de equipartición esto equivalente	a admitir que la temperatura del sistema no es muy baja.
Con esta hipótesis evitamos la posibilidad de que se formen agregados de partículas al ser atrapadas unas 	en el potencial de las otras, es decir, admitimos que las partículas siempre acaban separándose después de la interacción.

Imaginemos ahora que mentalmente eliminamos del sistema la partícula $i$ y consideramos una esfera de radio $r_0$ alrededor de su posición. Este \emph{hueco} tenderá a ser ocupado por un cierto número de partículas, que podemos considerar que son las que colisionan con la partícula $i$ al tratar de ir a ocupar su esfera de influencia.

Con estas consideraciones y dada la homogeneidad del sistema en el equilibrio podemos escribir que la probabilidad de que una partícula dada $j$ interaccione con la $i$ es del orden de
\begin{equation}\label{eq:prob_ji}
	\frac{r_0^3}{V}
\end{equation}
donde hemos prescindido de un factor numérico constante que no juega ningún papel en nuestras consideraciones.

El número medio de colisiones binarias simultáneas en el sistema, o sea, el número medio de pares de partículas interaccionantes en los microestados, se obtendrá multiplicando \eqref{eq:prob_ji} por el número de posibles parejas $i$, $j$ que puedan formarse y que es $N ( N - 1)/2$, que puede aproximarse a $N^2 /2$, ya que $N \gg 1$.
Resulta entonces que
\begin{equation}\label{eq:n_2col}
	\parbox{12em}{Número medio de colisiones\\binarias simultáneas} \sim \frac{N ( N - 1)}{2}\frac{r_0^3}{V} \simeq \frac{N}{2}nr_0^3
\end{equation}
donde hemos introducido la densidad numérica de partículas $n = N/V$. Obsérvese que \eqref{eq:n_2col} es una propiedad extensiva, es decir, proporcional a N, lo que refleja el hecho de que si duplicamos el sistema pasando de $N$ a $2N$ y de $V$ a $2V$ ---lo que mantiene $n$ constante--- el número medio de colisiones binarias simultáneas en el sistema se duplica.

La generalización de los razonamientos a colisiones triples, cuádruples, etc. es evidente.
Así, la probabilidad de que tres partículas dadas estén interaccionando
entre sí será del orden de
\begin{equation}
	\left( \frac{r_0^3}{V}\right)^2
\end{equation}
y
\begin{equation}\label{eq:n_3col}
	\parbox{12em}{Número medio de colisiones\\triples simultáneas} \sim  \frac{N (N-1)(N-2)}{3!}\left( \frac{r_0^3}{V}\right)^2 \simeq \frac{N}{6}(nr_0^3)^2
\end{equation}

En general, para colisiones simultáneas de $p$ partículas obtendríamos\footnote{Alguien podrá observar que \eqref{eq:n_pcol} no es válida si $p$ no es mucho menor que $N$. El análisis detallado de este hecho exigiría una discusión más profunda de la que hemos realizado aquí de los fundamentos de la Mecánica Estadística y en particular del llamado \emph{límite termodinámico}. En todo caso, para nuestros actuales razonamientos, basta observar que la expresión «exacta», manteniendo $N(N - 1) ... (N - p + 1)$, sería menor que \eqref{eq:n_pcol}.}
\begin{equation}\label{eq:n_pcol}
	\parbox{12em}{Número medio de colisiones\\simultáneas de $p$ partículas} \sim  \frac{N}{p!}(nr_0^3)^{p-1}
\end{equation}

Vamos a considerar ahora un gas muy diluido en el que
$$nr_0^3 \ll 1$$
condición que se entiende más claramente si observamos que es equivalente a
$$\frac{V}{N} \gg r_0^3$$
o sea, que el volumen por molécula ha de ser mucho mayor que el volumen efectivo de interacción.
Desde luego, ésta es una hipótesis sobre el modelo con claras implicaciones físicas, de manera que restringirá el rango de validez de nuestros resultados de acuerdo con la discusión que realizamos en el apartado anterior.

Como consecuencia de la hipótesis que acabamos de efectuar resulta que \eqref{eq:n_3col} en particular y \eqref{eq:n_pcol} en general, con $p > 2$, serán despreciables frente a \eqref{eq:n_2col}, de manera que las configuraciones en donde se producen colisiones triples, cuádruples, etc. son altamente improbables frente a las que presentan únicamente colisiones binarias.

Podemos simplificar aún más el problema matemático utilizando el carácter aditivo de \eqref{eq:n_2col} ya señalado.
Vamos a considerar a efectos de cálculo un sistema en el que $\frac{N ( N - 1)}{2}\frac{r_0^3}{V}$ sea mucho menor que la unidad de manera que la probabilidad de que en el sistema se produzcan dos o más colisiones binarias simultáneas sea despreciable.
Para ello tomaremos un valor de $N$ suficientemente pequeño, pero manteniendo $n$ constante.
Como éste es el parámetro intensivo del sistema y no se ve afectado por la elección, parece claro que las propiedades termodinámicas intensivas no van a verse afectadas.
De hecho, sabemos que, debido a la propiedad de aditividad, las fórmulas que se obtengan para un sistema de tipo serán automáticamente válidas para un sistema arbitrariamente grande.\footnote{Podemos precisar un poco más la idea. Desde ahora, y hasta que encontremos la expresión de una magnitud extensiva, olvidaremos que $N$ es grande y tomaremos $\frac{N ( N - 1)}{2}\frac{r_0^3}{V} \ll 1$. Cuando tengamos una magnitud extensiva, si ésta presenta una dependencia correcta respecto de $N$, es decir proporcional a ella, admitiremos que esa expresión es válida con independencia de su valor y más concretamente de que se cumpla o no la condición arriba señalada.}
Es decir, que estamos en el caso de una simplificación matemática sin implicaciones físicas.

Para trasladar todas estas ideas a un formalismo matemático es conveniente introducir una función $f(r)$ definida como
\begin{eqnarray}\label{eq:Mayer}
	f(r) = e^{-\beta u(r)} - 1
\end{eqnarray}

Esta función se denomina \emph{función de Mayer} y su comportamiento en el caso de un potencial como el de Lennard-Jones se ha representado en la Fig. 5.3. También hemos representado en la Fig. 5.4 la forma de $f(r)$ para el caso de un potencial como el de la Fig. 5.2. \colorbox{red!60}{\textcolor{white}{\textit{[Figuras a rellenar :D]}}}

La característica importante de la función de Mayer es que ya no es finita para valores pequeños de $r$, como sucedía con el potencial, y, además $f(r) = 0$ cuando $u(r) = 0$, es decir, que $f(r) \simeq 0$ para $r > r_0$, donde $r_0$ es, como siempre, el parámetro que caracteriza el alcance del potencial.
Con ayuda de la definición \eqref{eq:Mayer} podemos ahora escribir
\begin{align}\label{eq:e_beta_u}
	e^{-\beta u(r)} &= \exp \left[ -\beta \sum_{1 \leq i < j}^{N} u(|\mathbf{r}_i - \mathbf{r}_j|) \right] = \prod_{1 \leq i < j}^{N} e^{-\beta u(|\mathbf{r}_i - \mathbf{r}_j|)} \nonumber\\
					&= \prod_{1 \leq i < j}^{N} (1+f_{ij})
\end{align}
donde
\begin{eqnarray}
	f_{ij} \equiv f(|\mathbf{r}_i - \mathbf{r}_j|) \equiv e^{-\beta u(|\mathbf{r}_i - \mathbf{r}_j|)} - 1 
\end{eqnarray}

El desarrollo de \eqref{eq:e_beta_u} tiene la forma
\begin{align}\label{eq:des_e_beta_u}
	&1 + \sum_{1 \leq i < j}^{N} f_{ij} + \sum_{1 \leq i < j}^{N} \, \sum_{1 \leq l < m}^{N} f_{ij} f_{lm} \nonumber \\
	&+ \sum_{1 \leq i < j}^{N} \, \sum_{1 \leq l < m}^{N} \, \sum_{1 \leq p < q}^{N} f_{ij} f_{lm} f_{pq} + \cdots
\end{align}
que, introducido en la función de partición configuracional \eqref{eq:Z_u} da
\begin{align}\label{eq:z_u_2}
	Z_U &= \int d^3\mathbf{r}_1 \cdots d^3\mathbf{r}_N + \sum_{1 \leq i < j}^{N} \int d^3\mathbf{r}_1 \cdots d^3\mathbf{r}_N\, f_{ij} \nonumber \\
		& \quad + \sum_{1 \leq i < j}^{N} \, \sum_{1 \leq l < m}^{N} \, \int d^3\mathbf{r}_1 \cdots d^3\mathbf{r}_Nf_{ij} f_{lm} + \cdots \nonumber \\
		&= V^N + V^{N-2} \int d^3\mathbf{r}_i d^3\mathbf{r}_j f(|\mathbf{r}_i - \mathbf{r}_j|)  \nonumber \\
		& \quad + V^{N-4} \sum_{1 \leq i < j}^{N} \,  \sum_{1 \leq l < m}^{N} \, \int d^3\mathbf{r}_i d^3\mathbf{r}_j d^3\mathbf{r}_l  d^3\mathbf{r}_m \, f(|\mathbf{r}_i - \mathbf{r}_j|) f(|\mathbf{r}_l - \mathbf{r}_m|) + \cdots
\end{align}

Vamos a analizar este desarrollo. Si tenemos en cuenta que $f(r) \neq 0$ equivale a $u(r) \neq 0$, podemos interpretar rápidamente los distintos términos del segundo miembro de \eqref{eq:z_u_2} del siguiente modo:
\begin{enumerate}
	\item El primero, $V^N$ representa la función de partición configuracional cuando se desprecian todas las interacciones entre las partículas. Formalmente, puede	obtenerse a partir de \eqref{eq:Z_u} haciendo $U = 0$, y desde luego coincide con el valor de $Z_U$ para un gas ideal.
	
	\item El segundo término representa una primera corrección en la que se consideran todas las aportaciones a $Z_U$ de aquellas configuraciones en las que colisionan dos partículas, pero solo dos partículas. Dicho de otro modo, de \emph{todas las configuraciones posibles} se consideran aquellas en las que dos partículas, y solo dos partículas, están situadas a una distancia menor que $r_0$. Según nuestras consideraciones anteriores, éstas son las aportaciones que queremos retener en el \emph{caso} de nuestro gas suficientemente diluido.
	
	\item El tercer término contiene dos tipos de correcciones según que consideremos los cuatro índices $i$, $j$, $l$, $m$, distintos, o que alguno de los dos primeros sea igual a alguno de los dos últimos.\footnote{Un momento de reflexión acerca de \eqref{eq:des_e_beta_u}, como desarrollo de \eqref{eq:e_beta_u}, es suficiente para convencerse de que tal posibilidad puede darse, pero que no es posible que los dos primeros índices $i$, $j$ sean iguales a los otros dos $l$, $m$.} La primera posibilidad corresponde a las configuraciones con dos interacciones binarias, mientras que la segunda representa un tipo de corrección debida a las colisiones triples.
\end{enumerate}

Análogamente se podrían ir interpretando todos los demás términos del desarrollo, que contendrán correcciones debidas a colisiones múltiples de más de dos partículas y a colisiones simultaneas.

\section{Segundo coeficiente del virial}

Con la hipótesis de gas suficientemente diluido efectuada en el apartado anterior y la consideración de colisiones binarias aisladas allí discutida, resulta que podemos considerar en el caso de nuestro sistema que
\begin{equation}\label{eq:z_u_3}
	Z_U = V^N + V^{N-2} \frac{N(N-1)}{2}\int d^3\mathbf{r}_1 d^3\mathbf{r}_2 \, f(|\mathbf{r}_1 - \mathbf{r}_2|)
\end{equation}

Vamos a efectuar un cambio de coordenadas en la integral, introduciendo coordenadas relativas y del centro de masas
\begin{align}
	\mathbf{r} &= \mathbf{r}_1 - \mathbf{r}_2 \nonumber \\
	\mathbf{R} &= \frac{\mathbf{r}_1 + \mathbf{r}_2}{2}
\end{align}
con lo que se tiene, utilizando coordenadas esféricas,
\begin{equation}\label{eq:int_zu}
	\int d^3\mathbf{r}_1 d^3\mathbf{r}_2 \, f(|\mathbf{r}_1 - \mathbf{r}_2|) = \int d^3\mathbf{R} d^3\mathbf{r} \, f(r) = 4\pi V \int_0^{\infty} dr \, f(r) r^2
\end{equation}

En este punto hemos de hacer notar que hemos utilizado la invariancia traslacional del sistema al extender a todo el volumen las integrales respecto de \textbf{r} y \textbf{R}.
La justificación está en que estamos interesados en las propiedades de volumen del sistema y no en los efectos de superficie, que de otro modo habrían de considerarse.
Por otro lado, es evidente que podemos extender la integral respecto de $r$ hasta infinito en \eqref{eq:int_zu}, ya que $f (r)$ se anula para distancias grandes comparadas con $r_0$.

Utilizando \eqref{eq:Z_desc}, \eqref{eq:Z_t}, \eqref{eq:z_u_3} y \eqref{eq:int_zu}, y en el caso de un gas no ideal, pero suficientemente diluido en el sentido que hemos precisado. tenemos que
\begin{align}\label{eq:Z_virial}
	Z &= \frac{1}{h^{3N} N!} (2\pi m k_B T)^{3N/2} \left[ 1 + \frac{N(N-1)}{2V} \int_0^{\infty} dr \, 4 \pi r^2 f(r) \right] \nonumber \\
	  &= Z^\text{ideal} \left[ 1 + \frac{N(N-1)}{2V} \int_0^{\infty} dr \, 4 \pi r^2 f(r) \right]
\end{align}
donde hemos introducido la función de partición de un gas ideal
\begin{equation}
	Z^\text{ideal} = \frac{1}{h^{3N} N!} (2\pi m k_B T)^{3N/2}
\end{equation}

La forma cualitativa de la función $f(r)$ nos permite hacer una estimación del orden de magnitud del segundo sumando dentro del corchete de \eqref{eq:Z_virial}.
En efecto, como hemos admitido que la parte atractiva del potencial es muy débil comparada con la energía cinética media de las moléculas, resulta que $f(r)$ es típicamente del orden de la unidad para valores de $r$ menores que $r_0$ y nula en otro caso y, por lo tanto,
\begin{equation}
	\frac{N(N-1)}{2V} \int_0^{\infty} dr \, 4 \pi r^2 f(r) \sim \frac{N(N-1)}{2V} r_0^3
\end{equation}

Pero este número es muy pequeño en nuestro modelo por la elección que hemos hecho de $N$ en el apartado anterior. Tomando entonces logaritmos en \eqref{eq:Z_virial} y teniendo en cuenta que $\ln(1 + x) \simeq x$, cuando $x \ll 1$, obtenemos
\begin{align}
	\ln Z &= \ln Z^\text{ideal} + \ln \left[ 1 + \frac{N(N-1)}{2V} \int_0^{\infty} dr \, 4 \pi r^2 f(r) \right] \nonumber \\
		&\simeq \ln Z^\text{ideal} + \frac{N(N-1)}{2V} \int_0^{\infty} dr \, 4 \pi r^2 f(r) \\
		&=  \ln Z^\text{ideal} + \frac{N^2}{2V} \int_0^{\infty} dr \, 4 \pi r^2 f(r) \nonumber
\end{align}
donde hemos despreciado, además, $N$ frente a $N^2$.
Podemos ya calcular a partir de esta expresión la ecuación de estado utilizando la relación
\begin{equation}
	\left\langle p \right\rangle = \frac{1}{\beta} \left( \frac{\partial \ln Z}{\partial V}\right)_{\beta}
\end{equation}
que recordando que para un gas ideal es
\begin{equation}
	\left\langle p \right\rangle = \frac{N k_B T}{V}
\end{equation}
nos da
\begin{equation}
	\left\langle p \right\rangle = \frac{N k_B T}{V} - \frac{N^2 k_B T}{2V} \int_0^{\infty} dr \, 4 \pi r^2 f(r)
\end{equation}
que puede escribirse también como
\begin{equation}\label{eq:gas_real}
	\frac{\left\langle p \right\rangle V}{N k_B T} = 1 + \frac{N}{V} B_2(T)
\end{equation}
con
\begin{equation}\label{eq:seg_coef}
	B_2(T) = -2\pi \int_0^{\infty} dr \, r^2 f(r)
\end{equation}

Los dos términos del segundo miembro de \eqref{eq:gas_real} son los dos primeros términos del llamado \emph{desarrollo del virial}, es decir, de un desarrollo en potencias de la densidad
\begin{equation}\label{eq:gas_real_des}
	\frac{\left\langle p \right\rangle V}{N k_B T} = 1 + \frac{N}{V} B_2(T) + \left( \frac{N}{V} \right)^2 B_3(T) + \cdots
\end{equation}

Así, \eqref{eq:seg_coef} es la expresión del \emph{segundo coeficiente del virial} para gas clásico.
Es importante hacer notar que nosotros hemos admitido implícitamente la existencia de este desarrollo en potencias de la densidad, o sea que el primer miembro de \eqref{eq:gas_real_des} es una función analítica de la densidad.
Esto parece cierto para la clase de sistemas que estamos considerando, pero no es cierto si, por ejemplo, el potencial de interacción no disminuye de un modo suficientemente rápido con la distancia.




%--------------------------------------
%  Parte II
%--------------------------------------

\part{Mecánica Estadística Cuántica}



%--------------------------------------
%  Cap 6
%--------------------------------------

\chapter{Estadísticas Cuánticas}
\chead[\thechapter. \textsc{Estadísticas Cuánticas}]{\thechapter. \textsc{Estadísticas Cuánticas}}

Es bien sabido que, en principio, una descripción correcta de los sistemas físicos requiere el uso de la Mecánica Cuántica, por lo que una formulación correcta de la Mecánica Estadística solo puede hacerse dentro del marco de las ideas mecánico-cuánticas, a pesar de que, en muchos casos, la Mecánica Clásica proporciona una excelente aproximación al problema, como hemos visto hasta ahora.

Así, aunque el esquema general de la Mecánica Estadística desarrollado en los capítulos anteriores sigue siendo válido cuando la descripción mecánica de las partículas que componen el modelo microscópico se realiza mediante la Mecánica Cuántica, la aplicación de los postulados y las leyes cuánticas tiene importantes implicaciones en las propiedades finales de los sistemas. Además: corno analizaremos seguidamente, la Mecánica Estadística Cuántica no solo recupera los resultados obtenidos con la Clásica, sino que también soluciona las dificultades conceptuales anteriormente encontradas.

De entre los resultados que se derivan de tina descripción cuántica de los sistemas físicos, los que se deducen de la indistinguibilidad de las partículas y las propiedades de simetría de las funciones de onda jugarán un papel esencial en lo que sigue, por lo que comenzaremos este capítulo con una breve recopilación de los mismos.

\newpage
\section{Función de particion de un gas cuántico ideal}

En este apartado y los siguientes vamos a considerar el modelo microscópico de un gas ideal cuando la descripción del mismo se realiza mediante la Mecánica Cuántica.
Como veremos, aparecen hechos peculiares que son consecuencia de la indistinguibilidad cuántica de las partículas.

Así, pues, consideremos un sistema de $N$ partículas idénticas cuyas fuerzas de interacción pueden despreciarse. Vamos a utilizar la siguiente nomenclatura:

\begin{itemize}
	\item Las propiedades que se refieran a una partícula las representaremos mediante letras minúsculas, de forma que escribiremos $r$ para el estado cuántico de una partícula, $\varepsilon_r$ para la energía de ese estado y $n_r$ para el número de partículas que se encuentran en el mismo estado $r$.
	
	\item Utilizaremos letras mayúsculas para las propiedades del sistema total.
	Así por ejemplo $E_R$ representa la energía del sistema total cuando se encuentra en el estado cuántico $R$.
\end{itemize}

Con esta notación son evidentes las relaciones
\begin{equation} \label{eq:E_R}
	E_R = \sum_r n_{r,R} \varepsilon_r
\end{equation}
y
\begin{equation}\label{eq:N_R}
	N_R = \sum_r n_{r,R}
\end{equation}
donde $n_{r,R}$ representa el número de partículas en el estado $r$, cuando el sistema se encuentra en el estado $R$.

Un estado cuántico $R$ de un sistema de partículas idénticas queda totalmente determinado si se conoce el número de partículas que se encuentran en cada estado $r$, no siendo posible especificar cuáles son las partículas que se encuentran concretamente en cada estado.
En consecuencia, un estado $R$ se puede especificar mediante el conjunto de números $\{n_r,R\}$ que se denominan \emph{números de ocupación}.

Como sabemos, las propiedades termodinámicas de cualquier sistema pueden determinarse evaluando su función de partición $Z$, definida como
\begin{equation}
	Z = \sum_R e^{-\beta E_R}
\end{equation}
que, utilizando \eqref{eq:E_R}, y teniendo en cuenta lo que acabamos de decir sobre la especificación de los estados, podrá escribirse
\begin{equation}\label{eq:Z_cuant}
	Z = \sum_{\substack{n_1, n_2, \ldots\\(\sum n_r =N)}} e^{-\beta (n_1 \varepsilon_1 + n_2 \varepsilon_2 + \cdots)} = \sum_{\substack{n_1, n_2, \ldots\\(\sum n_r =N)}} \exp \left[ -\beta \sum_r n_{r,R} \varepsilon_r \right] 
\end{equation}
ya que la suma sobre todos los estados $R$ se transformará en una suma extendida a todos los valores posibles de los números de ocupación, con la restricción de que su suma sea $N$ |número de partículas del sistema|, puesto que en la función de partición la suma se extiende sobre todos los estados accesibles a un sistema cerrado, esto es, con un número constante de partículas.

Sin embargo, el cálculo de la función de partición es bastante complicado, debido precisamente a la presencia de la condición restrictiva $\sum n_r =N$ Y resulta mucho más cómoda la utilización del \emph{conjunto canónico generalizado}, donde no existe dicha restricción.
Recordemos que las distintas colectividades son equivalentes en lo que se refiere a valores medios y propiedades termodinámicas, y que las diferencias aparecen en el problema específico de las fluctuaciones.

Así, pues, pasamos a considerar la gran función de partición que viene dada por
\begin{equation}
	Q = \sum_{N=0}^\infty e^{-\alpha N} \sum_{R}^{} {}^{(N)} e^{-\beta E_R} = \sum_{N=0}^\infty e^{-\alpha N_R -\beta E_R}
\end{equation}
donde, en la última expresión, el sumatorio respecto de $R$ se extiende sobre todos los estados posibles con cualquier número de partículas.
Utilizando \eqref{eq:E_R} y  \eqref{eq:N_R} podemos escribir
\begin{equation}
	Q = \sum_{N=0}^\infty \exp \left[ -\alpha \sum_r n_{r,R} -\beta \sum_r n_{r,R} \varepsilon_r \right] 
\end{equation}
o
\begin{equation}\label{eq:Q_cuant}
	Q = \sum_R \prod_r \exp \left[ -\alpha n_{r,R} -\beta n_{r,R} \varepsilon_r \right]  = \sum_R \prod_r \exp \left[  -(\alpha  + \beta \varepsilon_r) n_{r,R}\right]
\end{equation}

En esta expresión la suma para todos los estados $R$ equivale a sumar para todos los valores posibles del conjunto de números $\{n_r,R\}$ como en \eqref{eq:Z_cuant}, pero ahora sin restricción para la suma.
Dicho de otro modo, cada $n_r$ toma valores en \eqref{eq:Q_cuant} con independencia de cuáles sean los valores de los restantes números de ocupación, de manera que tenemos
\begin{equation}
	Q = \sum_{n_1, n_2, \ldots} \prod_r \exp \left[  -(\alpha  + \beta \varepsilon_r) n_r\right] = \sum_{n_1} \sum_{n_2} \cdots \prod_r \exp \left[  -(\alpha  + \beta \varepsilon_r) n_{r}\right]
\end{equation}

Es fácil ver ahora que esta expresión puede escribirse también en la forma
\begin{equation}\label{eq:Q_cuant2}
	Q = \prod_r \sum_{n=0}^{n_{max}} \exp \left[  -(\alpha  + \beta \varepsilon_r) n \right]
\end{equation}
donde $n_{max}$ representa el valor máximo que pueden tomar los números de ocupación.

\colorbox{red!60}{\textcolor{white}{\textit{[Aquí falta alguna cosilla, a saber si la hago]}}}

Ahora vamos a estudiar algunas propiedades formales de la gran función de partición en Mecánica Cuántica.

\subsection*{Número medio de partículas en un estado de partícula $\bm{r}$}
A partir de la definición general de valor medio tenemos
$$\expval{n_r} = \frac{\sum_R e^{-\alpha N_R -\beta E_R}}{Q} = \frac{1}{Q} \sum_R e^{-\alpha \sum_r n_{r,R} -\beta \sum_r n_{r,R} \varepsilon_r} = - \frac{1}{\beta} \frac{1}{q} \frac{\partial Q}{\partial \varepsilon_r}$$
o sea
\begin{equation}
	\expval{n_r} = - \frac{1}{\beta} \frac{\partial \ln Q}{\partial \varepsilon_r}
\end{equation}

\subsection*{Cálculo de la presión}

La expresión puede deducirse ---a gusto del lector ;)---, obteniéndose
\begin{equation}
	\left\langle p \right\rangle = \frac{1}{\beta} \left( \frac{\partial \ln Q}{\partial V} \right)_{\alpha, \beta}
\end{equation}

Si ahora utilizamos la expresión de $Q$, \eqref{eq:Q_cuant}, resulta
\begin{align*}
	\left\langle p \right\rangle &= \frac{1}{\beta Q} \sum_R \left\lbrace \left[ -\beta \sum_r n_{r,R} \left( \frac{\partial \varepsilon_r}{\partial V} \right) \right] \exp \left( -\alpha \sum_r n_{r,R} -\beta \sum_r n_{r,R} \varepsilon_r \right)  \right\rbrace \\
	&= \frac{1}{Q} \sum_R  \sum_r \left( \frac{\partial \varepsilon_r}{\partial V} \right) \sum_R n_{r,R} \exp \left( -\alpha \sum_r n_{r,R} -\beta \sum_r n_{r,R} \varepsilon_r \right)
\end{align*}
que, teniendo en cuenta la expresión de $n_r$, puede escribirse
\begin{equation}\label{eq:p_cuant}
	\left\langle p \right\rangle = \sum_r \left( \frac{\partial \varepsilon_r}{\partial V} \right) \expval{n_r} 
\end{equation}

Esta relación toma una forma especialmente útil en el caso particular de un sistema de partículas ideales encerradas en un volumen $V = L_xL_yL_z$, de forma que la energía de las partículas sea únicamente de traslación, y cuyos niveles energéticos vienen dados por
\begin{equation}\label{eq:e_r_gen}
	\varepsilon_r \equiv \frac{\hbar^2 \pi^2}{2m} \left[\left( \frac{n_x}{L_x} \right)^2 + \left( \frac{n_y}{L_y} \right)^2 + \left( \frac{n_z}{L_z} \right)^2 \right] 
\end{equation}
donde $n_x$, $n_y$ y $n_z$ son números cuánticos que pueden tomar valores enteros positivos |y que evidentemente no tienen nada que ver con los números de ocupación|.
Como las propiedades termodinámicas de un sistema deben ser independientes de la forma de éste) vamos a particularizar \eqref{eq:e_r_gen} al caso concreto de un cubo $(L_x = L_y = L_z = L)$ que es la forma más sencilla, obteniendo,
\begin{equation}
\varepsilon_r \equiv \frac{\hbar^2 \pi^2}{2mV^{2/3}} \left[ n_x^2 + n_y^2 + n_z^2 \right] 
\end{equation}
donde $V = L^3$.
En esta expresión aparece explícita toda la dependencia respecto del volumen, por lo que podemos calcular
$$\frac{\partial \varepsilon_r}{\partial V} = - \frac{2}{3 V} \varepsilon_r$$
que sustituida en \eqref{eq:p_cuant} da
$$\left\langle p \right\rangle = \sum_r \frac{2}{3 V} \varepsilon_r \expval{n_r} \frac{2}{3} \frac{\left\langle E \right\rangle}{V}$$

Esta relación es válida independientemente de que las partículas que se consideren sean fermiones o bosones, con la condición de que los niveles energéticos de una partícula vengan dados por una expresión del tipo \eqref{eq:e_r_gen}.
En este sentido no será aplicable a gases poliatómicos ni a los fotones y fonones.
 
\section{Estadísticas de Fermi-Dirac y Bose-Einstein}
\subsection*{Estadística de Fermi-Dirac}

Vamos ahora a particularizar los resultados obtenidos en el apartado anterior para el caso de que las partículas que componen el sistema sean fermiones.
Sabemos que $n_r$ solo puede tomar entonces los valores 0 ó 1, de modo que $n_{max} = 1$ y \eqref{eq:Q_cuant2} toma la forma
\begin{equation}
	Q_{FD} = \prod_r \sum_{n=0}^{n=1} e^{-(\beta\varepsilon_r + \alpha) n} = \prod_r \left[ 1 + e^{ -( \beta\varepsilon_r + \alpha)} \right]
\end{equation}
de donde
\begin{equation}
	\ln Q_{FD} = \sum_r \ln \left[ 1 + e^{ -( \beta\varepsilon_r + \alpha)} \right]
\end{equation}

A partir de esta expresión obtenemos para el número medio de partículas en el sistema
\begin{equation}
	\expval{N} = - \frac{\partial \ln Q_{FD}}{\partial \alpha} = - \sum_r \frac{-e^{-\beta\varepsilon_r - \alpha}}{1 + e^{-\beta\varepsilon_r - \alpha}} = - \sum_r \frac{1}{1 + e^{\beta\varepsilon_r + \alpha}}
\end{equation}
y para el número medio de partículas en el estado r
\begin{equation}\label{eq:n_r_1}
	\expval{n_r} = - \frac{1}{\beta} \frac{\partial \ln Q_{FD}}{\partial \varepsilon_r} = \frac{1}{e^{\beta\varepsilon_r + \alpha} + 1}
\end{equation}
cumpliéndose, como es lógico, que $\expval{N} = \sum n_r$.
Además, como evidentemente se tiene
$$0 \leq e^{\beta\varepsilon_r + \alpha} \leq \infty$$
resulta que
$$0 \leq \expval{n_r} \leq 1$$
de acuerdo con el principio de exclusión de Pauli.

Esta expresión puede escribirse equivalentemente en función del potencial químico $\mu (T)$ ---que sabemos que está relacionado con $\alpha$ por $\alpha = -\beta\mu$--- en la forma
\begin{equation}\label{eq:n_r_2}
	\expval{n_r} = \sum_r \frac{1}{e^{\beta(\varepsilon_r - \mu)} + 1}
\end{equation}

Es usual en la bibliografía referirse a \eqref{eq:n_r_1} y \eqref{eq:n_r_2} como la \emph{distribución de Fermi}, reflejando el hecho de que indica el modo de distribuirse en valor medio las partículas entre los distintos estados.
Además, $\mu (T)$ recibe el nombre de \emph{nivel de Fermi} y su valor para $T = 0$, que representaremos por $\mu_0$, el de \emph{energía de Fermi} ---recordemos que el potencial químico tiene dimensiones de energía, como es fácil de comprobar---.
Así pues, escribiremos a partir de ahora
$$\mu_0 \equiv \mu(T=0)$$

\begin{wrapfigure}{r}{0.4\textwidth}
	\centering
	\hspace{3.5cm}
	% This file was created by matplotlib2tikz v0.6.15.

\begin{tikzpicture}

\definecolor{color0}{rgb}{0.12156862745098,0.466666666666667,0.705882352941177}
\definecolor{color1}{rgb}{1,0.498039215686275,0.0549019607843137}
\definecolor{color2}{rgb}{0.172549019607843,0.627450980392157,0.172549019607843}

\begin{axis}[
scale = 0.8,
xmin=-1.5, xmax=3.5,
ymin=0, ymax=1,
tick align=outside,
ytick pos=right,
yticklabel pos=right,
xtick = {1},
ytick = {0.5,1},
ylabel = {$\expval{n}$},
xlabel = {$\varepsilon_r/\mu$}]

\node at (2,0.9) {$\mathbf{T_0 = 0K}$};
\node at (2.2,0.6) {$\color{YellowOrange}\mathbf{T_1 > T_0}$};
\node at (2.2,0.4) {$\color{Sepia!50}\mathbf{T_2 > T_1}$};


\addplot [thick, black, dashed, forget plot]
table {%
	3.5 0.5
	-1.5 0.5
};

\addplot [ultra thick, black, forget plot]
table {%
-2 1
-1.93939393939394 1
-1.87878787878788 1
-1.81818181818182 1
-1.75757575757576 1
-1.6969696969697 1
-1.63636363636364 1
-1.57575757575758 1
-1.51515151515152 1
-1.45454545454545 1
-1.39393939393939 1
-1.33333333333333 1
-1.27272727272727 1
-1.21212121212121 1
-1.15151515151515 1
-1.09090909090909 1
-1.03030303030303 1
-0.96969696969697 1
-0.909090909090909 1
-0.848484848484848 1
-0.787878787878788 1
-0.727272727272727 1
-0.666666666666667 1
-0.606060606060606 1
-0.545454545454545 1
-0.484848484848485 1
-0.424242424242424 1
-0.363636363636364 1
-0.303030303030303 1
-0.242424242424242 1
-0.181818181818182 1
-0.121212121212121 1
-0.0606060606060606 1
0 1
0.0606060606060606 1
0.121212121212121 1
0.181818181818182 1
0.242424242424243 1
0.303030303030303 1
0.363636363636364 1
0.424242424242424 1
0.484848484848485 1
0.545454545454545 1
0.606060606060606 1
0.666666666666667 1
0.727272727272728 1
0.787878787878788 1
0.848484848484849 1
0.909090909090909 1
0.96969696969697 1
1.03030303030303 0
1.09090909090909 0
1.15151515151515 0
1.21212121212121 0
1.27272727272727 0
1.33333333333333 0
1.39393939393939 0
1.45454545454545 0
1.51515151515152 0
1.57575757575758 0
1.63636363636364 0
1.6969696969697 0
1.75757575757576 0
1.81818181818182 0
1.87878787878788 0
1.93939393939394 0
2 0
2.06060606060606 0
2.12121212121212 0
2.18181818181818 0
2.24242424242424 0
2.3030303030303 0
2.36363636363636 0
2.42424242424242 0
2.48484848484849 0
2.54545454545455 0
2.60606060606061 0
2.66666666666667 0
2.72727272727273 0
2.78787878787879 0
2.84848484848485 0
2.90909090909091 0
2.96969696969697 0
3.03030303030303 0
3.09090909090909 0
3.15151515151515 0
3.21212121212121 0
3.27272727272727 0
3.33333333333333 0
3.39393939393939 0
3.45454545454546 0
3.51515151515152 0
3.57575757575758 0
3.63636363636364 0
3.6969696969697 0
3.75757575757576 0
3.81818181818182 0
3.87878787878788 0
3.93939393939394 0
4 0
};
\addplot [ultra thick, Sepia!50, forget plot]
table {%
-2 0.998830489734944
-1.93939393939394 0.998659854927081
-1.87878787878788 0.998464362301224
-1.81818181818182 0.998240402586158
-1.75757575757576 0.997983846203057
-1.6969696969697 0.997689969430504
-1.63636363636364 0.997353370459146
-1.57575757575758 0.996967874067486
-1.51515151515152 0.996526423528682
-1.45454545454545 0.996020958237794
-1.39393939393939 0.995442275435552
-1.33333333333333 0.994779874306442
-1.27272727272727 0.994021780656891
-1.21212121212121 0.993154350348758
-1.15151515151515 0.9921620496943
-1.09090909090909 0.991027211138002
-1.03030303030303 0.989729762792429
-0.96969696969697 0.988246930803832
-0.909090909090909 0.986552914154096
-0.848484848484848 0.98461853242776
-0.787878787878788 0.982410848370265
-0.727272727272727 0.979892768836217
-0.666666666666667 0.977022630089974
-0.606060606060606 0.973753776504461
-0.545454545454545 0.970034145644023
-0.484848484848485 0.965805877646433
-0.424242424242424 0.961004972848204
-0.363636363636364 0.955561028785873
-0.303030303030303 0.949397096020685
-0.242424242424242 0.942429701491781
-0.181818181818182 0.934569097895778
-0.121212121212121 0.925719807196669
-0.0606060606060606 0.915781534655735
0 0.904650535100891
0.0606060606060606 0.892221513334469
0.121212121212121 0.878390132875692
0.181818181818182 0.863056188531365
0.242424242424243 0.846127465432218
0.303030303030303 0.827524257579285
0.363636363636364 0.807184451484523
0.424242424242424 0.785068996596897
0.484848484848485 0.761167489044236
0.545454545454545 0.73550349858496
0.606060606060606 0.708139185140206
0.666666666666667 0.679178699175393
0.727272727272728 0.648769858868011
0.787878787878788 0.617103662552354
0.848484848484849 0.584411335070708
0.909090909090909 0.550958815914465
0.96969696969697 0.517038854261009
1.03030303030303 0.482961145738991
1.09090909090909 0.449041184085535
1.15151515151515 0.415588664929292
1.21212121212121 0.382896337447646
1.27272727272727 0.351230141131988
1.33333333333333 0.320821300824607
1.39393939393939 0.291860814859794
1.45454545454545 0.26449650141504
1.51515151515152 0.238832510955764
1.57575757575758 0.214931003403103
1.63636363636364 0.192815548515476
1.6969696969697 0.172475742420715
1.75757575757576 0.153872534567782
1.81818181818182 0.136943811468635
1.87878787878788 0.121609867124308
1.93939393939394 0.107778486665531
2 0.0953494648991095
2.06060606060606 0.0842184653442654
2.12121212121212 0.0742801928033306
2.18181818181818 0.0654309021042215
2.24242424242424 0.0575702985082191
2.3030303030303 0.0506029039793152
2.36363636363636 0.0444389712141273
2.42424242424242 0.0389950271517962
2.48484848484849 0.0341941223535668
2.54545454545455 0.0299658543559764
2.60606060606061 0.0262462234955384
2.66666666666667 0.0229773699100256
2.72727272727273 0.0201072311637834
2.78787878787879 0.0175891516297347
2.84848484848485 0.0153814675722404
2.90909090909091 0.013447085845904
2.96969696969697 0.0117530691961683
3.03030303030303 0.0102702372075714
3.09090909090909 0.00897278886199761
3.15151515151515 0.00783795030569963
3.21212121212121 0.00684564965124238
3.27272727272727 0.00597821934310928
3.33333333333333 0.00522012569355839
3.39393939393939 0.00455772456444758
3.45454545454546 0.00397904176220586
3.51515151515152 0.00347357647131785
3.57575757575758 0.00303212593251372
3.63636363636364 0.00264662954085364
3.6969696969697 0.00231003056949582
3.75757575757576 0.00201615379694347
3.81818181818182 0.00175959741384183
3.87878787878788 0.00153563769877638
3.93939393939394 0.00134014507291926
4 0.00116951026505551
};
\addplot [ultra thick, YellowOrange, forget plot]
table {%
-2 0.99999999983081
-1.93939393939394 0.999999999733449
-1.87878787878788 0.99999999958006
-1.81818181818182 0.999999999338402
-1.75757575757576 0.999999998957681
-1.6969696969697 0.999999998357872
-1.63636363636364 0.999999997412897
-1.57575757575758 0.999999995924131
-1.51515151515152 0.999999993578643
-1.45454545454545 0.999999989883428
-1.39393939393939 0.999999984061774
-1.33333333333333 0.999999974890009
-1.27272727272727 0.999999960440287
-1.21212121212121 0.999999937675371
-1.15151515151515 0.999999901810223
-1.09090909090909 0.999999845306228
-1.03030303030303 0.999999756286619
-0.96969696969697 0.999999616040077
-0.909090909090909 0.999999395087745
-0.848484848484848 0.999999046987023
-0.787878787878788 0.99999849856976
-0.727272727272727 0.999997634563106
-0.666666666666667 0.999996273360716
-0.606060606060606 0.999994128852259
-0.545454545454545 0.999990750289837
-0.484848484848485 0.999985427555986
-0.424242424242424 0.999977041932086
-0.363636363636364 0.999963831026667
-0.303030303030303 0.999943018520036
-0.242424242424242 0.999910231066187
-0.181818181818182 0.999858580201049
-0.121212121212121 0.999777217303468
-0.0606060606060606 0.999649060409353
0 0.999447221363076
0.0606060606060606 0.999129397908724
0.121212121212121 0.998629090568688
0.181818181818182 0.997841893527604
0.242424242424243 0.996604213048194
0.303030303030303 0.994660517366896
0.363636363636364 0.991613642441852
0.424242424242424 0.986851109813223
0.484848484848485 0.979440056876836
0.545454545454545 0.967987443808298
0.606060606060606 0.95047788015511
0.666666666666667 0.924141819978756
0.727272727272728 0.885487519467777
0.787878787878788 0.830743967346632
0.848484848484849 0.757011372822882
0.909090909090909 0.664144375862977
0.96969696969697 0.556574869897311
1.03030303030303 0.443425130102689
1.09090909090909 0.335855624137023
1.15151515151515 0.242988627177118
1.21212121212121 0.169256032653367
1.27272727272727 0.114512480532223
1.33333333333333 0.0758581800212435
1.39393939393939 0.0495221198448899
1.45454545454545 0.0320125561917019
1.51515151515152 0.020559943123164
1.57575757575758 0.0131488901867773
1.63636363636364 0.00838635755814738
1.6969696969697 0.00533948263310347
1.75757575757576 0.00339578695180564
1.81818181818182 0.00215810647239593
1.87878787878788 0.00137090943131169
1.93939393939394 0.000870602091276129
2 0.0005527786369236
2.06060606060606 0.000350939590646793
2.12121212121212 0.000222782696532081
2.18181818181818 0.000141419798950703
2.24242424242424 8.97689338125817e-05
2.3030303030303 5.69814799643932e-05
2.36363636363636 3.61689733327415e-05
2.42424242424242 2.29580679137978e-05
2.48484848484849 1.45724440142074e-05
2.54545454545455 9.24971016298655e-06
2.60606060606061 5.8711477412452e-06
2.66666666666667 3.72663928418655e-06
2.72727272727273 2.36543689377272e-06
2.78787878787879 1.50143024043019e-06
2.84848484848485 9.53012976748569e-07
2.90909090909091 6.04912254635526e-07
2.96969696969697 3.83959923117318e-07
3.03030303030303 2.4371338079604e-07
3.09090909090909 1.54693772345089e-07
3.15151515151515 9.81897766388001e-08
3.21212121212121 6.23246294355404e-08
3.27272727272727 3.95597130002686e-08
3.33333333333333 2.51099909269281e-08
3.39393939393939 1.59382258667629e-08
3.45454545454546 1.01165724698257e-08
3.51515151515152 6.42135700517529e-09
3.57575757575758 4.07586915976196e-09
3.63636363636364 2.58710259838804e-09
3.6969696969697 1.64212824028369e-09
3.75757575757576 1.04231859930357e-09
3.81818181818182 6.61597575368682e-10
3.87878787878788 4.19940075870571e-10
3.93939393939394 2.66551259968483e-10
4 1.69189792232888e-10
};
\end{axis}

\end{tikzpicture}
	\vspace{-0.5cm}
\end{wrapfigure}
En la figura adyacente se representa la forma de $\expval{n_r}$ para distintos valores de la temperatura.
En el cero absoluto $(T = 0K)$, $\expval{n_r}$ es una función escalón y vale 1 para $\varepsilon_r \leq \mu_0$ y 0 para $\varepsilon_r > \mu_0$ de forma que todos los estados que correspondan a una energía menor o igual que la energía de Fermi están ocupados, mientras que los estados de energía superior están vacíos.
A temperaturas superiores, los niveles con energía inferior a $\mu$, pero próxima a él, comienzan a despoblarse en beneficio de los que poseen una energía superior.
Desde luego, en rigor, las curvas de la figura solo poseen significado para ciertos valores de $\varepsilon_r$ pues, como sabemos, los posibles valores de la energía de una partícula localizada en una cierta región del espacio tienen espectro discreto.

Obsérvese que para cualquier temperatura $T > 0 K$,
\begin{equation}
	\expval{n_r} (\varepsilon_r = \mu )= \frac{1}{2}
\end{equation}
y, por lo tanto, los estados con energía menor que la del nivel de Fermi, siempre tienen $\expval{n_r} > \sfrac{1}{2}$ y los que poseen una energía superior, $\expval{n_r} < \sfrac{1}{2}$.

\subsection*{Estadística de Bose-Einstein}

En el caso de bosones no existe limite para el número de partículas idénticas que pueden encontrarse en un mismo estado, de manera que $n_{max} = \infty$ y
\begin{equation}
	Q_{BE} = \prod_r \sum_{n=0}^{\infty} e^{-(\beta\varepsilon_r + \alpha) n}
\end{equation}

En esta expresión se observa que la suma se extiende ---para un $\varepsilon_r$ dado--- a los términos de una progresión geométrica indefinida de razón
$$e^{-(\beta\varepsilon_r + \alpha)} = e^{-\beta(\varepsilon_r - \mu)}$$
	
Esta serie es decreciente y, por lo tanto, su suma converge si la tazón es menor que la unidad o sea si
$$e^{-\beta(\varepsilon_r - \mu)} < 1$$
lo que equivale a
\begin{equation}\label{eq:cond_FD}
	\varepsilon_r - \mu > 0
\end{equation}

Esta condición se satisface siempre en las aplicaciones prácticas, para todos los estados $r$, ya que en otro caso el número de bosones en el sistema no estaría acotado.

Sumando la serie,\footnote{Recordemos que si $r < 1$ se tiene $\sum_{n=0}^{\infty} ar^n = \frac{a}{1-r}$} obtenemos
\begin{equation}
	Q_{BE} = \prod_r \frac{1}{1 - e^{-\beta\varepsilon_r + \alpha}}
\end{equation}
y, por lo tanto,
\begin{equation}\label{eq:Q_BE}
	\ln Q_{BE} = - \sum_r \ln (1 - e^{-\beta\varepsilon_r + \alpha})
\end{equation}

A partir de esta expresión deducimos
\begin{equation}
	\expval{N} = - \frac{\partial \ln Q_{BE}}{\partial \alpha} = \sum_r \frac{1}{e^{-\beta\varepsilon_r + \alpha} - 1}
\end{equation}
\begin{equation}\label{eq:n_r_BE}
	\expval{n_r} = \frac{1}{e^{-\beta\varepsilon_r + \alpha} - 1} = \frac{1}{e^{-\beta(\varepsilon_r - \mu)} - 1}
\end{equation}

Como es lógico, $\expval{n_r}$ puede ahora ser mayor que 1, y en particular tiende a infinito cuando $\varepsilon_r \rightarrow \mu$.
En la figura se representa el comportamiento de la función de distribución de Bose para distintos valores de la temperatura.
En el cero absoluto todos los bosones se agrupan en el estado de mínima energía, tendiendo a pasar a ocupar estados de mayor energía cuando se aumenta la temperatura.

Las expresiones correspondientes a ambas estadísticas pueden agruparse escribiendo
\begin{equation}
	\ln Q = \pm \sum_r 	\ln (1 \pm e^{-\beta\varepsilon_r + \alpha})
\end{equation}
\begin{equation}\label{eq:n_e_BE}
	\expval{n_r} = \frac{1}{e^{-\beta\varepsilon_r + \alpha} \pm 1}
\end{equation}
siendo válidos los signos superiores (+) en el caso de la estadística de Fermi-Dirac y
los signos inferiores (-) en el caso de la de Bose-Einstein.
Con este mismo convenio podemos escribir la ecuación de estado como
\begin{equation}
	\left\langle p \right\rangle V = \pm k_B T \sum_r \ln (1 \pm e^{-\beta\varepsilon_r + \alpha})
\end{equation}

Con vistas a la utilización práctica de todas estas expresiones es importante recordar que si se aplican a un sistema abierto, el potencial químico viene dado por el del foco con el que esté en contacto.
Pero como ya hemos dicho reiteradamente, estas expresiones son también aplicables a un sistema cerrado ($N$ constante), en cuyo caso el potencial químico $\mu$, se determinará por la condición
\begin{equation}
	N = \sum_r \expval{n_r} = \sum_r \frac{1}{e^{-\beta(\varepsilon_r - \mu)} \pm 1}
\end{equation}

\section{El límite clásico: la estadística de Maxwell-Boltzmann}

El número de estados cuánticos $r$ de una partícula depende del volumen del sistema.
Más concretamente, se demuestra que el número de estados de traslación es proporcional al mismo.
Utilizando este hecho vamos a analizar el comportamiento de las funciones de distribución \eqref{eq:n_e_BE} en dos casos límites:

\textbf{1)} Imaginemos que, manteniendo la temperatura del gas constante, tomamos el límite de bajas densidades, esto es, hacemos suficientemente pequeño el número de partículas por unidad de volumen,\footnote{Desde luego en un sistema abierto nos referimos a $\sfrac{\expval{N}}{V}$ y en uno cerrado a $\sfrac{N}{V}$.} como para que el número de estados de partícula por unidad de volumen sea mucho mayor que él. Consideremos la expresión
\begin{equation}\label{eq:sum_r_cas1}
	\sum_r \expval{n_r} = \sum_r \frac{1}{e^{\beta\varepsilon_r + \alpha} \pm 1} = \expval{N}
\end{equation}

De acuerdo con lo que acabamos de admitir, resulta que el número de sumandos que contienen los sumatorios es mucho mayor que $\expval{N}$. La consecuencia es que cada uno de ellos ha de ser mucho menor que la unidad, o sea
\begin{equation}\label{eq:sum_r_cas1_2}
	\expval{n_r} = \frac{1}{e^{\beta\varepsilon_r + \alpha} \pm 1} \ll 1
\end{equation}
y, por lo tanto,
\begin{equation}\label{eq:lim_exp}
	\beta\varepsilon_r + \alpha \gg 1
\end{equation}
para todos los estados de partícula $r$.
En efecto, si recordamos que los niveles energéticos de traslación para un sistema macroscópico son muy degenerados o sea que a un valor de la energía $\varepsilon_r$ corresponde un gran número de estados $r$, y que además los niveles están muy próximos, resulta que si en \eqref{eq:sum_r_cas1} hubiese un sumando del orden de la unidad, automáticamente habría un número prácticamente infinito de ellos que serían de ese mismo orden.

\vspace{0.2cm}
\textbf{2)} Supongamos ahora que mantenemos fija la densidad del gas y aumentamos su temperatura, o sea disminuimos $\beta$.
Es evidente que el número de sumandos que constituyen efectivamente a \eqref{eq:sum_r_cas1} aumenta, pues cuanto menor sea $\beta$ mayor puede ser $\varepsilon_r$ sin que sea $\beta\varepsilon_r \gg 1$.
Aplicando el mismo razonamiento que en el caso anterior, llegamos a la conclusión de que para temperaturas suficientemente altas también debe cumplirse \eqref{eq:lim_exp}.

\begin{wrapfigure}{l}{0.35\textwidth}
	\centering
	\hspace{-1.8cm}
	% This file was created by matplotlib2tikz v0.6.16.
\begin{tikzpicture}

\definecolor{color0}{rgb}{0.12156862745098,0.466666666666667,0.705882352941177}
\definecolor{color1}{rgb}{1,0.498039215686275,0.0549019607843137}
\definecolor{color2}{rgb}{0.172549019607843,0.627450980392157,0.172549019607843}

\begin{axis}[
scale=0.75,
y = 2cm,
xmin=-3.1, xmax=3.1,
ymin=0, ymax=2.6,
tick align=outside,
tick pos=left,
ylabel = {$\expval{n}$},
xtick = {0},
ytick = {1},
xlabel = {$\beta(\varepsilon - \mu)$},
x grid style={white!69.01960784313725!black},
y grid style={white!69.01960784313725!black}
]

\node at (-2.1,1.2) {\color{color0}\textbf{FD}};
\node at (-1.3,1.8) {\color{Sepia!80}\textbf{MB}};
\node at (1.4,1.5) {\color{color1}\textbf{BE}};



\addplot [thick, black, dashed, forget plot]
table {%
0 0
0 2.6
};
\addplot [thick, black, dashed, forget plot]
table {%
-3.1 1
3.1  1
};
\addplot [ultra thick, color0, forget plot]
table {%
-3 0.977022630089974
-2.93939393939394 0.975259027362491
-2.87878787878788 0.973363751617231
-2.81818181818182 0.971327557243786
-2.75757575757576 0.969140641919771
-2.6969696969697 0.966792627904751
-2.63636363636364 0.964272545296714
-2.57575757575758 0.961568817814162
-2.51515151515152 0.958669251744938
-2.45454545454545 0.955561028785873
-2.39393939393939 0.952230703584588
-2.33333333333333 0.948664206884684
-2.27272727272727 0.944846855266537
-2.21212121212121 0.940763368565415
-2.15151515151515 0.936397896133564
-2.09090909090909 0.931734053189477
-2.03030303030303 0.926754968561019
-1.96969696969697 0.9214433451738
-1.90909090909091 0.915781534655735
-1.84848484848485 0.909751627415579
-1.78787878787879 0.903335559499269
-1.72727272727273 0.896515237424059
-1.66666666666667 0.889272682027632
-1.60606060606061 0.881590192138079
-1.54545454545455 0.873450528562205
-1.48484848484848 0.864837118496269
-1.42424242424242 0.855734279979351
-1.36363636363636 0.846127465432218
-1.3030303030303 0.836003522654966
-1.24242424242424 0.825350970901034
-1.18181818181818 0.814160288815684
-1.12121212121212 0.802424210143189
-1.06060606060606 0.790138022195814
-1 0.777299861174691
-0.939393939393939 0.763910997580872
-0.878787878787879 0.74997610420435
-0.818181818181818 0.73550349858496
-0.757575757575757 0.720505351459346
-0.696969696969697 0.704997852599438
-0.636363636363636 0.689001325661223
-0.575757575757576 0.672540284239115
-0.515151515151515 0.655643422286646
-0.454545454545455 0.638343533424372
-0.393939393939394 0.620677355392887
-0.333333333333333 0.602685337978492
-0.272727272727272 0.584411335070708
-0.212121212121212 0.56590222400902
-0.151515151515151 0.547207457925119
-0.0909090909090908 0.528378559257178
-0.0303030303030303 0.509468564870001
0.0303030303030303 0.490531435129999
0.0909090909090908 0.471621440742822
0.151515151515151 0.452792542074881
0.212121212121212 0.43409777599098
0.272727272727273 0.415588664929292
0.333333333333333 0.397314662021508
0.393939393939394 0.379322644607113
0.454545454545455 0.361656466575628
0.515151515151515 0.344356577713354
0.575757575757576 0.327459715760885
0.636363636363637 0.310998674338777
0.696969696969697 0.295002147400562
0.757575757575758 0.279494648540653
0.818181818181818 0.26449650141504
0.878787878787879 0.25002389579565
0.939393939393939 0.236089002419128
1 0.222700138825309
1.06060606060606 0.209861977804186
1.12121212121212 0.197575789856811
1.18181818181818 0.185839711184316
1.24242424242424 0.174649029098967
1.3030303030303 0.163996477345034
1.36363636363636 0.153872534567782
1.42424242424242 0.144265720020649
1.48484848484849 0.135162881503731
1.54545454545455 0.126549471437795
1.60606060606061 0.118409807861921
1.66666666666667 0.110727317972368
1.72727272727273 0.103484762575941
1.78787878787879 0.0966644405007311
1.84848484848485 0.090248372584421
1.90909090909091 0.0842184653442654
1.96969696969697 0.0785566548261996
2.03030303030303 0.0732450314389812
2.09090909090909 0.0682659468105229
2.15151515151515 0.0636021038664357
2.21212121212121 0.0592366314345846
2.27272727272727 0.0551531447334633
2.33333333333333 0.0513357931153162
2.39393939393939 0.0477692964154116
2.45454545454546 0.0444389712141272
2.51515151515152 0.0413307482550625
2.57575757575758 0.0384311821858377
2.63636363636364 0.0357274547032864
2.6969696969697 0.0332073720952487
2.75757575757576 0.0308593580802294
2.81818181818182 0.0286724427562142
2.87878787878788 0.0266362483827693
2.93939393939394 0.0247409726375094
3 0.0229773699100256
};
\addplot [ultra thick, Sepia!80, forget plot]
table {%
-3 42.5210820000628
-2.93939393939394 39.4187828284453
-2.87878787878788 36.5428245610929
-2.81818181818182 33.8766935730744
-2.75757575757576 31.4050810583992
-2.6969696969697 29.1137951275307
-2.63636363636364 26.9896793181859
-2.57575757575758 25.0205370515121
-2.51515151515152 23.1950615998711
-2.45454545454545 21.5027711641105
-2.39393939393939 19.9339486875376
-2.33333333333333 18.4795860610098
-2.27272727272727 17.1313323987719
-2.21212121212121 15.8814460880394
-2.15151515151515 14.7227503370014
-2.09090909090909 13.6485919659999
-2.03030303030303 12.6528032052669
-1.96969696969697 11.7296662798641
-1.90909090909091 10.8738805784721
-1.84848484848485 10.0805322175153
-1.78787878787879 9.34506582586009
-1.72727272727273 8.66325838807586
-1.66666666666667 8.03119499606725
-1.60606060606061 7.44524636984554
-1.54545454545455 6.90204801836367
-1.48484848484848 6.39848092075779
-1.42424242424242 5.93165361706764
-1.36363636363636 5.49888560560163
-1.3030303030303 5.09769195161484
-1.24242424242424 4.72576901892389
-1.18181818181818 4.38098124253003
-1.12121212121212 4.06134886629952
-1.06060606060606 3.76503657529169
-1 3.49034295746184
-0.939393939393939 3.2356907342287
-0.878787878787879 2.99961770381068
-0.818181818181818 2.78076834532806
-0.757575757575757 2.57788603546213
-0.696969696969697 2.38980583297983
-0.636363636363636 2.21544778969276
-0.575757575757576 2.05381074944258
-0.515151515151515 1.90396659950666
-0.454545454545455 1.76505494141602
-0.393939393939394 1.63627815058539
-0.333333333333333 1.51689679638821
-0.272727272727272 1.40622539637875
-0.212121212121212 1.30362848028224
-0.151515151515151 1.20851694115277
-0.0909090909090908 1.12034465274726
-0.0303030303030303 1.0386053336928
0.0303030303030303 0.96282964044144
0.0909090909090908 0.892582472320324
0.151515151515151 0.82746047320291
0.212121212121212 0.767089715455969
0.272727272727273 0.711123552863686
0.333333333333333 0.659240630200444
0.393939393939394 0.611143038023359
0.454545454545455 0.566554602089466
0.515151515151515 0.525219297575446
0.575757575757576 0.486899778994442
0.636363636363637 0.451376017368786
0.696969696969697 0.418444036833364
0.757575757575758 0.387914743415231
0.818181818181818 0.359612839264404
0.878787878787879 0.333375816101369
0.939393939393939 0.309053022101747
1 0.28650479686019
1.06060606060606 0.265601669466525
1.12121212121212 0.246223615089522
1.18181818181818 0.228259365799635
1.24242424242424 0.211605771673477
1.3030303030303 0.196167208511535
1.36363636363636 0.18185502876825
1.42424242424242 0.168587052541742
1.48484848484849 0.156287095700454
1.54545454545455 0.14488453243724
1.60606060606061 0.134313889739118
1.66666666666667 0.124514471444123
1.72727272727273 0.115430009726641
1.78787878787879 0.107008342010043
1.84848484848485 0.0992011114514831
1.90909090909091 0.0919634892790508
1.96969696969697 0.0852539173869474
2.03030303030303 0.0790338697106846
2.09090909090909 0.07326763101213
2.15151515151515 0.0679220918041915
2.21212121212121 0.0629665582376102
2.27272727272727 0.0583725758582377
2.33333333333333 0.0541137662228216
2.39393939393939 0.0501656754351527
2.45454545454546 0.0465056337328773
2.51515151515152 0.0431126253187255
2.57575757575758 0.0399671676887354
2.63636363636364 0.0370511997645778
2.6969696969697 0.0343479781876454
2.75757575757576 0.0318419811794293
2.81818181818182 0.0295188194161549
2.87878787878788 0.027365153405922
2.93939393939394 0.0253686168939336
3 0.0235177458560091
};
\addplot [ultra thick, color1, forget plot]
table {%
0 inf
0.0303030303030303 25.9031564901751
0.0606060606060606 12.7063125275225
0.0909090909090908 8.30946765952429
0.121212121212121 6.11262143428024
0.151515151515152 4.79577340081625
0.181818181818182 3.91892310939073
0.212121212121212 3.29349868322777
0.242424242424242 2.82521396167669
0.272727272727273 2.46168754813066
0.303030303030303 2.17149042937069
0.333333333333333 1.93462216633464
0.363636363636364 1.73774898964402
0.393939393939394 1.57163969732416
0.424242424242424 1.4297004536184
0.454545454545455 1.30709566838314
0.484848484848485 1.20019854680692
0.515151515151515 1.10623556284684
0.545454545454545 1.02304944160068
0.575757575757576 0.948936989425245
0.606060606060606 0.882535575087063
0.636363636363636 0.82274204493208
0.666666666666667 0.768653752156565
0.696969696969697 0.719524969798075
0.727272727272727 0.674734200525591
0.757575757575758 0.633759332122564
0.787878787878788 0.596158526358522
0.818181818181818 0.561555354812744
0.848484848484849 0.529627119909685
0.878787878787879 0.500095592319619
0.909090909090909 0.472719600903755
0.939393939393939 0.447289056885139
0.96969696969697 0.423620098506363
1 0.401551118493013
1.03030303030303 0.380939492566744
1.06060606060606 0.36165886879616
1.09090909090909 0.34359690873183
1.12121212121212 0.326653394851001
1.15151515151515 0.31073863683218
1.18181818181818 0.295772123021803
1.21212121212121 0.281681374182797
1.24242424242424 0.268400964987579
1.27272727272727 0.255871685296652
1.3030303030303 0.244039818465591
1.33333333333333 0.232856518060986
1.36363636363636 0.222277267675996
1.39393939393939 0.212261411198757
1.42424242424242 0.202771743039528
1.45454545454545 0.193774149571544
1.48484848484848 0.185237294468317
1.51515151515152 0.177132341790955
1.54545454545455 0.169432711643245
1.57575757575758 0.162113864009539
1.60606060606061 0.155153107052441
1.63636363636364 0.148529426698852
1.66666666666667 0.142223334804301
1.6969696969697 0.136216733572651
1.72727272727273 0.130492794234371
1.75757575757576 0.125035848261984
1.78787878787879 0.119831289634786
1.81818181818182 0.114865486863319
1.84848484848485 0.110125703653264
1.87878787878788 0.105600027233005
1.90909090909091 0.101277303493045
1.93939393939394 0.0971470781919676
1.96969696969697 0.0931995435754281
2 0.089425489833852
2.03030303030303 0.0858162608931738
2.06060606060606 0.0823637140924376
2.09090909090909 0.0790601833538513
2.12121212121212 0.0758984454959869
2.15151515151515 0.0728716893801999
2.18181818181818 0.0699734876148052
2.21212121212121 0.0671977705717541
2.24242424242424 0.0645388024970951
2.27272727272727 0.0619911595198503
2.3030303030303 0.0595497093845231
2.33333333333333 0.0572095927506321
2.36363636363636 0.0549662059187438
2.39393939393939 0.0528151848567227
2.42424242424242 0.0507523904125633
2.45454545454545 0.0487738946114011
2.48484848484848 0.0468759679443064
2.51515151515152 0.0450550675653817
2.54545454545455 0.0433078263216422
2.57575757575758 0.0416310425472794
2.60606060606061 0.0400216705602786
2.63636363636364 0.0384768118050715
2.66666666666667 0.0369937065900355
2.6969696969697 0.0355697263732544
2.72727272727273 0.034202366554107
2.75757575757576 0.0328892397319808
2.78787878787879 0.0316280693967816
2.81818181818182 0.0304166840189485
2.84848484848485 0.0292530115094396
2.87878787878788 0.0281350740226385
2.90909090909091 0.027060983077392
2.93939393939394 0.0260289349734317
2.96969696969697 0.0250372064822928
3 0.0240841507935291
};
\end{axis}

\end{tikzpicture}
	\vspace{0.5cm}
\end{wrapfigure}
Resulta entonces que, en el límite de densidades bajas o temperaturas altas, debe cumplirse \eqref{eq:lim_exp}, cuya forma es independiente de la estadística de que se trate bien sea la de Fermi-Dirac o la de Bose-Einstein.
Despreciando la unidad frente a $e^{\beta\varepsilon_r + \alpha}$ en \eqref{eq:sum_r_cas1_2}, en el límite señalado obtenemos para ambas estadísticas (figura adyacente).
\begin{equation}\label{eq:n_r_MB}
	\expval{n_r} = \exp[-\beta\varepsilon_r - \alpha]
\end{equation}

El parámetro $\alpha$ puede determinarse por la condición
\begin{equation}
	\expval{N} = \sum_r \expval{n_r} = \sum_r \exp[-\beta\varepsilon_r - \alpha] = e^{-\alpha} \sum_r e^{-\beta\varepsilon_r}
\end{equation}
de donde se puede despejar $e^{\alpha}$ e introducirse en \eqref{eq:n_r_MB}, resultando
\begin{equation}
	\expval{n_r} = \expval{N} \frac{e^{-\beta\varepsilon_r}}{\sum_r e^{-\beta\varepsilon_r}}
\end{equation}
siendo $\expval{N} = N$ en el caso de un sistema cerrado.

En el mismo límite y utilizando la aproximación
$$\ln(1+x) \approx x \qquad x \ll 1$$
obtenemos para la función de partición generalizada
\begin{equation}
	\ln Q = \pm \sum_r \ln(1 \pm e^{-\beta\varepsilon_r}) = \sum_r e^{\beta\varepsilon_r}
\end{equation}

Sustituyendo en esta expresión el valor de $e^{-\alpha}$ que obtuvimos antes resulta
\begin{equation}
	\ln Q = \expval{N}
\end{equation}
que es el mismo resultado que se obtiene mediante la Mecánica Estadística Clásica.

\section{Gas ideal monoatómico}
En general, definimos la función de partición por partículas de un gas ideal en el límite clásico como
\begin{equation}\label{eq:zeta_lim}
	\zeta = e^{-\beta\varepsilon_r}
\end{equation}

En el caso de un gas ideal monoatómico (sin grados internos de libertad) el estudio de los estados cuánticos accesibles a un átomo es equivalente al estudio de los estados estacionarios de una partícula puntual encerrada en un volumen $V$ y que, como ya sabemos, vienen caracterizados por tres números cuánticos $n_x$, $n_y$ y $n_z$ y cuya energía asociada a un estado es
\begin{equation}
\varepsilon_r \equiv \frac{\hbar^2 \pi^2}{2m} \left[\left( \frac{n_x}{L_x} \right)^2 + \left( \frac{n_y}{L_y} \right)^2 + \left( \frac{n_z}{L_z} \right)^2 \right] 
\end{equation}

Teniendo en cuenta que sumar para todos los estados de una partícula es equivalente a sumar sobre todos los valores posibles de $n_x$, $n_y$ y $n_z$ que son enteros positivos, resulta
\begin{align}\label{eq:z_gen}
	\zeta &= \sum_{n_x=1}^{\infty} \sum_{n_y=1}^{\infty} \sum_{n_z=1}^{\infty} \exp \left\lbrace - \beta \frac{\hbar^2 \pi^2}{2m} \left[\left( \frac{n_x}{L_x} \right)^2 + \left( \frac{n_y}{L_y} \right)^2 + \left( \frac{n_z}{L_z} \right)^2 \right]  \right\rbrace \nonumber \\
		&= \left\lbrace \sum_{n_x=1}^{\infty} \exp \left[ -\beta \frac{\hbar^2 \pi^2}{2m} \left( \frac{n_x}{L_x} \right)^2 \right] \right\rbrace \left\lbrace \sum_{n_y=1}^{\infty} \exp \left[ -\beta \frac{\hbar^2 \pi^2}{2m} \left( \frac{n_y}{L_y} \right)^2 \right] \right\rbrace \times \\
		&\times \left\lbrace \sum_{n_z=1}^{\infty} \exp \left[ -\beta \frac{\hbar^2 \pi^2}{2m} \left( \frac{n_z}{L_z} \right)^2 \right] \right\rbrace \nonumber
\end{align}
y vamos a ver cómo puede evaluarse cada una de las sumas indicadas.
Consideremos, por ejemplo, la primera de ellas y calculemos la variación del exponente al pasar de un sumando al siguiente.

Resulta entonces, que la diferencia entre dos sumandos consecutivos de cada una de las sumas \eqref{eq:z_gen} es muy pequeña frente al valor de cada uno de los sumandos, pues
\begin{equation}
	\abs{\frac{\exp \left[ -\beta \frac{\hbar^2 \pi^2}{2m} \left( \frac{n_x + 1}{L_x} \right)^2 \right] - \exp \left[ -\beta \frac{\hbar^2 \pi^2}{2m} \left( \frac{n_x}{L_x} \right)^2 \right]}{\exp \left[ -\beta \frac{\hbar^2 \pi^2}{2m} \left( \frac{n_x + 1}{L_x} \right)^2 \right]}} = \abs{1 - \exp \left[ -\beta \frac{\hbar^2 \pi^2}{2m} \left( \frac{2n_x + 1}{L_x} \right)^2 \right]} \ll 1
\end{equation}

La consecuencia es que las sumas pueden sustituirse por integrales, como ya hemos hecho repetidamente, o sea, que podemos escribir
\begin{equation}
	\sum_{n_x=1}^{\infty} \exp \left[ -\beta \frac{\hbar^2 \pi^2}{2m} \left( \frac{n_x}{L_x} \right)^2 \right] = \int_{0}^{\infty} \exp \left[ -\beta \frac{\hbar^2 \pi^2}{2m} \left( \frac{n_x}{L_x} \right)^2 \right] = \frac{1}{2}\sqrt{\frac{2\pi m L_x^2}{\beta \hbar^2 \pi^2}}
\end{equation}
y análogamente para las sumas respecto de $n_y$ y $n_z$.
En resumen, sustituyendo estas expresiones en \eqref{eq:z_gen} obtenemos la función de partición de una partícula en el límite clásico
\begin{equation}
	\zeta = \frac{V}{(2\pi\hbar)^3}(2\pi m k_B T)^{3/2}= \frac{V}{h^3}(2\pi m k_B T)^{3/2}
\end{equation}
donde hemos introducido el volumen $V = L_x L_y L_z$ y hemos escrito $h = \hbar 2 \pi$. Con este resultado llegamos también a Q
\begin{equation}
	Q = \frac{\zeta^N}{N!} = \frac{V^N}{h^{3N}N!}(2\pi m k_B T)^{3N/2} = \frac{V^N}{N!} \left[ \frac{2\pi m k_B T}{h^2}\right] ^{\frac{3N}{2}}
\end{equation}

\section{Número de estados estacionarios}

El problema que vamos a resolver ahora es el siguiente: dado un valor de la energía
$\varepsilon$, ¿cuántos estados estacionarios corresponden a ese valor? 
El problema consiste en encontrar todos los valores posibles de $n_x$, $n_y$ y $n_z$ que satisfacen la relación dada.
Así planteado el problemas es de difícil resolución general; sin embargo, nosotros vamos a estar siempre interesados en volúmenes macroscópicos, de forma que $L_x$, $L_y$ y $L_z$ van a ser muy grandes y en consecuencia los valores de la energía van a estar muy próximos entre sí, permitiéndonos incluso hablar de una distribución continua de energía desde un punto de vista matemático.
Por ello, vamos a cambiar nuestra pregunta formulándola del modo siguiente: ¿cuántos estados estacionarios pueden encontrarse con una energía comprendida entre $\varepsilon$ y $\varepsilon + d\varepsilon$?

Consideremos los estados que existen en una componente $p_x$ de la cantidad de
movimiento cuyo valor absoluto esté comprendido entre $\abs{p_x}$ y $\abs{p_x} + d\abs{p_x}$, para valores
dados de $n_y$ y $n_z$.
Evidentemente dicho número vendrá dado por
\begin{equation}
	D(\abs{p_x}) d\abs{p_x} = \Delta n_x
\end{equation}
siendo $\Delta n_x$ el número de valores enteros de $n_x$ para los que $\abs{p_x}$ está dentro del intervalo citado.
Ahora bien, podemos calcular la expresión de $p_x$ y así
\begin{equation}
	p_x = \pm\hbar\frac{n_x\pi}{L_x} \Rightarrow \Delta n_x = \frac{L_x}{n_x\pi}d\abs{p_x}
\end{equation}
luego
\begin{equation}
	D(\abs{p_x}) d\abs{p_x} = \frac{L_x}{\hbar\pi}d\abs{p_x}
\end{equation}

Es evidente entonces que el número de estados correspondientes a cantidades de movimiento con componentes cuyo valor absoluto está comprendido entre $\abs{p_x}$ y $\abs{p_x} + d\abs{p_x}$, $\abs{p_y}$ y $\abs{p_y} + d\abs{p_y}$ y $\abs{p_z}$ y $\abs{p_z} + d\abs{p_z}$ será
\begin{equation}
	D(\abs{p_x}, \abs{p_y}, \abs{p_z}) \, d\abs{p_x} d\abs{p_y} d\abs{p_z}= \frac{L_xL_yL_z}{(\hbar\pi)^3} d\abs{p_x} d\abs{p_y} d\abs{p_z}
\end{equation}
es decir
\begin{equation}
	D(\abs{p_x}, \abs{p_y}, \abs{p_z}) \, d\abs{p_x} d\abs{p_y} d\abs{p_z}= \frac{V}{(\hbar\pi)^3} d\abs{p_x} d\abs{p_y} d\abs{p_z}
\end{equation}

Pasemos ahora a calcular el número de estados con cantidad de movimiento \textbf{p} con un módulo comprendido entre $p$ y $p+dp$.
De la figura se deduce que\footnote{El volumen de la corteza esférica de anchura $dp$ es $\sfrac{4}{3}\pi[(p + dp)^3 - p^3]=4\pi p^2 dp + \order{(dp)^2}$. El factor 8 proviene de que solo hay que elegir un octante.}
\begin{equation}
	D(p)dp = D(\abs{p_x}, \abs{p_y}, \abs{p_z}) \frac{4\pi p^2 dp}{8}
\end{equation}
o sea
\begin{equation}
	D(p)dp = \frac{V}{(2\hbar\pi)^3}4\pi p^2 dp = \frac{4\pi V}{h^3} p^2 dp
\end{equation}

De esta distribución de densidades para $p$ ya es fácil pasar a la distribución para la energía.
Del hecho de que $E = p^2/2m$ tenemos
\begin{equation}
	D(\varepsilon) d\varepsilon = D(p)\abs{\frac{dp}{d\varepsilon}}d\varepsilon = D(p) \abs{\frac{d\varepsilon}{dp}}^{-1} d\varepsilon
\end{equation}
o sea
\begin{equation}\label{eq:D_vareps}
	D(\varepsilon) d\varepsilon = \frac{4\pi V\sqrt{2m^3}}{h^3} d\varepsilon
\end{equation}
con lo que hemos obtenido la distribución buscada.





%--------------------------------------
%  Cap 7
%--------------------------------------

\chapter{Gases de Fermi-Dirac y Bose-Einstein degenerados}
\chead[\thechapter. \textsc{Gases de Fermi-Dirac y Bose-Einstein degenerados}]{\thechapter. \textsc{Gases de Fermi-Dirac y Bose-Einstein degenerados}}

En el capítulo anterior vimos que, a temperaturas suficientemente bajas o densidades altas, la aproximación clásica dejaba de ser válida para el estudio de las propiedades termodinámicas de los gases.
La experiencia confirma esta predicción teórica, y así se observa, por ejemplo, en el caso de hidrógeno y del helio, que a bajas temperaturas las discrepancias entre los resultados experimentales y los predichos por la teoría clásica se vuelven muy importantes.
Hay que señalar que esta discrepancia no es meramente cuantitativa, sino que se presentan efectos que requieren una descripción mecánico-cuántica del modelo considerado para su simple explicación cualitativa.

En este capítulo vamos a estudiar algunos de estos efectos en el caso de gases ideales.
Es decir, de acuerdo con la nomenclatura introducida en el capítulo anterior, vamos a estudiar gases ideales fuertemente degenerados. Como veremos, los desarrollos y las propiedades que se obtienen son muy distintas en el caso de fermiones y en el caso de bosones, por lo que estudiaremos ambos por separado.

\newpage
\section{Gas de Fermi-Dirac. Cálculo de valores medios a T=0K.}

El número de fermiones en un estado de partícula $r$ viene dado de acuerdo con \eqref{eq:n_r_1} por
\begin{equation}\label{eq:n_r_t8}
	\expval{n_r} = \frac{1}{e^{\beta\varepsilon_r + \alpha} + 1}
\end{equation}

Sabemos, además, que el número de estados de traslación de una partícula en el intervalo de energía comprendido entre $\varepsilon$ y $\varepsilon+d\varepsilon$ viene dado por \eqref{eq:D_vareps}.
Este número no coincide, sin embargo, con el número de estados de un electrón en el intervalo de energía considerado, En efecto, la especificación del estado de un electrón ---y de toda partícula con espín--- exige conocer también su estado de espín, o sea la orientación del mismo.
En el caso de electrones, que tienen espín $\sfrac{1}{2}$, son posibles dos orientaciones o rotados de espín.
Como estamos considerando un sistema aislado, la energía de un electrón será independiente de la orientación de su espín, resultando que con cada estado de traslación son posibles dos estados de espín que corresponden a la misma energía.\footnote{Evidentemente, como ocurre para el momento cinético ordinario, para partículas que no tengan espín $\sfrac{1}{2}$, el número de estados de espín por cada estado de traslación no será 2, sino en general $2s + 1$, siendo $s$ el valor del número cuántico de espín.}

En resumen, el número de estados electrónicos con energía comprendida entre $\varepsilon$ y $\varepsilon+d\varepsilon$ será
\begin{equation}
	D(\varepsilon) d\varepsilon = 2\frac{4\pi V}{h^3} (2m^3)^{1/2} \varepsilon^{1/2} d\varepsilon
\end{equation}

Conocido el número de estados y el número medio de electrones en cada estado ---que depende solo de la energía del mismo---, podemos calcular la siguiente función:
\begin{equation}
	f(\varepsilon) d\varepsilon \equiv \parbox{28em}{número medio de electrones que en el sistema de volumen $V$ \\tiene una energía comprendida entre $\varepsilon$ y $\varepsilon+d\varepsilon$} 
\end{equation}
y que vendrá dada por
\begin{equation}\label{eq:f_e}
	f(\varepsilon) d\varepsilon = D(\varepsilon)\expval{n_r} d\varepsilon = \frac{8\pi V}{h^3} (2m^3)^{1/2} \frac{\varepsilon^{1/2}}{e^{\beta(\varepsilon_r -\mu)} + 1} \dd{\varepsilon}
\end{equation}

\begin{wrapfigure}{r}{0.35\textwidth}
	\centering
	\hspace{1.8cm}
	% This file was created by matplotlib2tikz v0.6.16.
\begin{tikzpicture}

\definecolor{color0}{rgb}{0.12156862745098,0.466666666666667,0.705882352941177}
\definecolor{color1}{rgb}{1,0.498039215686275,0.0549019607843137}

\begin{axis}[
scale = 0.65,
axis lines = left,
y = 0.85cm,
xmin=0, xmax=3.5,
ymin=0, ymax=4.3,
every axis x label/.style={at={(current axis.right of origin)},anchor=west},
ylabel style={at={(ticklabel* cs:0.82)},anchor=south west, rotate=-90},
ytick=\empty,
xtick={2},
xlabel = {$\varepsilon_r$},
ylabel = {$f(\varepsilon_r)$},
xticklabel = {$\mu_F$},
tick pos=left]

\node at (2.8,3) {$\mathbf{T_0 = 0K}$};
\node at (2.8,1.5) {$\color{color1}\mathbf{T > T_0}$};

\addplot [ultra thick, color1, forget plot]
table {%
0 0
0.0472972972972973 0.652400733301861
0.0945945945945946 0.922619809546396
0.141891891891892 1.12995191841971
0.189189189189189 1.30472396537706
0.236486486486487 1.45868024525444
0.283783783783784 1.59784102152928
0.331081081081081 1.72577918731048
0.378378378378378 1.84481860301694
0.425675675675676 1.9565684976473
0.472972972972973 2.06219451885098
0.52027027027027 2.16256922445519
0.567567567567568 2.25836133076218
0.614864864864865 2.35009122818548
0.662162162162162 2.43816662422642
0.70945945945946 2.5229057473735
0.756756756756757 2.60455228276139
0.804054054054054 2.68328444224755
0.851351351351351 2.75921955382751
0.898648648648649 2.83241493369093
0.945945945945946 2.90286540122466
0.993243243243243 2.97049752156059
1.04054054054054 3.03516046634594
1.08783783783784 3.09661325375866
1.13513513513514 3.15450806704296
1.18243243243243 3.20836938115471
1.22972972972973 3.25756879734636
1.27702702702703 3.30129587261222
1.32432432432432 3.33852594586442
1.37162162162162 3.36798715026668
1.41891891891892 3.38813062437796
1.46621621621622 3.39711052766342
1.51351351351351 3.39278382956826
1.56081081081081 3.37274365693506
1.60810810810811 3.3344033373232
1.65540540540541 3.27514935295824
1.7027027027027 3.19257739249026
1.75 3.08481319018989
1.7972972972973 2.95089657368637
1.84459459459459 2.79117475351894
1.89189189189189 2.60761811079186
1.93918918918919 2.4039548607329
1.98648648648649 2.18553811514009
2.03378378378378 1.95891779705724
2.08108108108108 1.7311763224055
2.12837837837838 1.50916547689593
2.17567567567568 1.29881427254418
2.22297297297297 1.1046477847626
2.27027027027027 0.929582110861403
2.31756756756757 0.774978038345978
2.36486486486486 0.640879018982362
2.41216216216216 0.526340409645743
2.45945945945946 0.429769501401771
2.50675675675676 0.34922349151626
2.55405405405405 0.282640877603157
2.60135135135135 0.228002913077345
2.64864864864865 0.183433958782042
2.69594594594595 0.147254435096629
2.74324324324324 0.11800029418219
2.79054054054054 0.0944208424913346
2.83783783783784 0.0754639281856427
2.88513513513514 0.0602548258556417
2.93243243243243 0.0480729598526281
2.97972972972973 0.0383289785586019
3.02702702702703 0.0305435615491269
3.07432432432432 0.0243286018854659
3.12162162162162 0.0193709491384576
3.16891891891892 0.0154186358196685
3.21621621621622 0.0122693732894764
3.26351351351351 0.00976104523843906
3.31081081081081 0.00776391530132588
3.35810810810811 0.00617427936728086
3.40540540540541 0.00490931963432403
3.4527027027027 0.00390294858807993
3.5 0.00310246240514403
};
\addplot [ultra thick, black, forget plot]
table {%
0 0
0.0408163265306122 0.606091526731326
0.0816326530612245 0.857142857142857
0.122448979591837 1.04978131833565
0.163265306122449 1.21218305346265
0.204081632653061 1.35526185435788
0.244897959183673 1.48461497791618
0.285714285714286 1.60356745147455
0.326530612244898 1.71428571428571
0.36734693877551 1.81827458019398
0.408163265306122 1.91662969499982
0.448979591836735 2.01017818278147
0.489795918367347 2.0995626366713
0.530612244897959 2.18529407725405
0.571428571428571 2.26778683805536
0.612244897959184 2.34738238930785
0.653061224489796 2.42436610692531
0.693877551020408 2.49897938350513
0.73469387755102 2.57142857142857
0.775510204081633 2.64189171555813
0.816326530612245 2.71052370871575
0.857142857142857 2.77746029931765
0.897959183673469 2.84282124887606
0.938775510204082 2.90671284991083
0.979591836734694 2.96922995583236
1.02040816326531 3.03045763365663
1.06122448979592 3.09047252182628
1.10204081632653 3.14934395500694
1.14285714285714 3.20713490294909
1.18367346938776 3.26390275965596
1.22448979591837 3.31970001103493
1.26530612244898 3.37457480314792
1.30612244897959 3.42857142857143
1.3469387755102 3.48173074484398
1.38775510204082 3.53409053624371
1.42857142857143 3.58568582800318
1.46938775510204 3.63654916038796
1.51020408163265 3.68671082873255
1.55102040816327 3.73619909446343
1.59183673469388 3.78504037128336
1.63265306122449 3.83325938999964
1.6734693877551 3.88087934491604
1.71428571428571 3.92792202424786
1.75510204081633 3.97440792664102
1.79591836734694 4.02035636556294
1.83673469387755 4.06578556307363
1.87755102040816 4.11071273426804
1.91836734693878 4.15515416349971
1.95918367346939 4.19912527334259
2 4.2
2.00001 0
};
\end{axis}

\end{tikzpicture}
	\vspace{-1cm}
\end{wrapfigure}
Notemos que para escribir esta expresión, hemos considerado a $\varepsilon_r$ en \eqref{eq:n_r_t8} como un parámetro continuo.
La justificación para ello es la proximidad de los niveles energéticos, que fue de hecho lo que nos permitió también definir una densidad de estados por unidad de intervalo de energía.

A partir de \eqref{eq:f_e} podemos escribir para el número de electrones que constituyen el sistema y para la energía media del gas las expresiones
\begin{align}
	N &= \int_{0}^{\infty} \dd{\varepsilon} f(\varepsilon) = \frac{8\pi V}{h^3} (2m^3)^{1/2} \int_{0}^{\infty} \dd{\varepsilon} \frac{\varepsilon^{1/2}}{e^{\beta\varepsilon_r + \alpha} + 1} \label{eq:N_FD}\\
	\expval{E} &= \int_{0}^{\infty} \dd{\varepsilon} \varepsilon f(\varepsilon) = \frac{8\pi V}{h^3} (2m^3)^{1/2} \int_{0}^{\infty} \dd{\varepsilon} \frac{\varepsilon^{3/2}}{e^{\beta\varepsilon_r + \alpha} + 1} \label{eq:E_med_FD}
\end{align}

Nosotros consideraremos siempre un sistema cerrado, de manera que la primera de estas ecuaciones es precisamente la que define el valor del nivel de Fermi $\mu_F$.

Si particularizamos estas dos ecuaciones para $T = 0 K$ resulta, en virtud de lo que acabamos de decir
\begin{align}
	N &= \frac{8\pi V}{h^3} (2m^3)^{1/2} \int_{0}^{\mu_F} \dd{\varepsilon} \varepsilon^{1/2} = \frac{16\pi V}{3h^3} (2m^3)^{1/2} \mu_F^{3/2}\\
	\expval{E} &=  \frac{8\pi V}{h^3} (2m^3)^{1/2} \int_{0}^{\mu_F} \dd{\varepsilon} \varepsilon^{3/2} = \frac{16\pi V}{5h^3} (2m^3)^{1/2} \mu_F^{5/2}
\end{align}

De la primera de estas ecuaciones se obtiene para la energía de Fermi
\begin{equation}
	\mu_F = \frac{h^2}{8m} \left( \frac{3N}{\pi V}\right)^{3/2}
\end{equation}
mientras que de la comparación de ambas resulta
\begin{equation}\label{eq:E_med_2_FD}
	\expval{E} = \frac{3}{5} N \mu_F
\end{equation}

\section{Propiedades termodinámicas}

Si ahora utilizamos las relaciones \eqref{eq:E_med_2_FD} y \eqref{eq:pv_t6} resulta la ecuación de la isoterma correspondiente a $T=0K$
\begin{equation}\label{eq:pv_t7}
	\expval{p}V = \frac{2}{3}\expval{E} = \frac{2}{5} N \mu_F
\end{equation}
que puede escribirse, al sustituir el valor de $\mu_F$, \eqref{eq:pv_t7}
\begin{equation}\label{eq:pv_2_t7}
	\expval{p} = \frac{1}{20} \left( \frac{3}{\pi} \right)^{2/3} \frac{h^2}{m} \left( \frac{N}{V} \right)^{5/3} \propto \left( \frac{N}{V} \right)^{5/3} 
\end{equation}

Así, pues, en el cero absoluto la presión de un gas de Fermi es proporcional a la potencia $\sfrac{5}{3}$ de la densidad.

Es fácil comprobar que, por existir un único estado accesible al sistema, la entropía de un gas de Fermi en el cero absoluto es nula.
En efecto, tenemos que\footnote{Esta demostración resulta más sencilla en el colectivo canónico generalizado. es decir, considerado a $\mu_0$ como definido por el foco con el que el sistema está en contacto, y sustituyendo a $N$ por $\expval{N}$. Con esta sustitución todas las fórmulas obtenidas siguen siendo válidas.
Desde luego pueden utilizarse también el colectivo canónico.}
\begin{align}
	S &= k_B(\ln Q + \beta \expval{E} - \beta \expval{N} \mu_F)= k_B \left( \frac{\expval{p}V}{k_B T} + \beta \expval{E} - \beta \expval{N} \mu_F \right) \nonumber \\
	  &= k_B \left( \frac{2}{5} \frac{\expval{N}\mu_F}{k_B T} + \frac{3}{5}\expval{N}\mu_F \beta - \beta \expval{N} \mu_F \right) = 0
\end{align}

\textit{[Y algo más, TBD]}

\section{Gas de Fermi-Dirac para T>0K. Cálculo de valores medios}

Para el cálculo del número de partículas recurrimos a la relación \eqref{eq:N_FD} y al desarrollo del \hyperref[Anx4]{Anexo 4} para ver que
\begin{align}
	N &= \frac{8\pi V}{h^3} (2m^3)^{1/2} \int_{0}^{\infty} \dd{\varepsilon} \frac{\varepsilon^{1/2}}{e^{\beta\varepsilon_r + \alpha} + 1}  \nonumber \\
	&= \frac{8\pi V}{h^3} (2m^3)^{1/2} \left[ \int_0^{\mu} \varepsilon^{1/2} + \frac{\pi^2}{6}\frac{1}{\beta^2}\frac{\mu^{-1/2}}{2} + \cdots \right] \nonumber \\
	&= \frac{8\pi V}{h^3} (2m^3)^{1/2} \left[ \frac{2}{3}\mu^{3/2} + \frac{\pi^2}{6}\frac{1}{\beta^2}\frac{\mu^{-1/2}}{2} + \cdots \right]  \nonumber \\
	&= \frac{8\pi V}{h^3} (2m^3)^{1/2} \left[ \frac{2}{3}\mu^{3/2} + \frac{\pi^2}{6}\frac{1}{\beta^2}\mu^{3/2}\mu^{-2} + \cdots \right] \\
	&= \frac{16\pi V}{3h^3} (2m^3)^{1/2} \mu^{3/2} \left[ 1 + \frac{\pi^2}{8}\left( \frac{k_B T}{\mu} \right)^2+ \cdots \right]  \nonumber
\end{align}
y de la misma forma con \eqref{eq:E_med_FD}
\begin{align}
	\expval{E} &= \frac{8\pi V}{h^3} (2m^3)^{1/2} \int_{0}^{\infty} \dd{\varepsilon} \frac{\varepsilon^{3/2}}{e^{\beta\varepsilon_r + \alpha} + 1} \nonumber \\
	&= \frac{8\pi V}{h^3} (2m^3)^{1/2} \left[ \int_0^{\mu} \varepsilon^{3/2} + \frac{\pi^2}{6}\frac{1}{\beta^2}\frac{\mu^{-1/2}}{2} + \cdots \right] \nonumber \\
	&= \frac{8\pi V}{h^3} (2m^3)^{1/2} \left[ \frac{2}{5}\mu^{5/2} + \frac{\pi^2}{6}\frac{1}{\beta^2}\frac{\mu^{-1/2}}{2} + \cdots \right]  \nonumber \\
	&= \frac{8\pi V}{h^3} (2m^3)^{1/2} \left[ \frac{2}{3}\mu^{3/2} + \frac{\pi^2}{6}\frac{1}{\beta^2}\mu^{3/2}\mu^{-2} + \cdots \right] \\
	&= \frac{16\pi V}{5h^3} (2m^3)^{1/2} \mu^{5/2} \left[ 1 + \frac{5\pi^2}{8}\left( \frac{k_B T}{\mu} \right)^2+ \cdots \right] \nonumber
\end{align}
\textit{[Revisar la ecuación]}

\section{Gas de Bose-Einstein degenerado. Condensación de Bose-Einstein}

Pasemos a considerar un gas ideal de bosones degenerado.
Como ya dijimos al principio de este capítulo, el comportamiento, en condiciones de degeneración, de un gas de Bose es muy distinto al de un gas de Fermi.
Esta diferencia es previsible, pues en un gas de Bose, al disminuir la temperatura, las partículas pueden agruparse en los estados de menor energía, mientras que en un gas de Fermi existen, incluso en el límite $T \rightarrow 0$, partículas en niveles relativamente altos de energía, debido al principio de exclusión de Pauli.
Más concretamente, sabemos que en el cero absoluto el límite superior de energías es la energía de Fermi.

El número medio de partículas en un estado cuántico $r$ es en el caso de bosones
\begin{equation}\label{eq:n_r_BE_t7}
	\expval{n_r} = \frac{1}{e^{\beta(\varepsilon_r -\mu)} - 1}
\end{equation}
y, por lo tanto, si el sistema contiene $N$ partículas debe cumplirse
\begin{equation}\label{eq:N_BE}
	N = \sum_r \frac{1}{e^{\beta(\varepsilon_r -\mu)} - 1}
\end{equation}

Recordemos que por considerar un sistema cerrado esta ecuación es precisamente la que nos determina el potencial químico $\mu$.

Supongamos ahora que el nivel más bajo de energía accesible a una partícula es $\varepsilon = 0$.\footnote{Esto no es una hipótesis, sino que coincide con nuestros resultados. En efecto, la energía asociada con el movimiento de traslación es la única energía que poseen las partículas de un gas ideal aislado. Cuando consideramos un sistema suficientemente grande es claro que el nivel más bajo de energía tiende a cero.}
A partir de \eqref{eq:n_r_BE_t7} obtenemos que
\begin{equation}\label{eq:mu_t_BE}
	\mu(T) \leq 0
\end{equation}
ya que en otro caso existirán estados ---en particular el o los correspondientes a $\varepsilon_r = 0$--- para los que $\varepsilon_r -\mu < 0$ y $\exp[\beta(\varepsilon_r -\mu)] < 1$, de manera que resultaría $\expval{n_r} < 0$, que no tiene sentido.
De hecho, para obtener \eqref{eq:n_r_BE_t7} nos fue necesario admitir que se cumplía \eqref{eq:cond_FD}, que se convierte en \eqref{eq:mu_t_BE}, al considerar que el valor mínimo de $\varepsilon_r$ es cero.

Vamos a tratar de estudiar el gas de Bose degenerado de manera análoga a como hemos realizado el análisis del gas de Fermi.
Para ello consideremos la expresión que se obtiene al explicitar en \eqref{eq:N_BE} la suma respecto de los estados de traslación, o sea la expresión análoga a \eqref{eq:N_FD}
\begin{align}\label{eq:N_2_BE}
	N &= \int_0^{\infty} \dd{\varepsilon} f(\varepsilon) = \int_0^{\infty} \dd{\varepsilon} D(\varepsilon) \expval{n_r}(\varepsilon) \nonumber \\
	&= g \frac{4\pi V}{h^3} (2m^3)^{1/2} \int_0^{\infty} \dd{\varepsilon} \frac{\varepsilon^{1/2}}{e^{\beta(\varepsilon -\mu)} - 1}
\end{align}

Las diferencias existentes respecto a \eqref{eq:N_FD} son dos: la sustitución de la distribución de Fermi por la de Bose y del factor 2 asociado a los electrones, para los que $s = 1/2$, por un factor de degeneración más genérico $g = 2s + 1$.

Si para una densidad dada del gas, $\sfrac{N}{V}$, disminuimos la temperatura ---aumentamos $\beta$--- las diferencias $\varepsilon -\mu$ tendrán que disminuir para que la integral que aparece en \eqref{eq:N_2_BE} puede conservar su valor.
Como $\varepsilon -\mu = \abs{\varepsilon} + \abs{\mu}$, resulta que, al disminuir la temperatura disminuye en valor absoluto $\mu$, que es negativo.
Como acabamos de ver que $\mu$ no puede ser positivo, resultará que $\mu$ irá aumentando hasta alcanzar el valor límite $\mu = 0$. La temperatura $T_0$ a la cual se alcanza este valor vendrá definida por la igualdad $\beta_0 = 1/k_BT_0$
\begin{align}
	N &= g \frac{4\pi V}{h^3} (2m^3)^{1/2} \int_0^{\infty} \frac{\varepsilon^{1/2}}{e^{\beta_0\varepsilon } - 1} \nonumber \\
	&= g \frac{4\pi V}{h^3} (2m^3)^{1/2} \beta_0^{-3/2} \int_0^{\infty} \dd{z} \frac{z^{1/2}}{e^z - 1}
\end{align}
donde hemos efectuado el cambio de variable $z = \varepsilon\beta_0$.
La integral que aparece en esta está resulta en el \hyperref[Anx5]{Anexo 5}, que nos permite escribir la anterior ecuación en la forma
\begin{equation}\label{eq:N/V_BE}
	\frac{N}{V} = \frac{2\pi \sqrt{\pi}}{h^3} (2m^3)^{1/2} \zeta(\sfrac{3}{2}) \beta_0^{-3/2}
\end{equation}
de donde se obtiene
\begin{equation}\label{eq:T_0_BE}
	T_0 = \frac{h^2}{2\pi mk_B} \left[ \frac{N}{gV\zeta(\sfrac{3}{2})} \right]^{2/3} = 3,31 \frac{\hbar^2}{km^{5/3}} \left[ \frac{Nm}{g} \right]^{2/3} 
\end{equation}

Resulta entonces que para temperaturas inferiores a $T_0$ la expresión \eqref{eq:N_2_BE} no tiene sentido, pues es imposible encontrar un valor de $\mu$ que la satisfaga ---evidentemente, para $\mu > 0$ la integral es divergente, pues el denominador tiende a cero cuando $\varepsilon$ tiende a $\mu$---.

Veamos otro modo de interpretar los resultados obtenidos.
Podemos imaginar ahora que mantenemos constante la temperatura, aumentando el número de partículas del sistema.
De acuerdo con \eqref{eq:N_2_BE}, aumento de $N$ exigirá una disminución de $\exp[\beta(\varepsilon_r -\mu)]$ y, por lo tanto, un aumento de $\mu$ que es negativo como sabemos.
¿Hasta qué valor podemos aumentar $N$ de modo que se satisfaga \eqref{eq:N_2_BE}? Pues hasta alcanzar el valor límite $\mu = 0$.
Es decir, que el valor del número máximo de partículas que según \eqref{eq:N_2_BE} podría tener el sistema a una temperatura $T$ vendría dado por
\begin{equation}\label{eq:N/V_max_BE}
	\frac{N_{max}}{V} = \frac{2\pi \sqrt{\pi}}{h^3} (2m^3)^{1/2} \zeta(\sfrac{3}{2}) \beta^{-3/2}
\end{equation}

Resultaría, según esto, que en un gas de bosones el número de partículas estaría acotado y, además, sería proporcional a $T^{3/2}$, lo que en particular implicaría que no podría existir un gas de bosones en el límite $T \rightarrow 0$.
Tratemos de entender lo que ha sucedido y volvamos a \eqref{eq:n_r_BE_t7} y \eqref{eq:N_BE}, haciendo tender $T$ a cero. Es evidente que
$$\expval{n_r}(\varepsilon_r - \mu) \rightarrow 0$$
y
$$N = g\expval{n_r} = g \frac{1}{e^{-\beta\mu} - 1}$$

En esta última ecuación siempre podría ajustarse $\mu$ de manera que se cumpliese la igualdad por pequeña que fuese $T$ ---$\beta$ grande---.
En particular, mientras no sea $T$ idénticamente cero es evidente que $\mu \rightarrow 0$ lleva a $N \rightarrow \infty$.

Resulta entonces que la.s dificultades han aparecido al pasar de \eqref{eq:N_BE} a \eqref{eq:N_2_BE}, es decir, al introducir la densidad de estados $D(\varepsilon)$ que es todo lo que necesitamos para pasar de una a otra.
En este paso se ha utilizado el valor
\begin{equation}\label{eq:D_vareps_BE}
	D(\varepsilon) \dd{\varepsilon} = g \frac{4\pi V\sqrt{2m^3}}{h^3} (2m^3)^{1/2} \varepsilon^{1/2} \dd{\varepsilon}
\end{equation}
que se obtiene al sustituir los niveles discretos de energía por una distribución continua sobre toda la recta real.
Si en esta ecuación tomamos el límite $\varepsilon \rightarrow 0$, Y debido al factor $\varepsilon^{1/2}$, obtenemos
$$\lim\limits_{\varepsilon \rightarrow 0} D(\varepsilon) = 0$$
y análogamente en el caso del número de partículas por unidad de intervalo de energía:
$$\lim\limits_{\varepsilon \rightarrow 0} f(\varepsilon) = 0$$

Así, pues, al admitir el paso a un espectro continuo en el sentido considerado hasta ahora, estamos implícitamente admitiendo que el número medio de partículas en el estado ---o los estados--- de energía $\varepsilon \rightarrow 0$ es nulo.
Este hecho no tiene importancia en la estadística de Fermi, donde en cada estado puede haber como máximo una partícula, ni en la estadística de Bose a temperaturas altas, en que el número de partículas en $\varepsilon \rightarrow 0$ es muchísimo menor que el número total de partículas que contiene el sistema.
Sin embargo, en el caso de un gas de Bose a temperaturas suficientemente bajas, la existencia de un estado de traslación que corresponde a $\varepsilon = 0$ es de fundamental importancia puesto que, como hemos visto antes, las partículas tienden a agruparse precisamente en ese nivel energético.

Para solucionar esta dificultad hemos de escribir el sumatorio \eqref{eq:N_BE} en la forma
\begin{equation}\label{eq:N_3_BE}
	N = g \frac{1}{e^{-\beta\mu} - 1} + \sum_{\substack{r\\(\varepsilon_r \neq 0)}} \frac{1}{e^{\beta(\varepsilon_r -\mu)} - 1}
\end{equation}

El segundo sumando del esta expresión es el que puede aproximarse por una integral, de manera que en lugar de \eqref{eq:N_2_BE} la expresión correcta será
\begin{equation}\label{eq:N_4_BE}
	N = N_0 + N' 
\end{equation}
donde $N_0$ representa el número de partículas en el nivel fundamental $(\varepsilon = 0)$ y $N'$ el número de partículas en estados excitados ---$\varepsilon > 0$---. Es decir,
\begin{equation}
	N_0 = g \frac{1}{e^{-\beta\mu} - 1}
\end{equation}
\begin{equation}
	N' = g \frac{4\pi V\sqrt{2m^3}}{h^3} (2m^3)^{1/2} \int_{0}^{\infty} \frac{\varepsilon^{1/2}}{e^{\beta(\varepsilon -\mu)} - 1}
\end{equation}

El estudio de las propiedades de un gas de Bose degenerado sobre todo el rango de temperaturas es muy complicado y requiere el empleo del cálculo numérico, por lo que nos vamos a limitar aquí a una discursión cualitativa.

a) Consideremos el gas de Bose a una cierta densidad $\sfrac{N}{V}$ y a una temperatura $T \gg T_0$, siendo $T_0$ la temperatura definida por \eqref{eq:T_0_BE}.
A partir de \eqref{eq:N/V_BE} y \eqref{eq:N/V_max_BE} obtenemos que
\begin{equation}\label{eq:N/N_max}
	\frac{N_{max}}{N} = \left( \frac{T}{T_0} \right)^{3/2}
\end{equation}
que en la zona que estamos considerando es mucho mayor que la unidad.
Es decir, que para $T \gg T_0$ el número máximo de partículas que pueden existir en estados excitados es mucho mayor, que $N$.
Resulta entonces que prácticamente todas las partículas se encuentran en estados excitados, lo que corresponde a $\abs{\mu} \gg T_0$ y
\begin{equation}
	N_0 = g \frac{1}{e^{-\beta\mu} - 1} \ll 1
\end{equation}

b) Disminuyamos ahora la temperatura, manteniendo $\sfrac{N}{V}$ constante.
Al alcanzar $T = T_0$ se hace $N = N'_{max}$.
Hay que observar ahora claramente que $N = N'_{max}$ no implica que $\mu$ sea nulo.
Ambas cosas eran equivalentes cuando no considerábamos ---erróneamente--- el estado fundamental, pero una vez introducida la descomposición \eqref{eq:N_4_BE}, aunque sea $N = N'_{max}$ el potencial químico será todavía distinto a cero ---aunque muy pequeño--- y el número de partículas en estados excitados será menor que $N'_{max}$

c) Si seguimos disminuyendo la temperatura por trabajo de $T_0$, $N'_{max}$ continuará disminuyendo y en consecuencia el número de partículas en el estado fundamental irá aumentando.
Para temperaturas menores que $T_0$ puede escribirse
\begin{equation}
	N_0 = N - N' \simeq N - N'_{max}
\end{equation}
o utilizando \eqref{eq:N/N_max}
\begin{equation}
	N_0 = N  \left[ 1 - \left( \frac{T}{T_0} \right)^{3/2} \right]
\end{equation}

\begin{wrapfigure}{l}{0.4\textwidth}
	\centering
	% This file was created by matplotlib2tikz v0.6.16.
\begin{tikzpicture}

\definecolor{color0}{rgb}{0.12156862745098,0.466666666666667,0.705882352941177}

\begin{axis}[
scale = 0.55,
xmin=0, xmax=1.05,
ymin=0, ymax=1.05,
tick align=outside,
tick pos=left,
ylabel = {$N_0/N$},
xlabel = {$T/T_0$},
x grid style={white!69.01960784313725!black},
y grid style={white!69.01960784313725!black}
]
\addplot [ultra thick, color0, forget plot]
table {%
0 1
0.0204081632653061 0.997084548104956
0.0408163265306122 0.991753856779166
0.0612244897959184 0.984850867572284
0.0816326530612245 0.97667638483965
0.102040816326531 0.967404256887758
0.122448979591837 0.957151782925076
0.142857142857143 0.946005075284396
0.163265306122449 0.934030854233325
0.183673469387755 0.921282798833819
0.204081632653061 0.907805316030076
0.224489795918367 0.893635939667903
0.244897959183673 0.878806940578271
0.26530612244898 0.863346453116525
0.285714285714286 0.847279290335758
0.306122448979592 0.830627550457402
0.326530612244898 0.813411078717201
0.346938775510204 0.795647826135568
0.36734693877551 0.777354133037472
0.387755102040816 0.758544956480254
0.408163265306122 0.739234055102065
0.428571428571429 0.719434141125153
0.448979591836735 0.699157006658556
0.469387755102041 0.678413629632092
0.489795918367347 0.657214263400605
0.510204081632653 0.635568513119533
0.530612244897959 0.613485401302005
0.551020408163265 0.590973424451665
0.571428571428571 0.568040602275169
0.591836734693878 0.54469452067959
0.612244897959184 0.520942369529009
0.63265306122449 0.496790975954138
0.653061224489796 0.4722468338666
0.673469387755102 0.447316130216458
0.693877551020408 0.422004768440991
0.714285714285714 0.396318389479631
0.73469387755102 0.370262390670554
0.755102040816326 0.343841942795819
0.775510204081633 0.317062005501979
0.795918367346939 0.289927341289978
0.816326530612245 0.262442528240611
0.836734693877551 0.234611971618814
0.857142857142857 0.20643991448067
0.877551020408163 0.177930447390711
0.897959183673469 0.149087517343221
0.918367346938775 0.119914935969471
0.938775510204082 0.0904163871027338
0.959183673469388 0.0605954337642884
0.979591836734694 0.0304555246261679
1 0
};
\end{axis}

\end{tikzpicture}
	\vspace{-0.1cm}
\end{wrapfigure}

Además, a estas temperaturas el potencial químico $\mu$ es ya prácticamente nulo.
Cuando más se disminuye la temperatura más y más partículas se agrupan en el nivel fundamental ---$\varepsilon = 0$---.
En la figura se representa el comportamiento de $\sfrac{N_0}{N}$ frente a $\sfrac{T}{T_0}$.

Este fenómeno recibe el nombre de \emph{condensación de Bose-Einstein}, denominándose a $T_0$ \emph{temperatura de condensación}, ya que en cierto sentido recuerda el proceso de condensación de un vapor en la fase líquida.
Sin embargo, estos dos procesos son conceptualmente muy diferentes.
En la condensación normal de un gas, las partículas se agrupan en un cierto volumen que depende de la densidad del líquido y de la cantidad de gas condensado, lo que se traduce en una acumulación de puntos representativos en el espacio de las fases en una zona correspondiente a un intervalo limitado de posiciones.
Por el contrario, en la condensación de Base-Einstein, la agrupación tiene lugar en el espacio de momentos, concretamente alrededor del valor $p = 0$.

\section{Propiedades del gas de Bose para T<T\textsubscript{0}}

Para temperaturas inferiores a $T_0$ sabemos que $\mu$ es muy pequeño de forma que puede escribirse con buena aproximación
\begin{equation}
	\expval{E} = g \frac{4\pi V}{h^3} (2m^3)^{1/2} \int_0^{\infty} \dd{\varepsilon} \frac{\varepsilon^{1/2}}{e^{ \beta\epsilon} - 1}
\end{equation}
donde hemos tenido en cuenta que las $N_0$ partículas existentes en el estado fundamental poseen una energía nula.
Haciendo el cambio de variable $z = \beta\varepsilon$ obtenemos
\begin{equation}
	\expval{E} = g \frac{4\pi V}{h^3} (2m^3)^{1/2} \beta^{-5/2} \int_0^{\infty} \dd{z} \frac{z^{3/2}}{e^z - 1}
\end{equation}
que podemos resolver con el \hyperref[Anx5]{Anexo 5}
\begin{equation}
	\expval{E} = g \frac{4\pi V}{h^3} (2m^3)^{1/2} \, \beta^{-5/2} \, \Gamma\left( \sfrac{5}{2} \right) \zeta\left( \sfrac{5}{2} \right) 
\end{equation}

Si recordamos ahora la definición de $T_0$, podemos reescribirla como
\begin{equation}
	\expval{E} = \frac{5}{2} Nk_BT \left( \frac{T}{T_0} \right)^{3/2} \frac{\zeta\left( \sfrac{5}{2} \right)}{\zeta\left( \sfrac{3}{2} \right)} \approx 0.770 Nk_BT \left( \frac{T}{T_0} \right)^{3/2}
\end{equation}

A partir de esta expresión obtenemos para la capacidad calorífica a volumen constante
\begin{equation}
	c_V = \pdv{\expval{E}}{V} = 1.925 \left( \frac{T}{T_0} \right)^{3/2}
\end{equation}



%--------------------------------------
%  Cap 8
%--------------------------------------

\chapter{Radiación electromagnética}
\chead[\thechapter. \textsc{Radiación electromagnética}]{\thechapter. \textsc{Radiación electromagnética}}

En este capítulo vamos a analizar dos problemas aparentemente dispares, el estudio estadístico de la radiación electromagnética y las propiedades termodinámicas de los sólidos, pero que, desde el punto de vista físico y de la manera de abordarlos, tienen mucho en común:
\begin{enumerate}
	\item En ambos casos, la base del problema físico es un fenómeno ondulatorio: oscilaciones del campo electromagnético, en el caso de la radiación, y vibraciones alrededor de sus posiciones de equilibrio de los átomos que conforman el cristal.

	\item La descripción cuántica de los dos fenómenos se puede hacer en términos del sencillo oscilador armónico y de sus niveles de energía discretos. Desde un punto de vista práctico, la suma sobre los estados discretos de energía puede aproximarse por una integral sobre una distribución continua de frecuencias y, en las dos situaciones la densidad de estados es, esencialmente, la misma.
	
	\item Desde un punto de vista cuántico fundamental, la descripción del estado del sistema se puede hacer mediante el concepto de «cuasi-partícula»: el fotón en el caso de la radiación y el fonón en el de los sólidos, bosones en los dos casos. El sistema se puede considerar como un gas ideal de fotones o de fonones, según el caso, que obedecen a la estadística de Base-Einstein con potencial químico nulo.

	\item Desde un punto de vista histórico, la solución correcta a los dos problemas se enmarca en los comienzos de la moderna Teoría Cuántica, cuando las ideas de cuantificación de la energía fueron introducidas, a comienzos del siglo XX. La comprobación experimental de las predicciones teóricas supuso un importante impulso para la aceptación de las nuevas ideas.
\end{enumerate}

\newpage
\section{Radiación electromagnética}

Como es sabido, se conoce con el nombre genérico de radiación a una de las formas posibles de propagación de la energía, una de cuyas características esenciales es que no necesita la presencia de un medio material.

La teoría clásica considera a la radiación como un fenómeno ondulatorio regido por las ecuaciones de Maxwell del Electromagnetismo, es decir, como un conjunto de ondas planas electromagnéticas.
Es importante señalar que estas ondas electromagnéticas son transversales, es decir, que la perturbación se produce en un plano perpendicular a la dirección de propagación, punto sobre el que más adelante volveremos.
Esta teoría se basa en la observación de que las ondas electromagnéticas se propagan a la velocidad de la luz, que es un ejemplo típico de radiación, y en que, además, no necesitan medio material para propagarse, al contrario de lo que sucede en las ondas elásticas.
Mediante esta teoría se puede justificar la existencia y propiedades de fenómenos bien conocidos, como la interferencia y la difracción.
Los distintos procesos de radiación, tales como la propagación de la luz, emisión de rayos X o la transmisión de señales de radio, aparecen entonces como casos concretos del fenómeno general de la radiación, en el sentido de que corresponden a valores distintos de la frecuencia, pero teniendo en común el que se propagan con la velocidad de la luz.
Dicho de otro modo, cada proceso de radiación puede caracterizarse simplemente por su longitud de onda $\lambda$, o lo que es equivalente, por su frecuencia $\nu$
\begin{equation}\label{eq:freq_t8}
	\nu = \frac{c}{\lambda}
\end{equation}
o su frecuencia angular $\omega$
\begin{equation}
	\omega = 2\pi\nu
\end{equation}

Las longitudes de onda asociadas con los fenómenos de radiación que se presentan en la naturaleza, se distribuyen en un amplísimo margen que va desde valores del orden de $\lambda = 10^{-14}$m e inferiores, para los rayos gamma, hasta $\lambda = 10^{-6}$m y superiores correspondientes ondas de radiofrecuencia.
La denominada radiación térmica, que puede ser detectada por nuestros sentidos, corresponde a los valores de $\lambda$ comprendidos entre $10^{-7}$ y $10^{-4}$m, e incluye una parte de la radiación ultravioleta y de la infrarroja.

El primer intento de explicar las propiedades de la radiación a partir de un modelo estadístico se debió a Rayleigh y Jeans, quienes consideraron a la radiación contenida en un recinto en equilibrio a la temperatura $T$ como una superposición de ondas planas electromagnéticas y utilizaron las técnicas de la Mecánica Estadística Clásica.
De este modo, obtuvieron para la densidad espectral de energía, o sea para la energía media asociada con las ondas electromagnéticas cuya frecuencia angular estuviese comprendida entre $\omega$ y $\omega + d\omega$, la expresión
\begin{equation}\label{eq:E_med_t8}
	\expval{E}(\omega) \dd{\omega}= V \frac{k_B T}{\pi^2 c^3} \omega^2 \dd{\omega}
\end{equation}
donde $V$ es el volumen del recinto considerado. Para valores pequeños de $\omega$ esta expresión presenta un comportamiento acorde con los datos experimentales, pero a altas frecuencias el valor de $\expval{E}(\omega)$ determinado experimentalmente presenta un descenso exponencial con $\omega$, mientras que la expresión \eqref{eq:E_med_t8} predice un aumento proporcional a $\omega^2$ para todos los valores de $\omega$.
De hecho, es fácil prever que la expresión \eqref{eq:E_med_t8} no puede se correcta para valores grandes de la frecuencia angular $\omega$. En efecto, de acuerdo con ella, resulta que la energía total de radiación contenida en un recinto vendrá dada por
\begin{equation}\label{eq:E_tot_t8}
	\expval{E}_{total} =\int_{0}^{\infty} \dd{\omega} \expval{E}(\omega) \propto \eval{\omega^3}_{0}^{\infty} \rightarrow \infty
\end{equation}
es decir, tiende a infinito. Esto es una consecuencia directa del hecho de que
\begin{equation}
	\lim\limits_{\omega\rightarrow\infty} \expval{E}(\omega) \neq 0
\end{equation}

Este resultado anómalo, y durante mucho tiempo inexplicable, se conoce con el nombre de \emph{catástrofe ultravioleta}, por tener su origen en el comportamiento de \eqref{eq:E_med_t8} para valores grandes de $\omega$, o sea, para valores pequeños de $\lambda$, que es la zona del espectro donde se sitúa la radiación ultravioleta.
Hoy día se sabe que las causas de la anomalía están en la incapacidad intrínseca de la Mecánica Clásica para dar una explicación teórica correcta del fenómeno de radiación.
Más aún, los orígenes históricos de la Mecánica Cuántica pueden situarse en este problema concreto.
En efecto, fue precisamente en la solución empírica a este problema dada por Planck, cuando se introdujo por primera vez, en Física una cuantificación de la energía. 
La expresión obtenida por Planck no presentaba la catástrofe ultravioleta, y además concordaba con los resultados experimentales. Este éxito llevó a intentar un análisis semejante de otras paradojas existentes en Mecánica Clásica, lo que condujo al desarrollo de ia Mecánica Cuántica.

Pasemos ahora directamente a un tratamiento cuántico del problema.
Para ello necesitamos utilizar algunos conceptos fundamentales y bien conocidos de Mecánica Cuántica.

Sabemos que la teoría cuántica asigna a la materia, de acuerdo con los resultados experimentales, un doble carácter, en virtud del cual presenta tanto propiedades corpusculares o de partículas, como propiedades ondulatorias.
Es la denominada \emph{dualidad onda-corpúsculo}.
Así, por ejemplo, la radiación presenta propiedades, como la interferencia o la difracción, que son típicamente ondulatorias, y propiedades que exigen una interpretación corpuscular, corno el efecto fotoeléctrico y el efecto Compton.
El paso de una. descripción a otra, es decir, de ondas a partículas y viceversa, se efectúa mediante las relaciones de Einstein y de Broglie
\begin{align}
	\varepsilon = h\nu = \hbar\omega \label{eq:eps_t8}\\
	\vb{p} = \frac{h\bm{\kappa}}{2\pi} = \hbar\bm{\kappa} \label{eq:p_t8}
\end{align}

Aquí $\varepsilon$ y $\mathbf{p}$ son la energía y el momento o cantidad de movimiento de las partículas asociadas, mientras que $\nu$ y $\bm{\kappa}$, representan, respectivamente, la frecuencia y el vector de onda de la onda electromagnética, Este último se define de modo que su dirección coincida con la de propagación de la onda, mientras que su módulo viene dado por
\begin{equation}\label{eq:valor_kappa_t8}
	\abs{\bm{\kappa}} = \frac{2\pi}{\lambda} = \frac{\omega}{c}
\end{equation}

Las partículas asociadas a la radiación electromagnética se denominan \emph{fotones} y es fácil ver que, por poseer una velocidad igual a la de la luz, han de poseer una masa en reposo nula.
En efecto, a partir de la relación relativista
\begin{equation}
	m = \frac{m_0}{\sqrt{1 - \left(\frac{v}{c}\right)^2}}
\end{equation}
donde $v$ es la velocidad de la partícula, $m_0$ su masa en reposo y $m$ la masa relativista, se deduce que, si $m_0$ fuese distinta de cero, al ser $v = c$ tendería $m$ a infinito.
A partir de \eqref{eq:p_t8}, \eqref{eq:valor_kappa_t8}, \eqref{eq:freq_t8} y \eqref{eq:eps_t8} se tiene que
\begin{equation}
	\abs{\vb{p}} = \hbar \abs{\bm{\kappa}} = \frac{2\pi \hbar}{\lambda} = \frac{h\nu}{c} = \frac{\varepsilon}{c}
\end{equation}
y, por tanto,
\begin{equation}
	m = \frac{\abs{\vb{p}}}{c} = \frac{h\nu}{c^2} = \frac{\varepsilon}{c^2}
\end{equation}

que nos indica qué en general $m$ no es ni nula ni infinita.

Así pues, mientras que en la descripción ondulatoria se considera el campo electromagnético como una superposición de ondas planas, en la descripción corpuscular se le considera compuesto por un conjunto de fotones.
Como ya dijimos al principio, las ondas planas electromagnéticas son transversales.
Además, la perturbación que se propaga consiste en oscilaciones de los campos eléctrico \textbf{E} y magnético \textbf{H}, los cuales a su vez son perpendiculares entre sí en cada punto de las onda.

Restringiéndonos al campo eléctrico ---el campo magnético en cada punto es función del campo eléctrico en ese mismo punto--- una onda plana tiene la forma\footnote{Como siempre, únicamente tiene significado físico la parte real o la parte imaginaria por separado.}
\begin{equation}
	\vb{E}(\vb{r}, t) = \vb{E}_0 e^{i(\bm{\kappa}\cdot\vb{r}-\omega t)}
\end{equation}

La especificación de una onda plana exige, en principio, la determinación de $\vb{E}_0$, $\bm{\kappa}$ y $\omega$.
Ahora bien, una vez dado $\bm{\kappa}$, resulta que, por \eqref{eq:valor_kappa_t8}, también está dada su frecuencia,\footnote{ Esta no es una propiedad general de las ondas, sino solo de aquellas en las que la velocidad de todas las ondas planas es la misma ---onda no dispersiva---, como sucede con las ondas electromagnéticas.} luego sólo son independientes $\vb{E}_0$ Y $\bm{\kappa}$.
Por otro lado, si utilizamos el hecho de que las ondas son transversales, o sea que $\vb{E}_0$ es perpendicular a $\bm{\kappa}$ podemos tomar dos vectores unitarios $\vb{e}_1$ y $\vb{e}_2$ en el plano perpendicular a $\bm{\kappa}$, y escribir
\begin{equation}\label{eq:E_1_t8}
	\vb{E}(\vb{r}, t) = E_{01} \vb{e}_1 e^{i(\bm{\kappa}\cdot\vb{r}-\omega t)} + E_{02} \vb{e}_2 e^{i(\bm{\kappa}\cdot\vb{r}-\omega t)}
\end{equation}
con lo que resulta que, una vez fija la dirección de $\bm{\kappa}$ únicamente son posibles dos direcciones independientes de polarización de \textbf{E}. Cualquier otra dirección es combinación de estas dos.

Cuando se pasa a una descripción corpuscular, a cada onda plana independiente se le asocia un estado de fotón independiente, de acuerdo con las relaciones \eqref{eq:eps_t8} y \eqref{eq:p_t8}.
La amplitud de la onda mide el número de fotones que se encuentra en el estado asociado.
Según todo esto, ¿qué será necesario especificar para determinar cada uno de los estados independientes de un fotón?
Pues su cantidad de movimiento \textbf{p} ---que se obtiene a partir del vector de onda $\bm{\kappa}$--- y su estado de polarización, teniendo en cuenta que, una vez fijado \textbf{p} o $\bm{\kappa}$, sólo son posibles dos orientaciones independientes.
Es decir, que a cada valor de \textbf{p} corresponde dos estados independientes de un fotón con las dos polarizaciones posibles.

Veamos qué propiedades deben tener los fotones:
\begin{enumerate}
	\item Como las ondas planas son indistinguibles, los fotones han de ser partículas cuánticamente idénticas, diferenciándose únicamente por el estado en que se encuentran.
	
	\item Dado que las amplitudes con que aparecen las ondas planas en una superposición son totalmente arbitrarias, deducimos que el número de fotones existentes en un estado dado puede ser tan grande como queramos, y en consecuencia, los fotones obedecerán a la estadística de Bose-Einstein, es decir, serán \emph{bosones}.
	
	\item Una característica de las ondas electromagnéticas es que son absorbidas y emitidas por la materia.
	Los fotones aparecerán y desaparecerán cuando la radiación sea, respectivamente, emitida y absorbida por la materia.
	En consecuencia, su número, incluso en un sistema cerrado, no será constante.
	
	\item Veamos ahora si los fotones se comportan o no como un gas ideal, es decir, si existen interacciones entre ellos.
	La respuesta está en el principio de superposición o, lo que es equivalente, en el carácter lineal de la ecuación que gobierna el comportamiento de las ondas ---ecuación de onda---, y puede verse claramente mediante un sencillo razonamiento.
	En efecto, sabemos que cuando dos ondas se encuentran en una cierta región del espacio, no se distorsionan, sino que actúan aditivamente, de manera que, por ejemplo, el campo eléctrico resultante en cada punto es la suma de los campos 	eléctricos asociados a cada una de las ondas en ese punto.
	Concluimos entonces que los fotones no interaccionan entre sí, aunque, como inmediatamente veremos, sí que lo hacen con otros cuerpos extraños que encuentren en su camino.
\end{enumerate}

En resumen, se obtiene la conclusión de que \emph{un gas de fotones constituye un gas ideal de Bose.}
Este resultado puede entenderse como un postulado basado en las propiedades del campo electromagnéticos, es decir, en las ecuaciones de Maxwell.

\section{Distribución de Plank: Ley de Rayleigh-Jeans y ley de Wien}

La radiación electromagnética que existe en equilibrio dentro de un recinto de volumen $V$, cuyas paredes se mantienen a la temperatura $T$, puede tratarse como un sistema termodinámico en equilibrio, al que le son aplicables las leyes de la Termodinámica y de la Mecánica Estadística.
De hecho, hemos visto en el apartado anterior que la radiación puede considerarse corno un gas de fotones.
Hay que señalar, sin embargo, que existen profundas diferencias entre un gas molecular y un gas de fotones.
En un gas molecular sabemos que la distribución de velocidades que caracteriza el equilibrio es la de Maxwell-Boltzmann, mientras que en un gas de fotones la situación será evidentemente distinta, ya que todos los fotones se mueven con la velocidad de la luz, independientemente de las condiciones en que se encuentre el gas.
Veremos que lo que caracteriza el equilibrio en el caso de los fotones es la distribución existente entre las frecuencias $\omega$.
Entonces, si se consideran, por ejemplo, dos estados de equilibrio de un mismo recinto a dos temperaturas diferentes, lo que se modifica al pasar de uno a otro es la distribución de frecuencias, es decir el número de fotones existentes en cada intervalo de frecuencias.

Otra importante diferencia de principio surge si se considera el mecanismo que permite alcanzar y mantener la situación de equilibrio.
En el caso de los gases moleculares, dicho mecanismo lo constituyen las colisiones entre las moléculas.
A este respecto conviene recordar que cuando hablábamos de un gas ideal, nos referíamos a un sistema de partículas cuyas energías de interacción eran cuantitativamente despreciables.

El estudio detallado de la interacción de la radiación con la materia, es decir, la absorción y emisión de fotones, constituye un problema en el que intervienen de modo fundamental las propiedades de los átomos que constituyen el cuerpo emisor.
Sin embargo, aquí lo único que vamos a necesitar es admitir que la materia presente, constituida por las paredes del sistema o por cualquier otro cuerpo, emite y absorbe fotones, de manera que se establece una situación de equilibrio, la cual se mantiene mientras no se alteren las condiciones externas del sistema.
Un gas de fotones en equilibrio en el interior de un recinto se denomina \emph{radiación negra} o \emph{radiación del cuerpo negro}

El hecho de que el número de fotones que existen en el sistema varíe en virtud de un proceso que tiene lugar en el interior del mismo, y no como consecuencia de un intercambio con otro sistema, hace que la distribución estadística de los fotones sea un caso muy especial de la distribución de Bose.
En efecto, hemos admitido que se establece una situación de equilibrio, la cual depende de las condiciones externas en que se encuentra el sistema.
Esto quiere decir que el número de partículas que existen en el sistema, cuando está en equilibrio, no es una variable independiente, sino que será una función de las condiciones externas.
En el colectivo canónico estas condiciones externas son la temperatura y el volumen, de manera que resulta que si ambas se mantienen constantes ha de cumplirse en el equilibrio que
\begin{equation}\label{eq:dF_t8}
	(dF)_{T,V} = 0
\end{equation}
ya que en general se tiene $dF = -SdT - pdV$.

Consideremos ahora una situación general en la que modificaremos el número de partículas, manteniendo constantes la temperatura y el volumen.
Entonces,
\begin{equation}\label{eq:dF_2_t8}
	(dF)_{T,V} = \left( \pdv{F}{N} \right)_{T,V} dN
\end{equation}

Comparando \eqref{eq:dF_t8} y \eqref{eq:dF_2_t8}, resulta
\begin{equation}
	\left( \pdv{F}{N} \right)_{T,V, \expval{N}} = 0
\end{equation}
donde, al indicar que la derivada se calcula a $N = \expval{N}$, estamos explicitando que se trata de un estado de equilibrio.

Dado que $\left( \pdv{F}{N} \right)_{T,V, \expval{N}}$ es el potencial químico $\mu$ resulta que, para un gas de fotones, se tiene que
\begin{equation}
	\mu = 0
\end{equation}

Esta es la característica especial a que nos hemos referido antes.
Si tenemos en cuenta que la energía de un fotón en el estado $r$ es, de acuerdo con \eqref{eq:eps_t8}
\begin{equation}
	\varepsilon_r = h\nu_r = \hbar\omega_r
\end{equation}
obtenemos, a partir de \eqref{eq:Q_BE}
\begin{equation}
	\ln Q_F = -\sum_r \ln (1 - e^{-\beta\hbar\omega_r})
\end{equation}
y a partir de \eqref{eq:n_r_BE}
\begin{equation}\label{eq:n_r_Fot}
	\expval{n_r} = \frac{1}{e^{-\beta\hbar\omega_r} - 1}
\end{equation}
La ecuación \eqref{eq:n_r_Fot} se conoce con el nombre de \emph{distribución de Planck} para el gas de fotones.

El paso siguiente consiste en la determinación de los estados cuánticos accesibles a un fotón.
El estudio es completamente análogo al realizado en la sección 6.5.
Si admitimos que la longitud de onda de las ondas electromagnéticas es mucho menor que la dimensión lineal más pequeña del recinto, podremos despreciar los efectos de las paredes y escoger las condiciones en los límites más convenientes para nuestros cálculos.
Vamos entonces a considerar un paralelepípedo de dimensiones $L_x$, $L_y$ y $L_z$, y a tomar condiciones periódicas en los límites, o sea vamos a admitir que se cumple que
\begin{equation}
	\mathbf{E}(x, y, z) = \mathbf{E}(x + L_x, y, z) = \mathbf{E}(x, y + L_y, z) = \mathbf{E}(x, y, z + L_z)
\end{equation}

Al exigir que \eqref{eq:E_1_t8} verifique estas condiciones, resulta que han de satisfacerse las relaciones siguientes:
\begin{equation}
	\kappa_x L_x = 2\pi n_x \qquad | \qquad \kappa_y L_y = 2\pi n_y \qquad | \qquad \kappa_z L_z = 2\pi n_z
\end{equation}
siendo $n_x$, $n_y$ y $n_z$ enteros.
Consecuentemente, el número de ondas planas con polarización $\alpha$ y vector de onda comprendido entre $\kappa$ y $\kappa+d\kappa$, vendrá dado por
\begin{equation}
	D(\bm{\kappa}, \alpha) \dd[3]{\bm{\kappa}} = \left( \frac{L_x d\kappa_x}{2\pi}\right) \left( \frac{L_y d\kappa_y}{2\pi}\right) \left( \frac{L_z d\kappa_z}{2\pi}\right) = \frac{V}{(2\pi)^3} \dd[3]{\bm{\kappa}}
\end{equation}
y a partir de aquí obtenemos para el número de ondas planas con polarización $\alpha$ y vector de onda cuyo módulo esté comprendido entre $\kappa$ y $\kappa+d\kappa$,
\begin{equation}
	D(\kappa, \alpha) \dd{\kappa} = 4\pi \kappa^2 \dd{\kappa} D(\bm{\kappa}, \alpha) = \frac{V}{(2\pi)^2} \kappa^2 \dd{\kappa}
\end{equation}

Finalmente, si introducimos la frecuencia angular $\omega = \kappa c$ y multiplicamos por 2 para tener en cuenta las dos direcciones independientes de polarización de una onda plana, obtenemos para el número de estados cuánticos de fotón con frecuencia comprendida en el intervalo entre $\omega$ y $\omega + d\omega$ la expresión
\begin{equation}\label{eq:d_omega_t8}
	D(\omega) \dd{\omega} = \frac{V}{\pi^2}\frac{\omega^2 dd{\omega}}{c^3}
\end{equation}

A partir de \eqref{eq:n_r_BE} y \eqref{eq:d_omega_t8} se obtiene que el número medio de fotones en el intervalo de frecuencias considerado es
\begin{equation}\label{eq:f_omega_t8}
	f(\omega) \dd{\omega} = \expval{n_r}(\omega) D(\omega) \dd{\omega} = \frac{V}{\pi^2 c^3}\frac{\omega^2 dd{\omega}}{e^{\hbar\omega / k_BT} - 1}
\end{equation}
y, en consecuencia, la distribución espectral de energía resulta finalmente
\begin{equation}\label{eq:E_med_t8}
	\expval{E} = \frac{V}{\pi^2 c^3}\frac{\omega^3 dd{\omega}}{e^{\hbar\omega / k_BT} - 1}
\end{equation}
expresión que suele denominarse \emph{fórmula de Planck para la distribución espectral de energía de la radiación del cuerpo negro}.
A continuación vamos a estudiar, como siempre, el comportamiento de esta expresión en algunos casos límites.




%--------------------------------------
%  Anexos
%--------------------------------------

\part*{Anexos}
\chead[\textsc{Anexos}]{\textsc{Anexos}}

\newpage
\section*{Anexo 1}\label{Anx1}
\renewcommand{\theequation}{A\textsubscript{1}.\arabic{equation}}

En este primer anexo vamos a deducir el volumen de una hiperesfera de radio $R$ de dimensión $n$, de ecuación
\begin{equation}
	x_1^2+x_2^2 + \cdots + x_n^2 = R^n
\end{equation}
En general, sabemos que será proporcional a su radio de la forma
\begin{equation}
	V_n = C_nR^n
\end{equation}
donde $C_n$ es una constante que debemos determinar.
Para ello comenzamos con la definición de la superficie, que no será más que la derivada del volumen
\begin{equation}
	dS_n(R) = C_n n R^{n-1} dR
\end{equation}

\begin{align}
	\int_{-\infty}^{\infty} \dd{x_1} \int_{-\infty}^{\infty} \dd{x_2} \cdots \int_{-\infty}^{\infty} \dd{x_n}e^{-(x_1^2+x_2^2 + \cdots + x_n^2)} &= \int_{0}^{\infty} \dd{S_n} e^{-R^2} \nonumber \\
	\int_{-\infty}^{\infty} \dd{x_1} e^{-x_1^2} \int_{-\infty}^{\infty} \dd{x_2} e^{-x_2^2} \cdots \int_{-\infty}^{\infty} \dd{x_n} e^{-x_n^2} &= \int_{0}^{\infty} \dd{R} C_n n R^{n-1} e^{-R^2} \\
	\left[ 2 \int_{0}^{\infty} \dd{x_1} e^{-x_1^2} \right]^N &=  C_n n \int_{0}^{\infty} \dd{R} R^{n-1} e^{-R^2}\nonumber
\end{align}

Evaluemos ambos miembros por separado.
Para la integral del lado izquierdo realizamos el cambio de variable $x^2=t$
\begin{equation}
	2 \int_{0}^{\infty} \dd{x} e^{-x^2} = 2\frac{1}{2} \int_{0}^{\infty} \dd{t} t^{1/2} e^{-t} = \Gamma\left(\frac{1}{2}\right) = \sqrt{\pi}
\end{equation}
y para la del derecho, $R^2 = t$
\begin{equation}
	C_n n \int_{0}^{\infty} \dd{R} R^{n-1} e^{-R^2} = C_n n \frac{1}{2} \int_{0}^{\infty} \dd{t} t^{n - 1/2} e^{-t} = C_n \frac{n}{2} \Gamma\left(\frac{n}{2}\right)
\end{equation}
y recomponiendo
\begin{equation}
	\pi^{n/2} = C_n \frac{n}{2} \Gamma\left(\frac{n}{2}\right) \Rightarrow C_n = \pi^{n/2}\frac{2}{n}\frac{1}{\Gamma\left(\frac{n}{2}\right)} = \frac{\pi^{n/2}}{\Gamma\left(\frac{n}{2} + 1 \right)}
\end{equation}

Con lo que finalizamos con
\begin{equation}
	V_n = \frac{\pi^{n/2}}{\Gamma\left(\frac{n}{2} + 1 \right)}R^n
\end{equation}

\newpage
\section*{Anexo 2}\label{Anx2}
\renewcommand{\theequation}{A\textsubscript{2}.\arabic{equation}}
\setcounter{equation}{0}

En el anexo 2 vamos a desarrollar la \emph{fórmula de Stirling}, una aproximación válida para números muy grandes que es usada en varias ocasiones en el desarrollo de este documento.

La fórmula de Stirling comienza su desarrollo en la definición del factorial
\begin{equation}
	N! = N\cdot (N-1) \cdots 2 \cdot 1
\end{equation}
y, por tanto,
\begin{equation}
	\ln N! = \ln N + \cdots + \ln 1 = \sum_{n=1}^{N} \ln n
\end{equation}

\begin{wrapfigure}{r}{0.25\textwidth}
	\centering
	\hspace{2.5cm}
	\begin{tikzpicture}
	\begin{axis}[
	scale = 0.65,
	samples=300,
	xmin = 1, xmax = 10,
	ymin = 0,
	axis lines=left,
	xlabel={$n$},
	ylabel={$\ln n$},
	xmajorgrids,
	x grid style={white!70!black},
	ymajorgrids,
	y grid style={white!70!black}]
	
	\addplot[purple, ultra thick, domain=0:10] {ln(x)};
	\addplot[NavyBlue, ultra thick, domain=0:10] {ln(floor(x))};
	
	\addplot [fill=NavyBlue, opacity=0.3, draw=none,domain=0:10] {ln(floor(x))} \closedcycle;
	
	\end{axis}
	\end{tikzpicture}
	\vspace{-0.5cm}
\end{wrapfigure}

Esta suma es igual al área encerrada por la línea poligonal de la figura entre $n = 1$ y $n = N$ (área sombreada).
Consideremos ahora el área encerrada entre los mismos valores de $n$ por la curva $y = \ln n$, también representada en la figura (línea sólida).
Para valores pequeños de $N$ ambas áreas difieren apreciablemente, pero al ir creciendo $N$ las dos curvas tienden a superponerse y las áreas encerradas por cada una de ellas se aproximan entre sí cada vez más.
No olvidemos que deseamos que sea pequeño el error relativo cometido al sustituir un área por otra y no su diferencia absoluta.
Recordando que el área encerrada entre una curva y el eje de abscisas viene dada por la integral definida, podemos escribir para valores grandes de $N$
\begin{equation}
	\ln N! \approx \int_{1}^{N} dn\ln n = \left[ n\ln n - n \right]_1^N
\end{equation}
o, despreciando la unidad frente a N,
\begin{equation}\label{eq:Stirl}
	\boxed{\ln N! \approx N\ln N - N}
\end{equation}
que es la fórmula de Stirling.

Desde luego existen expresiones más exactas, como la que podemos obtener a partir de la función $\Gamma$. Por definición
\begin{equation}
	n! = \int_0^\infty x^n e^{-x}\, dx
\end{equation}
que con el cambio de variable $x = ny$ y agrupando obtenemos
\begin{equation}
	n! = \int_0^\infty e^{n\ln x-x}\, dx = e^{n \ln n} n \int_0^\infty e^{n(\ln y -y)}\, dy
\end{equation}
y aplicando el método de Laplace obtenemos
\begin{equation}
	\int_0^\infty e^{n(\ln y -y)}\, dy \sim  \sqrt{\frac{2\pi}{n}} e^{-n}
\end{equation}
con lo que se obtiene la fórmula de Stirling
$$n! \approx \sqrt{2 \pi n} \; \left(\frac{n}{e}\right)^{n}$$

La expresión obtenida es una mera aproximación usando un orden bajo del método.
La expresión completa recibe el nombre de \emph{serie de Stirling}
\begin{equation}
	n! = \sqrt{2 \pi n} \; \left(\frac{n}{e}\right)^{n} \exp\left[ 
		{\frac{1}{12n}}
		-{\frac{1}{360n^3}}
		+{\frac{1}{1260n^5}}
		-{\frac{1}{1680n^7}}
		+\cdots \right] 
\end{equation}

Desarrollando dicha exponencial también se puede reescribir la fórmula como
\begin{equation}
	n! = \sqrt{2 \pi n} \; \left(\frac{n}{e}\right)^{n}
		\left(
			1
			+{\frac{1}{12n}}
			+{\frac{1}{288n^2}}
			-{\frac{139}{51840n^3n}}
			-{\frac{571}{2488320n^4}}
			+ \cdots
		\right)
\end{equation}
\begin{wrapfigure}{l}{0.4\textwidth}
	\centering
	\hspace{-1.5cm}
	% This file was created by matplotlib2tikz v0.6.15.
\begin{tikzpicture}
\definecolor{color0}{rgb}{0.12156862745098,0.466666666666667,0.705882352941177}
\definecolor{color1}{rgb}{1,0.498039215686275,0.0549019607843137}
\definecolor{color2}{rgb}{0.172549019607843,0.627450980392157,0.172549019607843}

\begin{axis}[
scale = 0.85,
xmin=-0.1, xmax=3.2,
ymin=0, ymax=6.8,
y = 1.25cm,
tick align=outside,
tick pos=left,
xmajorgrids,
ytick={0, 1, 2, 3, 4, 5, 6},
x grid style={white!69.01960784313725!black},
ymajorgrids,
y grid style={white!69.01960784313725!black},
legend cell align={left},
legend entries={{n!},{$\Gamma(n+1)$},{$\sqrt{2\pi n}(n/e)^n$}},
legend style={at={(0.03,0.97)}, anchor=north west, draw=white!80.0!black}
]
\addlegendimage{only marks, mark=*, mark size=3, mark options={solid}, color0}
\addlegendimage{no markers, color1}
\addlegendimage{no markers, color2}
\addplot [ultra thick, color2]
table {%
0 0
0.0311551563539123 0.384937373613077
0.0623103127078247 0.494535561784619
0.093465469061737 0.559262142698111
0.124620625415649 0.602635481943539
0.155775781769562 0.633721422320621
0.186930938123474 0.657055617672748
0.218086094477386 0.675233984886832
0.249241250831299 0.689872113596591
0.280396407185211 0.702035254008984
0.311551563539123 0.712456308427749
0.342706719893036 0.721656393607462
0.373861876246948 0.730015994308697
0.40501703260086 0.737819129829469
0.436172188954773 0.745281898420448
0.467327345308685 0.752571545394972
0.498482501662597 0.759819557422599
0.52963765801651 0.767130867644206
0.560792814370422 0.774590458546594
0.591947970724334 0.782268182259388
0.623103127078247 0.790222334587661
0.654258283432159 0.798502342047535
0.685413439786071 0.807150807607951
0.716568596139984 0.816205086292017
0.747723752493896 0.825698511835393
0.778878908847808 0.835661361501657
0.810034065201721 0.846121622492807
0.841189221555633 0.857105606725176
0.872344377909545 0.868638448839339
0.903499534263458 0.88074451370798
0.93465469061737 0.893447733414097
0.965809846971282 0.906771889023421
0.996965003325195 0.920740849007603
1.02812015967911 0.935378773565599
1.05927531603302 0.950710292111525
1.09043047238693 0.966760659684805
1.12158562874084 0.983555896874775
1.15274078509476 1.00112291695112
1.18389594144867 1.01948964319033
1.21505109780258 1.0386851188399
1.24620625415649 1.05873961173103
1.27736141051041 1.07968471521098
1.30851656686432 1.10155344679789
1.33967172321823 1.12438034574783
1.37082687957214 1.14820157055603
1.40198203592606 1.17305499728042
1.43313719227997 1.19898031947073
1.46429234863388 1.22601915040361
1.49544750498779 1.25421512825991
1.5266026613417 1.2836140248313
1.55775781769562 1.31426385830583
1.58891297404953 1.34621501065564
1.62006813040344 1.3795203501317
1.65122328675735 1.41423535935907
1.68237844311127 1.45041826952166
1.71353359946518 1.48813020112585
1.74468875581909 1.52743531183706
1.775843912173 1.56840095189337
1.80699906852692 1.61109782761274
1.83815422488083 1.65560017352739
1.86930938123474 1.70198593369835
1.90046453758865 1.75033695278654
1.93161969394256 1.80073917748209
1.96277485029648 1.85328286892278
1.99393000665039 1.90806282676392
2.0250851630043 1.96517862559661
2.05624031935821 2.02473486444871
2.08739547571213 2.08684143014333
2.11855063206604 2.1516137753331
2.14970578841995 2.21917321207523
2.18086094477386 2.28964722186234
2.21201610112778 2.36316978307736
2.24317125748169 2.43988171689816
2.2743264138356 2.51993105273819
2.30548157018951 2.60347341437442
2.33663672654342 2.6906724279832
2.36779188289734 2.7817001533781
2.39894703925125 2.87673753982238
2.43010219560516 2.97597490787237
2.46125735195907 3.07961245879659
2.49241250831299 3.18786081321044
2.5235676646669 3.30094158066666
2.55472282102081 3.41908796204903
2.58587797737472 3.54254538673074
2.61703313372864 3.67157218658005
2.64818829008255 3.80644030902516
2.67934344643646 3.94743607152728
2.71049860279037 4.09486095995737
2.74165375914429 4.24903247352732
2.7728089154982 4.41028501909212
2.80396407185211 4.57897085781585
2.83511922820602 4.75546110738176
2.86627438455993 4.94014680312689
2.89742954091385 5.13344002169385
2.92858469726776 5.33577507101944
2.95973985362167 5.54760975072049
2.99089500997558 5.76942668719472
3.0220501663295 6.00173474802735
3.05320532268341 6.24507054058641
3.08436047903732 6.49999999999955
};
\addplot [ultra thick, color1]
table {%
-0.2 1.1642297137253
-0.167037649753208 1.12916026205829
-0.134075299506415 1.09772343912734
-0.101112949259623 1.06952821057738
-0.0681505990128309 1.044241236184
-0.0351882487660386 1.02157696969694
-0.00222589851924637 1.00128973392686
0.0307364517275459 0.983167327956818
0.0636988019743382 0.967025833361151
0.0966611522211305 0.952705366539138
0.129623502467923 0.94006658341285
0.162585852714715 0.928987786775995
0.195548202961507 0.919362519680909
0.2285105532083 0.911097553349791
0.261472903455092 0.904111197285956
0.294435253701884 0.898331874047231
0.327397603948676 0.89369691262089
0.360359954195469 0.89015152331036
0.393322304442261 0.887647924101847
0.426284654689053 0.886144594066693
0.459247004935846 0.885605633805239
0.492209355182638 0.886000216501927
0.52517170542943 0.887302116031165
0.558134055676222 0.889489300876218
0.591096405923015 0.892543584512719
0.624058756169807 0.896450324452455
0.657021106416599 0.901198163410824
0.689983456663391 0.906778807106792
0.722945806910184 0.913186834070101
0.755908157156976 0.920419533550674
0.788870507403768 0.928476768226662
0.821832857650561 0.937360858912055
0.854795207897353 0.94707648888949
0.887757558144145 0.957630625853103
0.920719908390937 0.969032459751106
0.95368225863773 0.981293355077833
0.986644608884522 0.994426816387561
1.01960695913131 1.00844846599398
1.05256930937811 1.02337603298472
1.0855316596249 1.03922935282415
1.11849400987169 1.05603037694334
1.15145636011848 1.07380319182641
1.18441871036528 1.09257404720009
1.21738106061207 1.11237139302019
1.25034341085886 1.13322592502647
1.28330576110565 1.155170638708
1.31626811135244 1.17824089158486
1.34923046159924 1.20247447377132
1.38219281184603 1.22791168684013
1.41515516209282 1.25459543105885
1.44811751233961 1.28257130111765
1.48107986258641 1.31188769051425
1.5140422128332 1.34259590480654
1.54700456307999 1.37475028398673
1.57996691332678 1.40840833427413
1.61292926357358 1.44363086966559
1.64589161382037 1.480482163626
1.67885396406716 1.51903011134327
1.71181631431395 1.55934640301668
1.74477866456074 1.60150670869122
1.77774101480754 1.64559087519702
1.81070336505433 1.69168313579988
1.84366571530112 1.73987233321795
1.87662806554791 1.79025215671103
1.90959041579471 1.84292139400216
1.9425527660415 1.89798419884707
1.97551511628829 1.95555037512618
2.00847746653508 2.01573567839561
2.04143981678187 2.07866213589887
2.07440216702867 2.14445838611035
2.10736451727546 2.21326003895441
2.14032686752225 2.2852100579215
2.17328921776904 2.36045916538484
2.20625156801584 2.43916627250808
2.23921391826263 2.52149893522732
2.27217626850942 2.60763383788883
2.30513861875621 2.69775730622873
2.33810096900301 2.79206585149207
2.3710633192498 2.89076674760729
2.40402566949659 2.99407864345838
2.43698801974338 3.10223221243124
2.46995036999017 3.2154708415545
2.50291272023697 3.33405136270719
2.53587507048376 3.45824482852927
2.56883742073055 3.58833733584405
2.60179977097734 3.72463089958734
2.63476212122414 3.86744438043565
2.66772447147093 4.01711446953689
2.70068682171772 4.17399673397207
2.73364917196451 4.33846672681724
2.7666115222113 4.51092116593128
2.7995738724581 4.69177918586955
2.83253622270489 4.8814836676161
2.86549857295168 5.08050265113983
2.89846092319847 5.28933083611442
2.93142327344527 5.50849117649825
2.96438562369206 5.73853657505278
2.99734797393885 5.98005168428501
3.03031032418564 6.23365482073555
3.06327267443243 6.49999999999999
};
\addplot [ultra thick, color0, mark=*, mark size=3, mark options={solid}, only marks]
table {%
0 1
1 1
2 2
3 6
};
\end{axis}

\end{tikzpicture}
	\vspace{-1cm}
\end{wrapfigure}

Pero para valores de $n$ muy grandes podemos despreciar los valores del desarrollo y quedarnos con la expresión aproximada.
Tomando logaritmos y de nuevo aproximando llegamos a \eqref{eq:Stirl}
\begin{equation}
	\ln n! \approx n\ln n - n
\end{equation}

La diferencia entre la aproximación y el valor real disminuye cuando $n$ toma valores grandes, como se puede apreciar en la figura que se adjunta.
En los sistemas en los que vamos a estudiar, que tendrán moléculas en el orden del mol ($10^{23}$), resultará muy útil.

\newpage
\section*{Anexo 3}\label{Anx3}
\renewcommand{\theequation}{A\textsubscript{3}.\arabic{equation}}
\setcounter{equation}{0}


\newpage
\section*{Anexo 4}\label{Anx4}
\renewcommand{\theequation}{A\textsubscript{4}.\arabic{equation}}
\setcounter{equation}{0}

En este cuarto anexo vamos a calcular la integral
\begin{equation}
	\int_{0}^{\infty} \dd{\varepsilon} \frac{f(\varepsilon)}{e^{\beta(\varepsilon_r - \mu)} + 1}
\end{equation}

Comenzamos con el cambio de variable $z = \beta(\varepsilon_r - \mu)$, con lo que obtenemos
\begin{align}\label{eq:des1}
	\int_{0}^{\infty} \dd{\varepsilon} \frac{f(\varepsilon)}{e^{\beta(\varepsilon_r - \mu)} + 1} &= \int_{-\beta\mu}^{\infty} \frac{dz}{\beta} \frac{f(\mu + \frac{z}{\beta})}{e^z + 1} = \frac{1}{\beta} \int_{-\beta\mu}^{0} \dd{z} \frac{f(\mu + \frac{z}{\beta})}{e^z + 1} + \frac{1}{\beta} \int_{0}^{\infty} \dd{\beta} \frac{f(\mu + \frac{z}{\beta})}{e^z + 1} \nonumber\\
	&= \frac{1}{\beta} \int_{0}^{\beta\mu} \dd{\beta} \frac{f(\mu - \frac{z}{\beta})}{e^{-z} + 1} + \frac{1}{\beta} \int_{0}^{\infty} \dd{\beta} \frac{f(\mu + \frac{z}{\beta})}{e^z + 1}
\end{align}

A continuación evaluamos la fracción que se encuentra en la primera integral resultante
\begin{equation}
	\frac{1}{e^{-z} + 1} = \frac{e^{-z} + 1 - e^{-z}}{e^{-z} + 1} = 1 - \frac{-e^{-z}}{e^{-z} + 1} = 1 - \frac{1}{e^{z} + 1}
\end{equation}
y retomamos con esta relación el desarrollo
\begin{align}\label{eq:des2}
	&\rightsquigarrow \frac{1}{\beta} \left[  \int_{0}^{\beta\mu} \dd{z} f\left( \mu - \frac{z}{\beta}\right) \left( 1 - \frac{1}{e^{z} + 1} \right) + \int_{0}^{\infty} \dd{z} \frac{f(\mu + \frac{z}{\beta})}{e^z + 1}\right] \nonumber \\
	&= \frac{1}{\beta} \left[ - \int_{\mu}^{0} \beta \dd{\varepsilon} f(\varepsilon) +  \int_{0}^{\beta\mu} \dd{z} \frac{f\left( \mu - \frac{z}{\beta}\right)}{e^{z} + 1} + \int_{0}^{\infty} \dd{z} \frac{f(\mu + \frac{z}{\beta})}{e^z + 1}\right]\\
	&= \int_{0}^{\mu} \dd{\varepsilon} f(\varepsilon) + \frac{1}{\beta} \left[ -\int_{0}^{\infty} \dd{z} \frac{f\left( \mu - \frac{z}{\beta}\right)}{e^z + 1} + \int_{0}^{\infty} \dd{z} \frac{f\left( \mu + \frac{z}{\beta}\right)}{e^z + 1} \right] \nonumber
\end{align}
donde, en el último paso, hemos aproximado en los límites de las integrales como $1 \ll \beta\mu \approx \infty$.

Ahora desarrollamos las funciones $f\left( \mu - \frac{z}{\beta}\right)$ y $f\left( \mu + \frac{z}{\beta}\right)$ en $z=0$
\begin{align}
	f\left( \mu - \frac{z}{\beta}\right) &= f(\mu) + f'\left( \mu - \frac{z}{\beta}\right)_{z=0}\frac{z}{\beta} + \frac{1}{2!}f''\left( \mu - \frac{z}{\beta}\right)_{z=0}\frac{z^2}{\beta^2} + \cdots \nonumber \\
	&= f(\mu) + f'(\mu)\frac{z}{\beta} + \frac{1}{2!}f''(\mu)\frac{z^2}{\beta^2} + \cdots\\
	f\left( \mu + \frac{z}{\beta}\right) &= f(\mu) - f'(\mu)\frac{z}{\beta} + \frac{1}{2!}f''(\mu)\frac{z^2}{\beta^2} - \cdots
\end{align}
y restando ambas
\begin{equation}
	 f\left( \mu + \frac{z}{\beta}\right) - f\left( \mu - \frac{z}{\beta}\right) = 2f'(\mu)\frac{z}{\beta} + \frac{2}{3!}f'''(\mu)\frac{z^3}{\beta^3} + \cdots
\end{equation}

Así, podemos introducirla en el desarrollo en \eqref{eq:des2}, obteniendo
\begin{equation}\label{eq:des3}
	\rightsquigarrow \int_{0}^{\mu} \dd{\varepsilon} f(\varepsilon) + 2\frac{1}{\beta}f'(\mu) \int_{0}^{\infty} dz \frac{z}{e^z + 1} + \frac{2}{3!}\frac{1}{\beta^3}f'''(\mu) \int_{0}^{\infty} dz \frac{z^3}{e^z + 1} + \cdots
\end{equation}

Solo nos queda evaluar cada una de las integrales de la serie restante.
Podemos observar que la única diferencia entre ellas será la potencia del denominador, así que calcularemos un equivalente genérico
\begin{align}\label{eq:des_sub_1}
	I_k &= \int_{0}^{\infty} \dd{z} \frac{z^{2k+1}}{e^z + 1} = \int_{0}^{\infty} \dd{z} z^{2k+1} e^{-z} \frac{e^z}{e^z + 1} = \int_{0}^{\infty} \dd{z} z^{2k+1} e^{-z} \frac{1}{1 + e^{-z}} \nonumber \\
		&= \int_{0}^{\infty} \dd{z} z^{2k+1} e^{-z} \left[1 - e^{-z} + e^{z} - \cdots \right] = \int_{0}^{\infty} \dd{z} z^{2k+1} \left[ e^{-z} - e^{-2z} + e^{3z} - \cdots \right] \nonumber\\
		&= \int_{0}^{\infty} \dd{z} z^{2k+1} \sum_{n=1}^{\infty} (-1)^{n+1}e^{-nz} = \sum_{n=1}^{\infty} (-1)^{n+1}  \int_{0}^{\infty} \dd{z} z^{2k+1} e^{-nz} \\
		\textcolor{NavyBlue}{[t = nz]} &= \sum_{n=1}^{\infty} (-1)^{n+1} \frac{1}{n^{2k}} \int_{0}^{\infty} \dd{t} t^{2k+1} e^{-t} = \Gamma(2k) \sum_{n=1}^{\infty} (-1)^{n+1} \frac{1}{n^{2k}} \nonumber
\end{align}

Una vez aquí debemos introducir la función $\zeta$ de Riemann, definida como
\begin{equation}\label{eq:zeta}
	\zeta(s) = \sum_{n=1}^{\infty} \frac{1}{n^{s}}
\end{equation}
y que podemos manipular de la siguiente forma
\begin{align}
	\zeta(2k) &= \sum_{n=1}^{\infty} \frac{1}{n^{2k}} = \frac{1}{1^{2k}} + \frac{1}{2^{2k}} + \frac{1}{3^{2k}} + \cdots = \frac{1}{1^{2k}} + \frac{1}{3^{2k}} + \cdots + \frac{1}{2^{2k}} + \frac{1}{4^{2k}} + \cdots \nonumber \\
		&= \frac{1}{1^{2k}} + \frac{1}{3^{2k}} + \cdots + \frac{1}{2^{2k}}\left[\frac{1}{1^{2k}}  + \frac{1}{2^{2k}} + \cdots \right] =  \frac{1}{1^{2k}} + \frac{1}{3^{2k}} + \cdots + \frac{1}{2^{2k}}\zeta(2k)
\end{align}
y así
\begin{equation}
	\frac{1}{1^{2k}} + \frac{1}{3^{2k}} + \cdots = \left[1 - \frac{1}{2^{2k}} \right] \zeta(2k)
\end{equation}
que podemos introducir en \eqref{eq:des_sub_1} si desarrollamos la serie y reordenamos de tal forma que
\begin{align}\label{eq:des_sub_2}
	I_k &= \Gamma(2k) \sum_{n=1}^{\infty} (-1)^{n+1} \frac{1}{n^{2k}} = \Gamma(2k) \left[ \frac{1}{1^{2k}} + \frac{1}{3^{2k}} + \cdots - \frac{1}{2^{2k}} - \frac{1}{4^{2k}} - \cdots \right]  \nonumber \\
		&= \Gamma(2k) \left[ \left(1 - \frac{1}{2^{2k}} \right) \zeta(2k) - \frac{1}{2^{2k}}\left(\frac{1}{1^{2k}}  + \frac{1}{2^{2k}} + \cdots \right) \right] \\
		&= \Gamma(2k) \left[ \left(1 - \frac{1}{2^{2k}} \right) \zeta(2k) - \frac{1}{2^{2k}}\zeta(2k) \right] = \Gamma(2k)\left[ 1-2^{1-2k} \right] \zeta(2k)\nonumber
\end{align}

Ahora que tenemos el valor de cada una de estas \emph{pequeñas} integrales podemos, finalmente, recuperar el desarrollo de \eqref{eq:des3}
\begin{align}\label{eq:des4}
	I &= \int_{0}^{\mu} \dd{\varepsilon} f(\varepsilon) + \frac{2}{\beta^2}f'(\mu) \Gamma(2)(1-2^{-1})\zeta(2) + \frac{1}{3\beta^4}f'''(\mu) \Gamma(4)(1-2^{-3})\zeta(4) + \cdots  \nonumber \\
		&= \int_{0}^{\mu} \dd{\varepsilon} f(\varepsilon) + \frac{1}{\beta^2}f'(\mu) \frac{\pi^2}{6} + \frac{1}{3\beta^4}f'''(\mu) \cdot 6 \frac{7}{8}\frac{\pi^4}{90} + \cdots
\end{align}
que es el resultado que buscábamos.

\newpage
\section*{Anexo 5}\label{Anx5}
\renewcommand{\theequation}{A\textsubscript{5}.\arabic{equation}}
\setcounter{equation}{0}

Al igual que en el Anexo anterior, vamos a resolver un tipo de integral. En este caso será del tipo
\begin{equation}
	I(2k) = \int_{0}^{\infty} \dd{z} \frac{z^{2k-1}}{e^z-1}
\end{equation}

Así:
\begin{align}
	I(2k) &= \int_{0}^{\infty} \dd{z} \frac{z^{2k-1}}{e^z-1} = \int_{0}^{\infty} \dd{z} z^{2k-1} e^{-z}\frac{e^z}{e^z-1} \nonumber \\
		&= \int_{0}^{\infty} \dd{z} z^{2k-1} e^{-z}\frac{1}{1 - e^{-z}} = \int_{0}^{\infty} \dd{z} z^{2k-1} e^{-z} \left[1 + e^{-z} + e^{-2z} + \cdots \right] \\
		&= \int_{0}^{\infty} \dd{z} z^{2k-1} \sum_{n=1}^{\infty} e^{-nz} = \sum_{n=1}^{\infty}  \int_{0}^{\infty} \dd{z} z^{2k-1} e^{-nz}  \nonumber
\end{align}
donde hemos usado las mismas propiedades que en \eqref{eq:des_sub_1}

Una vez aquí realizamos el cambio de variable $t=nz$
\begin{equation}
	I(2k) = \sum_{n=1}^{\infty} \frac{1}{n^{2k}} \int_{0}^{\infty} \dd{t} t^{2k-1} e^{-t} = \sum_{n=1}^{\infty} \frac{\Gamma(2k)}{n^{2k}}
\end{equation}
y finalizamos con la ecuación $\zeta$ de Riemann \eqref{eq:zeta}
\begin{equation}
	\int_{0}^{\infty} \dd{z} \frac{z^{2k-1}}{e^z-1} = \Gamma(2k)\zeta(2k)
\end{equation}





\end{document}

