En este capítulo vamos a analizar dos problemas aparentemente dispares, el estudio estadístico de la radiación electromagnética y las propiedades termodinámicas de los sólidos, pero que, desde el punto de vista físico y de la manera de abordarlos, tienen mucho en común:
\begin{enumerate}
	\item En ambos casos, la base del problema físico es un fenómeno ondulatorio: oscilaciones del campo electromagnético, en el caso de la radiación, y vibraciones alrededor de sus posiciones de equilibrio de los átomos que conforman el cristal.

	\item La descripción cuántica de los dos fenómenos se puede hacer en términos del sencillo oscilador armónico y de sus niveles de energía discretos. Desde un punto de vista práctico, la suma sobre los estados discretos de energía puede aproximarse por una integral sobre una distribución continua de frecuencias y, en las dos situaciones la densidad de estados es, esencialmente, la misma.
	
	\item Desde un punto de vista cuántico fundamental, la descripción del estado del sistema se puede hacer mediante el concepto de «cuasi-partícula»: el fotón en el caso de la radiación y el fonón en el de los sólidos, bosones en los dos casos. El sistema se puede considerar como un gas ideal de fotones o de fonones, según el caso, que obedecen a la estadística de Base-Einstein con potencial químico nulo.

	\item Desde un punto de vista histórico, la solución correcta a los dos problemas se enmarca en los comienzos de la moderna Teoría Cuántica, cuando las ideas de cuantificación de la energía fueron introducidas, a comienzos del siglo XX. La comprobación experimental de las predicciones teóricas supuso un importante impulso para la aceptación de las nuevas ideas.
\end{enumerate}

\newpage
\section{Radiación electromagnética}

Como es sabido, se conoce con el nombre genérico de radiación a una de las formas posibles de propagación de la energía, una de cuyas características esenciales es que no necesita la presencia de un medio material.

La teoría clásica considera a la radiación como un fenómeno ondulatorio regido por las ecuaciones de Maxwell del Electromagnetismo, es decir, como un conjunto de ondas planas electromagnéticas.
Es importante señalar que estas ondas electromagnéticas son transversales, es decir, que la perturbación se produce en un plano perpendicular a la dirección de propagación, punto sobre el que más adelante volveremos.
Esta teoría se basa en la observación de que las ondas electromagnéticas se propagan a la velocidad de la luz, que es un ejemplo típico de radiación, y en que, además, no necesitan medio material para propagarse, al contrario de lo que sucede en las ondas elásticas.
Mediante esta teoría se puede justificar la existencia y propiedades de fenómenos bien conocidos, como la interferencia y la difracción.
Los distintos procesos de radiación, tales como la propagación de la luz, emisión de rayos X o la transmisión de señales de radio, aparecen entonces como casos concretos del fenómeno general de la radiación, en el sentido de que corresponden a valores distintos de la frecuencia, pero teniendo en común el que se propagan con la velocidad de la luz.
Dicho de otro modo, cada proceso de radiación puede caracterizarse simplemente por su longitud de onda $\lambda$, o lo que es equivalente, por su frecuencia $\nu$
\begin{equation}\label{eq:freq_t8}
	\nu = \frac{c}{\lambda}
\end{equation}
o su frecuencia angular $\omega$
\begin{equation}
	\omega = 2\pi\nu
\end{equation}

Las longitudes de onda asociadas con los fenómenos de radiación que se presentan en la naturaleza, se distribuyen en un amplísimo margen que va desde valores del orden de $\lambda = 10^{-14}$m e inferiores, para los rayos gamma, hasta $\lambda = 10^{-6}$m y superiores correspondientes ondas de radiofrecuencia.
La denominada radiación térmica, que puede ser detectada por nuestros sentidos, corresponde a los valores de $\lambda$ comprendidos entre $10^{-7}$ y $10^{-4}$m, e incluye una parte de la radiación ultravioleta y de la infrarroja.

El primer intento de explicar las propiedades de la radiación a partir de un modelo estadístico se debió a Rayleigh y Jeans, quienes consideraron a la radiación contenida en un recinto en equilibrio a la temperatura $T$ como una superposición de ondas planas electromagnéticas y utilizaron las técnicas de la Mecánica Estadística Clásica.
De este modo, obtuvieron para la densidad espectral de energía, o sea para la energía media asociada con las ondas electromagnéticas cuya frecuencia angular estuviese comprendida entre $\omega$ y $\omega + d\omega$, la expresión
\begin{equation}\label{eq:E_med_t8}
	\expval{E}(\omega) \dd{\omega}= V \frac{k_B T}{\pi^2 c^3} \omega^2 \dd{\omega}
\end{equation}
donde $V$ es el volumen del recinto considerado. Para valores pequeños de $\omega$ esta expresión presenta un comportamiento acorde con los datos experimentales, pero a altas frecuencias el valor de $\expval{E}(\omega)$ determinado experimentalmente presenta un descenso exponencial con $\omega$, mientras que la expresión \eqref{eq:E_med_t8} predice un aumento proporcional a $\omega^2$ para todos los valores de $\omega$.
De hecho, es fácil prever que la expresión \eqref{eq:E_med_t8} no puede se correcta para valores grandes de la frecuencia angular $\omega$. En efecto, de acuerdo con ella, resulta que la energía total de radiación contenida en un recinto vendrá dada por
\begin{equation}\label{eq:E_tot_t8}
	\expval{E}_{total} =\int_{0}^{\infty} \dd{\omega} \expval{E}(\omega) \propto \eval{\omega^3}_{0}^{\infty} \rightarrow \infty
\end{equation}
es decir, tiende a infinito. Esto es una consecuencia directa del hecho de que
\begin{equation}
	\lim\limits_{\omega\rightarrow\infty} \expval{E}(\omega) \neq 0
\end{equation}

Este resultado anómalo, y durante mucho tiempo inexplicable, se conoce con el nombre de \emph{catástrofe ultravioleta}, por tener su origen en el comportamiento de \eqref{eq:E_med_t8} para valores grandes de $\omega$, o sea, para valores pequeños de $\lambda$, que es la zona del espectro donde se sitúa la radiación ultravioleta.
Hoy día se sabe que las causas de la anomalía están en la incapacidad intrínseca de la Mecánica Clásica para dar una explicación teórica correcta del fenómeno de radiación.
Más aún, los orígenes históricos de la Mecánica Cuántica pueden situarse en este problema concreto.
En efecto, fue precisamente en la solución empírica a este problema dada por Planck, cuando se introdujo por primera vez, en Física una cuantificación de la energía. 
La expresión obtenida por Planck no presentaba la catástrofe ultravioleta, y además concordaba con los resultados experimentales. Este éxito llevó a intentar un análisis semejante de otras paradojas existentes en Mecánica Clásica, lo que condujo al desarrollo de ia Mecánica Cuántica.

Pasemos ahora directamente a un tratamiento cuántico del problema.
Para ello necesitamos utilizar algunos conceptos fundamentales y bien conocidos de Mecánica Cuántica.

Sabemos que la teoría cuántica asigna a la materia, de acuerdo con los resultados experimentales, un doble carácter, en virtud del cual presenta tanto propiedades corpusculares o de partículas, como propiedades ondulatorias.
Es la denominada \emph{dualidad onda-corpúsculo}.
Así, por ejemplo, la radiación presenta propiedades, como la interferencia o la difracción, que son típicamente ondulatorias, y propiedades que exigen una interpretación corpuscular, corno el efecto fotoeléctrico y el efecto Compton.
El paso de una. descripción a otra, es decir, de ondas a partículas y viceversa, se efectúa mediante las relaciones de Einstein y de Broglie
\begin{align}
	\varepsilon = h\nu = \hbar\omega \label{eq:eps_t8}\\
	\vb{p} = \frac{h\bm{\kappa}}{2\pi} = \hbar\bm{\kappa} \label{eq:p_t8}
\end{align}

Aquí $\varepsilon$ y $\mathbf{p}$ son la energía y el momento o cantidad de movimiento de las partículas asociadas, mientras que $\nu$ y $\bm{\kappa}$, representan, respectivamente, la frecuencia y el vector de onda de la onda electromagnética, Este último se define de modo que su dirección coincida con la de propagación de la onda, mientras que su módulo viene dado por
\begin{equation}\label{eq:valor_kappa_t8}
	\abs{\bm{\kappa}} = \frac{2\pi}{\lambda} = \frac{\omega}{c}
\end{equation}

Las partículas asociadas a la radiación electromagnética se denominan \emph{fotones} y es fácil ver que, por poseer una velocidad igual a la de la luz, han de poseer una masa en reposo nula.
En efecto, a partir de la relación relativista
\begin{equation}
	m = \frac{m_0}{\sqrt{1 - \left(\frac{v}{c}\right)^2}}
\end{equation}
donde $v$ es la velocidad de la partícula, $m_0$ su masa en reposo y $m$ la masa relativista, se deduce que, si $m_0$ fuese distinta de cero, al ser $v = c$ tendería $m$ a infinito.
A partir de \eqref{eq:p_t8}, \eqref{eq:valor_kappa_t8}, \eqref{eq:freq_t8} y \eqref{eq:eps_t8} se tiene que
\begin{equation}
	\abs{\vb{p}} = \hbar \abs{\bm{\kappa}} = \frac{2\pi \hbar}{\lambda} = \frac{h\nu}{c} = \frac{\varepsilon}{c}
\end{equation}
y, por tanto,
\begin{equation}
	m = \frac{\abs{\vb{p}}}{c} = \frac{h\nu}{c^2} = \frac{\varepsilon}{c^2}
\end{equation}

que nos indica qué en general $m$ no es ni nula ni infinita.

Así pues, mientras que en la descripción ondulatoria se considera el campo electromagnético como una superposición de ondas planas, en la descripción corpuscular se le considera compuesto por un conjunto de fotones.
Como ya dijimos al principio, las ondas planas electromagnéticas son transversales.
Además, la perturbación que se propaga consiste en oscilaciones de los campos eléctrico \textbf{E} y magnético \textbf{H}, los cuales a su vez son perpendiculares entre sí en cada punto de las onda.

Restringiéndonos al campo eléctrico ---el campo magnético en cada punto es función del campo eléctrico en ese mismo punto--- una onda plana tiene la forma\footnote{Como siempre, únicamente tiene significado físico la parte real o la parte imaginaria por separado.}
\begin{equation}
	\vb{E}(\vb{r}, t) = \vb{E}_0 e^{i(\bm{\kappa}\cdot\vb{r}-\omega t)}
\end{equation}

La especificación de una onda plana exige, en principio, la determinación de $\vb{E}_0$, $\bm{\kappa}$ y $\omega$.
Ahora bien, una vez dado $\bm{\kappa}$, resulta que, por \eqref{eq:valor_kappa_t8}, también está dada su frecuencia,\footnote{ Esta no es una propiedad general de las ondas, sino solo de aquellas en las que la velocidad de todas las ondas planas es la misma ---onda no dispersiva---, como sucede con las ondas electromagnéticas.} luego sólo son independientes $\vb{E}_0$ Y $\bm{\kappa}$.
Por otro lado, si utilizamos el hecho de que las ondas son transversales, o sea que $\vb{E}_0$ es perpendicular a $\bm{\kappa}$ podemos tomar dos vectores unitarios $\vb{e}_1$ y $\vb{e}_2$ en el plano perpendicular a $\bm{\kappa}$, y escribir
\begin{equation}\label{eq:E_1_t8}
	\vb{E}(\vb{r}, t) = E_{01} \vb{e}_1 e^{i(\bm{\kappa}\cdot\vb{r}-\omega t)} + E_{02} \vb{e}_2 e^{i(\bm{\kappa}\cdot\vb{r}-\omega t)}
\end{equation}
con lo que resulta que, una vez fija la dirección de $\bm{\kappa}$ únicamente son posibles dos direcciones independientes de polarización de \textbf{E}. Cualquier otra dirección es combinación de estas dos.

Cuando se pasa a una descripción corpuscular, a cada onda plana independiente se le asocia un estado de fotón independiente, de acuerdo con las relaciones \eqref{eq:eps_t8} y \eqref{eq:p_t8}.
La amplitud de la onda mide el número de fotones que se encuentra en el estado asociado.
Según todo esto, ¿qué será necesario especificar para determinar cada uno de los estados independientes de un fotón?
Pues su cantidad de movimiento \textbf{p} ---que se obtiene a partir del vector de onda $\bm{\kappa}$--- y su estado de polarización, teniendo en cuenta que, una vez fijado \textbf{p} o $\bm{\kappa}$, sólo son posibles dos orientaciones independientes.
Es decir, que a cada valor de \textbf{p} corresponde dos estados independientes de un fotón con las dos polarizaciones posibles.

Veamos qué propiedades deben tener los fotones:
\begin{enumerate}
	\item Como las ondas planas son indistinguibles, los fotones han de ser partículas cuánticamente idénticas, diferenciándose únicamente por el estado en que se encuentran.
	
	\item Dado que las amplitudes con que aparecen las ondas planas en una superposición son totalmente arbitrarias, deducimos que el número de fotones existentes en un estado dado puede ser tan grande como queramos, y en consecuencia, los fotones obedecerán a la estadística de Bose-Einstein, es decir, serán \emph{bosones}.
	
	\item Una característica de las ondas electromagnéticas es que son absorbidas y emitidas por la materia.
	Los fotones aparecerán y desaparecerán cuando la radiación sea, respectivamente, emitida y absorbida por la materia.
	En consecuencia, su número, incluso en un sistema cerrado, no será constante.
	
	\item Veamos ahora si los fotones se comportan o no como un gas ideal, es decir, si existen interacciones entre ellos.
	La respuesta está en el principio de superposición o, lo que es equivalente, en el carácter lineal de la ecuación que gobierna el comportamiento de las ondas ---ecuación de onda---, y puede verse claramente mediante un sencillo razonamiento.
	En efecto, sabemos que cuando dos ondas se encuentran en una cierta región del espacio, no se distorsionan, sino que actúan aditivamente, de manera que, por ejemplo, el campo eléctrico resultante en cada punto es la suma de los campos 	eléctricos asociados a cada una de las ondas en ese punto.
	Concluimos entonces que los fotones no interaccionan entre sí, aunque, como inmediatamente veremos, sí que lo hacen con otros cuerpos extraños que encuentren en su camino.
\end{enumerate}

En resumen, se obtiene la conclusión de que \emph{un gas de fotones constituye un gas ideal de Bose.}
Este resultado puede entenderse como un postulado basado en las propiedades del campo electromagnéticos, es decir, en las ecuaciones de Maxwell.

\section{Distribución de Plank: Ley de Rayleigh-Jeans y ley de Wien}

La radiación electromagnética que existe en equilibrio dentro de un recinto de volumen $V$, cuyas paredes se mantienen a la temperatura $T$, puede tratarse como un sistema termodinámico en equilibrio, al que le son aplicables las leyes de la Termodinámica y de la Mecánica Estadística.
De hecho, hemos visto en el apartado anterior que la radiación puede considerarse corno un gas de fotones.
Hay que señalar, sin embargo, que existen profundas diferencias entre un gas molecular y un gas de fotones.
En un gas molecular sabemos que la distribución de velocidades que caracteriza el equilibrio es la de Maxwell-Boltzmann, mientras que en un gas de fotones la situación será evidentemente distinta, ya que todos los fotones se mueven con la velocidad de la luz, independientemente de las condiciones en que se encuentre el gas.
Veremos que lo que caracteriza el equilibrio en el caso de los fotones es la distribución existente entre las frecuencias $\omega$.
Entonces, si se consideran, por ejemplo, dos estados de equilibrio de un mismo recinto a dos temperaturas diferentes, lo que se modifica al pasar de uno a otro es la distribución de frecuencias, es decir el número de fotones existentes en cada intervalo de frecuencias.

Otra importante diferencia de principio surge si se considera el mecanismo que permite alcanzar y mantener la situación de equilibrio.
En el caso de los gases moleculares, dicho mecanismo lo constituyen las colisiones entre las moléculas.
A este respecto conviene recordar que cuando hablábamos de un gas ideal, nos referíamos a un sistema de partículas cuyas energías de interacción eran cuantitativamente despreciables.

El estudio detallado de la interacción de la radiación con la materia, es decir, la absorción y emisión de fotones, constituye un problema en el que intervienen de modo fundamental las propiedades de los átomos que constituyen el cuerpo emisor.
Sin embargo, aquí lo único que vamos a necesitar es admitir que la materia presente, constituida por las paredes del sistema o por cualquier otro cuerpo, emite y absorbe fotones, de manera que se establece una situación de equilibrio, la cual se mantiene mientras no se alteren las condiciones externas del sistema.
Un gas de fotones en equilibrio en el interior de un recinto se denomina \emph{radiación negra} o \emph{radiación del cuerpo negro}

El hecho de que el número de fotones que existen en el sistema varíe en virtud de un proceso que tiene lugar en el interior del mismo, y no como consecuencia de un intercambio con otro sistema, hace que la distribución estadística de los fotones sea un caso muy especial de la distribución de Bose.
En efecto, hemos admitido que se establece una situación de equilibrio, la cual depende de las condiciones externas en que se encuentra el sistema.
Esto quiere decir que el número de partículas que existen en el sistema, cuando está en equilibrio, no es una variable independiente, sino que será una función de las condiciones externas.
En el colectivo canónico estas condiciones externas son la temperatura y el volumen, de manera que resulta que si ambas se mantienen constantes ha de cumplirse en el equilibrio que
\begin{equation}\label{eq:dF_t8}
	(dF)_{T,V} = 0
\end{equation}
ya que en general se tiene $dF = -SdT - pdV$.

Consideremos ahora una situación general en la que modificaremos el número de partículas, manteniendo constantes la temperatura y el volumen.
Entonces,
\begin{equation}\label{eq:dF_2_t8}
	(dF)_{T,V} = \left( \pdv{F}{N} \right)_{T,V} dN
\end{equation}

Comparando \eqref{eq:dF_t8} y \eqref{eq:dF_2_t8}, resulta
\begin{equation}
	\left( \pdv{F}{N} \right)_{T,V, \expval{N}} = 0
\end{equation}
donde, al indicar que la derivada se calcula a $N = \expval{N}$, estamos explicitando que se trata de un estado de equilibrio.

Dado que $\left( \pdv{F}{N} \right)_{T,V, \expval{N}}$ es el potencial químico $\mu$ resulta que, para un gas de fotones, se tiene que
\begin{equation}
	\mu = 0
\end{equation}

Esta es la característica especial a que nos hemos referido antes.
Si tenemos en cuenta que la energía de un fotón en el estado $r$ es, de acuerdo con \eqref{eq:eps_t8}
\begin{equation}
	\varepsilon_r = h\nu_r = \hbar\omega_r
\end{equation}
obtenemos, a partir de \eqref{eq:Q_BE}
\begin{equation}\label{eq:Q_Fot}
	\ln Q_F = -\sum_r \ln (1 - e^{-\beta\hbar\omega_r})
\end{equation}
y a partir de \eqref{eq:n_r_BE}
\begin{equation}\label{eq:n_r_Fot}
	\expval{n_r} = \frac{1}{e^{-\beta\hbar\omega_r} - 1}
\end{equation}
La ecuación \eqref{eq:n_r_Fot} se conoce con el nombre de \emph{distribución de Planck} para el gas de fotones.

El paso siguiente consiste en la determinación de los estados cuánticos accesibles a un fotón.
El estudio es completamente análogo al realizado en la sección 6.5.
Si admitimos que la longitud de onda de las ondas electromagnéticas es mucho menor que la dimensión lineal más pequeña del recinto, podremos despreciar los efectos de las paredes y escoger las condiciones en los límites más convenientes para nuestros cálculos.
Vamos entonces a considerar un paralelepípedo de dimensiones $L_x$, $L_y$ y $L_z$, y a tomar condiciones periódicas en los límites, o sea vamos a admitir que se cumple que
\begin{equation}
	\mathbf{E}(x, y, z) = \mathbf{E}(x + L_x, y, z) = \mathbf{E}(x, y + L_y, z) = \mathbf{E}(x, y, z + L_z)
\end{equation}

Al exigir que \eqref{eq:E_1_t8} verifique estas condiciones, resulta que han de satisfacerse las relaciones siguientes:
\begin{equation}
	\kappa_x L_x = 2\pi n_x \qquad | \qquad \kappa_y L_y = 2\pi n_y \qquad | \qquad \kappa_z L_z = 2\pi n_z
\end{equation}
siendo $n_x$, $n_y$ y $n_z$ enteros.
Consecuentemente, el número de ondas planas con polarización $\alpha$ y vector de onda comprendido entre $\kappa$ y $\kappa+d\kappa$, vendrá dado por
\begin{equation}
	D(\bm{\kappa}, \alpha) \dd[3]{\bm{\kappa}} = \left( \frac{L_x d\kappa_x}{2\pi}\right) \left( \frac{L_y d\kappa_y}{2\pi}\right) \left( \frac{L_z d\kappa_z}{2\pi}\right) = \frac{V}{(2\pi)^3} \dd[3]{\bm{\kappa}}
\end{equation}
y a partir de aquí obtenemos para el número de ondas planas con polarización $\alpha$ y vector de onda cuyo módulo esté comprendido entre $\kappa$ y $\kappa+d\kappa$,
\begin{equation}
	D(\kappa, \alpha) \dd{\kappa} = 4\pi \kappa^2 \dd{\kappa} D(\bm{\kappa}, \alpha) = \frac{V}{(2\pi)^2} \kappa^2 \dd{\kappa}
\end{equation}

Finalmente, si introducimos la frecuencia angular $\omega = \kappa c$ y multiplicamos por 2 para tener en cuenta las dos direcciones independientes de polarización de una onda plana, obtenemos para el número de estados cuánticos de fotón con frecuencia comprendida en el intervalo entre $\omega$ y $\omega + d\omega$ la expresión
\begin{equation}\label{eq:d_omega_t8}
	D(\omega) \dd{\omega} = \frac{V}{\pi^2}\frac{\omega^2 dd{\omega}}{c^3}
\end{equation}

A partir de \eqref{eq:n_r_BE} y \eqref{eq:d_omega_t8} se obtiene que el número medio de fotones en el intervalo de frecuencias considerado es
\begin{equation}\label{eq:f_omega_t8}
	f(\omega) \dd{\omega} = \expval{n_r}(\omega) D(\omega) \dd{\omega} = \frac{V}{\pi^2 c^3}\frac{\omega^2 \dd{\omega}}{e^{\hbar\omega / k_BT} - 1}
\end{equation}
y, en consecuencia, la distribución espectral de energía resulta finalmente
\begin{equation}\label{eq:E_med_2_t8}
	\expval{E} = \frac{V}{\pi^2 c^3}\frac{\omega^3 \dd{\omega}}{e^{\hbar\omega / k_BT} - 1}
\end{equation}
expresión que suele denominarse \emph{fórmula de Planck para la distribución espectral de energía de la radiación del cuerpo negro}.
A continuación vamos a estudiar, como siempre, el comportamiento de esta expresión en algunos casos límites.
\begin{enumerate}
	\item Frecuencias suficientemente bajas a una temperatura dada para que se cumpla que Podemos entonces aproximar con lo que resulta
	\begin{equation}
		\hbar\omega/k_B T \ll 1
	\end{equation}

	Podemos entonces aproximar con lo que resulta
	\begin{equation}
		e^{\hbar\omega / k_BT} \approx 1 + \frac{\hbar\omega}{k_B T}
	\end{equation}
	con lo que resulta
	\begin{equation}
		\expval{E} = \frac{V}{\pi^2 c^3} k_B T \omega^2 \dd{\omega}
	\end{equation}
	que coincide con la fórmula derivada clásicamente por Rayleigh y Jeans \eqref{eq:E_med_t8}.

	\item Frecuencias suficientemente altas a una temperatura dada para que sea
	\begin{equation}
		\hbar\omega/k_B T \gg 1
	\end{equation}

	Podemos entonces despreciar la unidad frente a $\exp[\hbar\omega / k_BT]$ en el denominador de \eqref{eq:E_med_2_t8}, con lo que esta expresión toma la forma
	\begin{equation}\label{eq:E_med_Wien_t8}
		\expval{E} = \frac{V}{\pi^2 c^3} \omega^3 e^{-\hbar\omega / k_BT} \dd{\omega}
	\end{equation}

	Es decir, que en el límite de altas frecuencias $\expval{E}(\omega)$ presenta un decrecimiento exponencial, de acuerdo con 108 resultados experimentales citados en el apartado anterior.
	La expresión \eqref{eq:E_med_Wien_t8} se suele denominar \emph{fórmula de Wien}.
\end{enumerate}

Volviendo a la expresión general \eqref{eq:E_med_2_t8}, vemos que a una temperatura dada la densidad espectral de energía es proporcional a la función
\begin{equation}
	\frac{\eta^3}{e^{\eta} - 1}
\end{equation}
siendo $\eta = \hbar\omega / k_BT$.

\begin{figure}[h]
	\centering
	% This file was created by matplotlib2tikz v0.6.17.
\begin{tikzpicture}

\definecolor{color0}{rgb}{0.12156862745098,0.466666666666667,0.705882352941177}
\definecolor{color1}{rgb}{1,0.498039215686275,0.0549019607843137}
\definecolor{color2}{rgb}{0.172549019607843,0.627450980392157,0.172549019607843}

\begin{axis}[
x = 4cm,
xmin=-40, xmax=840,
ymin=0, ymax=500,
tick align=outside,
tick pos=left,
x grid style={white!69.01960784313725!black},
y grid style={white!69.01960784313725!black}
]
\addplot [ultra thick, color0, forget plot]
table {%
0 0
13.5593220338983 0.55156564205688
27.1186440677966 2.20626256822752
40.6779661016949 4.96409077851192
54.2372881355932 8.82505027291008
67.7966101694915 13.789141051422
81.3559322033898 19.8563631140477
94.9152542372881 27.0267164607871
108.474576271186 35.3002010916403
122.033898305085 44.6768170066073
135.593220338983 55.156564205688
149.152542372881 66.7394426888825
162.71186440678 79.4254524561907
176.271186440678 93.2145935076127
189.830508474576 108.106865843149
203.389830508475 124.102269462798
216.949152542373 141.200804366561
230.508474576271 159.402470554438
244.067796610169 178.707268026429
257.627118644068 199.115196782534
271.186440677966 220.626256822752
284.745762711864 243.240448147084
298.305084745763 266.95777075553
311.864406779661 291.77822464809
325.423728813559 317.701809824763
338.983050847458 344.72852628555
352.542372881356 372.858374030451
366.101694915254 402.091353059466
379.661016949153 432.427463372594
393.220338983051 463.866704969836
406.779661016949 496.409077851192
420.338983050847 530.054582016662
433.898305084746 564.803217466245
447.457627118644 600.654984199943
461.016949152542 637.609882217753
474.576271186441 675.667911519678
488.135593220339 714.829072105717
501.694915254237 755.093363975869
515.254237288136 796.460787130135
528.813559322034 838.931341568515
542.372881355932 882.505027291008
555.932203389831 927.181844297616
569.491525423729 972.961792588337
583.050847457627 1019.84487216317
596.610169491525 1067.83108302212
610.169491525424 1116.92042516518
623.728813559322 1167.11289859236
637.28813559322 1218.40850330365
650.847457627119 1270.80723929905
664.406779661017 1324.30910657857
677.966101694915 1378.9141051422
691.525423728814 1434.62223498995
705.084745762712 1491.4334961218
718.64406779661 1549.34788853778
732.203389830508 1608.36541223786
745.762711864407 1668.48606722206
759.322033898305 1729.70985349038
772.881355932203 1792.0367710428
786.440677966102 1855.46681987935
800 1920
};
\addplot [ultra thick, color1, forget plot]
table {%
0 nan
13.5593220338983 0.515016170895403
27.1186440677966 1.92061287371626
40.6779661016949 4.02270771018934
54.2372881355932 6.64710054856547
67.7966101694915 9.6390214337599
81.3559322033898 12.8625129292946
94.9152542372881 16.1996614175019
108.474576271186 19.5496950254173
122.033898305085 22.8279683099723
135.593220338983 25.9648555607215
149.152542372881 28.9045755339857
162.71186440678 31.6039706300006
176.271186440678 34.0312630104502
189.830508474576 36.1648090032174
203.389830508475 37.9918714524858
216.949152542373 39.5074275586494
230.508474576271 40.7130273341994
244.067796610169 41.6157151991997
257.627118644068 42.2270245670597
271.186440677966 42.562052630614
284.745762711864 42.6386200374511
298.305084745763 42.4765178122567
311.864406779661 42.0968417947961
325.423728813559 41.5214130495399
338.983050847458 40.7722811848876
352.542372881356 39.8713063000566
366.101694915254 38.8398143475378
379.661016949153 37.6983200405907
393.220338983051 36.466311023699
406.779661016949 35.1620868298018
420.338983050847 33.8026461396582
433.898305084746 32.4036160035244
447.457627118644 30.9792169518806
461.016949152542 29.5422582807205
474.576271186441 28.1041582210417
488.135593220339 26.6749841680379
501.694915254237 25.2635086328374
515.254237288136 23.8772770716557
528.813559322034 22.5226842304244
542.372881355932 21.2050561068527
555.932203389831 19.9287350687753
569.491525423729 18.6971660722168
583.050847457627 17.5129822915969
596.610169491525 16.3780888063226
610.169491525424 15.2937432824522
623.728813559322 14.2606328459893
637.28813559322 13.2789465673139
650.847457627119 12.3484431664535
664.406779661017 11.4685137088924
677.966101694915 10.6382391941754
691.525423728814 9.85644304750714
705.084745762712 9.12173861071625
718.64406779661 8.43257179605232
732.203389830508 7.78725911691
745.762711864407 7.18402134611982
759.322033898305 6.62101307711084
772.881355932203 6.09634847802212
786.440677966102 5.60812353548901
800 5.15443508492152
};
\addplot [ultra thick, color2, forget plot]
table {%
0 0
13.5593220338983 0.0653052044440338
27.1186440677966 0.456194865551828
40.6779661016949 1.34442562868005
54.2372881355932 2.78269498445502
67.7966101694915 4.74578715890942
81.3559322033898 7.16085051289229
94.9152542372881 9.92927614646855
108.474576271186 12.9421384889036
122.033898305085 16.0907476060407
135.593220338983 19.2735319008934
149.152542372881 22.4002040826214
162.71186440678 25.3939505994012
176.271186440678 28.1922151962372
189.830508474576 30.7465126813449
203.389830508475 33.0216026551616
216.949152542373 34.9942693868677
230.508474576271 36.6518887352334
244.067796610169 37.9909123549288
257.627118644068 39.0153604407446
271.186440677966 39.7353845363168
284.745762711864 40.1659395269853
298.305084745763 40.3255872810655
311.864406779661 40.235442240494
325.423728813559 39.9182605817601
338.983050847458 39.3976685665215
352.542372881356 38.6975217390812
366.101694915254 37.8413841996992
379.661016949153 36.8521158913165
393.220338983051 35.7515553723829
406.779661016949 34.5602856694966
420.338983050847 33.2974713254847
433.898305084746 31.9807555405956
447.457627118644 30.6262072408555
461.016949152542 29.2483089200553
474.576271186441 27.8599771335433
488.135593220339 26.472608532892
501.694915254237 25.096145293372
515.254237288136 23.7391546835227
528.813559322034 22.4089183478501
542.372881355932 21.1115276151217
555.932203389831 19.8519818051328
569.491525423729 18.6342870882199
583.050847457627 17.4615539581204
596.610169491525 16.3360918151406
610.169491525424 15.2594995288088
623.728813559322 14.2327511633638
637.28813559322 13.2562763117154
650.847457627119 12.3300346998837
664.406779661017 11.4535849000674
677.966101694915 10.6261471317072
691.525423728814 9.84666024107108
705.084745762712 9.11383303541882
718.64406779661 8.42619021166472
732.203389830508 7.78211316518731
745.762711864407 7.17987599513086
759.322033898305 6.61767704092659
772.881355932203 6.09366629318148
786.440677966102 5.60596902256374
800 5.15270596458258
};
\end{axis}

\end{tikzpicture}
\end{figure}
Esta función, que se ha representado en la figura adyacente, presenta un máximo para $\eta = 2,8214$, lo que equivale a decir que la densidad espectral de energía presenta un máximo a una frecuencia $\widetilde{\omega}$ dada por
\begin{equation}
	\hbar\widetilde{\omega}/k_B T \approx 2.8214
\end{equation}

Por otro lado, si a una temperatura $T_1$ el máximo se presenta a un frecuencia $\widetilde{\omega}_1$ ya otra temperatura $T_2$ se presenta a $\widetilde{\omega}_2$ se cumplirá que
\begin{equation}
	\frac{\hbar\widetilde{\omega}_1}{k_B T_1} = \frac{\hbar\widetilde{\omega}_2}{k_B T_2}
\end{equation}
o sea
\begin{equation}
	\frac{\widetilde{\omega}_1}{T_1} = \frac{\widetilde{\omega}_2}{T_2}
\end{equation}
resultado que constituye la \emph{ley del desplazamiento de Wien} y que indica que al aumentar la temperatura $T$ la frecuencia $\widetilde{\omega}$ que hace máxima la distribución de Planck se desplaza hacia valores más altos.
Por ello, a veces también se conoce este fenómeno con el nombre de \emph{corrimiento hacía el azul}.

\section{Propiedades de la radiación del cuerpo negro: ley de Stefan – Boltzmann}

Calculemos ahora la función de partición generalizada \eqref{eq:Q_Fot} para lo cual sustituimos la suma respecto de todos los estados $r$ por una integral extendida a todas las frecuencias posibles, del producto de la densidad de estados $D(\omega)$ por la función $\ln [1 - \exp(-\beta\hbar\omega)]$.
Tenemos entonces
\begin{align}
	\ln Q_F &= -\frac{V}{\pi^2 c^3} \int_0^\infty \dd{\omega} \omega^2 \ln [1 - e^{-\beta\hbar\omega}] \nonumber \\
			&= -\frac{V}{\pi^2 c^3} \left( \frac{k_B T}{\hbar} \right)^3 \int_0^\infty \dd{\eta} \eta^2 \ln [1 - e^{-\eta}]
\end{align}
donde de nuevo hemos introducido $\eta = \hbar\omega / k_BT$.
Integrando por partes se encuentra
\begin{equation}
	\int_0^\infty \dd{\eta} \eta^2 \ln [1 - e^{-\eta}] = \eval{\frac{\eta^3}{3} - \ln [1 - e^{-\eta}]}_{0}^{\infty} - \frac{1}{3} \int_0^\infty \dd{\eta} \frac{\eta^3}{e^{\eta} - 1}
\end{equation}

El primer sumando del segundo miembro es nulo, mientras que la integral que aparece en el segundo se encuentra resuelta en el \hyperref[Anx5]{Anexo 5} y vale
\begin{equation}
	\int_0^\infty \dd{\eta} \frac{\eta^3}{e^{\eta} - 1} = \Gamma(4)\zeta(4) = \frac{\pi^4}{15}
\end{equation}

En resumen, resulta que
\begin{equation}
	\ln Q_F = \frac{\pi^2 V}{45 c^3 \hbar^3} (k_B T)^3 = \frac{\pi^2 V}{45 c^3 \hbar^3} \frac{1}{\beta^3}
\end{equation}

Una vez conocido $\ln Q_F$, las propiedades termodinámicas surgen por aplicación de las fórmulas usuales.
Así, para la energía media tenemos
\begin{equation}
	\expval{E} = - \left( \pdv{\ln Q_F}{\beta} \right)_V = \frac{1}{15} \frac{\pi^2 V}{45 c^3 \hbar^3} \frac{1}{\beta^4}
\end{equation}
que suele escribirse en la forma
\begin{equation}\label{eq:Ley_SB}
	\expval{E} =  \frac{4\sigma}{c} V T^4
\end{equation}
donde se ha introducido la constante
\begin{equation}
	\sigma = \frac{\pi^2 k_B^4}{60 c^2 \hbar^3} = 5.67 \cdot 10^{-8} \si{\watt\per\square\m\per\kelvin\tothe{4}}
\end{equation}

La expresión \eqref{eq:Ley_SB} es la \emph{ley de Stefan-Boltzmann}, correspondiente a la energía media de un gas de fotones en equilibrio a la temperatura $T$, y la constante $\sigma$ es la \emph{constante de Stefan-Boltzmann}.

Procediendo de manera análoga se encuentran sin dificultad
\begin{equation}
	S = k_B (\ln Q_F + \beta\expval{E}) = \frac{16}{3}\frac{\sigma}{c}TV^3 = \frac{4}{3}\frac{\expval{E}}{T}
\end{equation}
y
\begin{equation}
	\expval{p}V = k_B  T \ln Q_F = \frac{4}{3}\frac{\sigma}{c}TV^4 = \frac{1}{3} \expval{E}
\end{equation}
pudiéndose comprobar que $G = \expval{E} + \expval{p}V - TS = 0$ como debía ser, ya que $G = \expval{N}\mu$ y para un gas de fotones $\mu = 0$. 

Por último, vamos a calcular el número medio de- fotones que existen en el equilibrio.
El cálculo resulta más sencillo si en vez de la expresión general se utiliza la relación
\begin{equation}
	\expval{N} = \sum_r \expval{n_r} = \int_0^\infty \dd{\omega} f(\omega)
\end{equation}

En efecto, sustituyendo la expresión \eqref{eq:f_omega_t8} de $f(\omega)$, se encuentra
\begin{equation}
	\expval{N} = \frac{V}{\pi^2 c^3 } \int_0^\infty \frac{\dd{\omega} \omega^2}{e^{-\hbar\omega / k_BT} - 1} = \frac{V}{\pi^2 c^3 } \left( \frac{k_B T}{\hbar} \right)^3 \int_0^\infty \dd{\eta} \frac{\eta^2}{e^\eta - 1}
\end{equation}

La integral puede calcularse con el \hyperref[Anx5]{Anexo 5}, y vale
\begin{equation}
	\int_0^\infty \dd{\eta} \frac{\eta^2}{e^\eta - 1} = \Gamma(3)\zeta(3) \approx 2.404
\end{equation}
con lo cual obtenemos
\begin{equation}
	\expval{N} = \approx \frac{0.244}{c^3 } V \left( \frac{k_B T}{\hbar} \right)^3 
\end{equation}

Vemos entonces que al aumentar la temperatura $T$, la densidad de fotones $\sfrac{\expval{N}}{V}$ aumenta, y al disminuir $T$, disminuye.
En el límite $T \rightarrow 0$, también $\sfrac{\expval{N}}{V} \rightarrow 0$.
Este resultado indica que no existe condensación de Base, como sucedía en el caso de bosones con potencial químico distinto a cero y masa en reposo no nula.